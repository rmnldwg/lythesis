\providecommand{\relativeRoot}{../..}
\documentclass[\relativeRoot/main.tex]{subfiles}
\graphicspath{
	{\relativeRoot/figures/}
    {\subfix{./figures/}}
}


\begin{document}

\section{Documentation}
\label{sec:lyprox:documentation}

\begin{figure}
    \centering
    \includegraphics[width=\textwidth, frame]{figures/docs.png}
    \caption[
        Documentation for LyProX
    ]{
        Screenshot of a part of the \acrshort{api} documentation for LyProX. The displayed page documents the patient related model classes that represent SQLite3 database tables. It is auto-generated by \texttt{pydoctor} \cite{hudson_pydoctor_2022} using static code analysis.
    }
    \label{fig:lyprox:documentation}
\end{figure}

To improve the maintainability of the implementation, we automatically build a documentation from the docstrings in \inlinelyproxlogo{}' source code and make it available alongside the \faIcon{github}~GitHub repository.

Since a Django app cannot be executed or imported like a standalone Python module, dynamic auto-generation of documentation is cumbersome to set up and error-prone for Django-based projects. Hence, we resorted to a tool called \superhref{https://pydoctor.readthedocs.io/en/latest/}{\faIcon{external-link-alt}~\texttt{pydoctor}} \cite{hudson_pydoctor_2022} that statically analyses code for docstrings without importing or executing anything. Consequently, very little configuration is necessary to generate an informative static webpage containing information on the internal \acrshortpl{api}.

Like the deployment, building the documentation and hosting it is performed automatically -- using \superhref{https://github.com/features/actions}{\faIcon{github}~GitHub actions} -- whenever a new version is merged into the \texttt{main} branch of the development repository \repolink{lyprox}. Subsequenty, the built documentation that \texttt{pydoctor} outputs is hosted on \superhref{https://pages.github.com/}{\faIcon{external-link-alt}~GitHub pages} under the URL \faIcon{external-link-alt}~\url{rmnldwg.github.io/lyprox}. A screenshot of it is shown in \cref{fig:lyprox:documentation}.

At the time of writing this thesis, the documentation is in a basic state. It essentially only collects and nicely displays how we have documented the source code. However, in the future we intend to extend this substantially and make it a comprehensive guide, aiding potential collaborators in understanding how \inlinelyproxlogo{} is built and thus make contributing to it easier. Moreover, as the scope and complexity of the project grows, we ourselves might need to depend on a well-written documentation so that adding new features remains a smooth and time-efficient process.

\end{document}
