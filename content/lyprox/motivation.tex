\providecommand{\relativeRoot}{../..}
\documentclass[\relativeRoot/main.tex]{subfiles}
\graphicspath{{\subfix{./figures/}}}


\begin{document}

\section{Motivation}
\label{sec:lyprox:motivation}

Publishing and sharing the data of \cref{chap:dataset_usz} with the scientific community is a valuable contribution, since it allows other researchers to test a range of hypotheses. For example, one might be interested in how the involvement of upstream \glspl{lnl} influences the risk of nodal metastases in a given level. Something we have given an answer to in \cref{table:dataset_usz:upstream}. But also theories we did not think of at the time of writing our publications \cite{ludwig_detailed_2021,ludwig_dataset_2021} may be tested by someone who searches the literature for support in favor or against a particular hypothesis.

However, in such a case there are still some hurdles to discovering and taking advantage of freely available data: The researcher -- who we assume to have already conceived their hypothesis -- needs to understand what is reported in the dataset, what format it is provided in and how to implement a test using the found information. Our data is provided as a \gls{csv} table, meaning it can be opened in a spreadsheet program like Microsoft Excel. This would allow the user to create e.g. pivot tables and derived variables from observed ones, after they have made themselves familiar with the description and meaning of all the provided columns.

Taking these steps is not particularly difficult, and a determined scientist may be able to accomplish that in a matter of hours. However, if a researcher came up with a quick idea, not yet fully developed into a hypothesis, the outlined procedure could -- in their eyes -- very well not be worth the potential insight. Consequently, an interesting idea and a valuable cohort of patients might be left unexplored. Even more so with someone who did not even come up with a research question our data may be able to answer.

Figures and tables, like \cref{fig:dataset_usz:statistics} and \cref{table:dataset_usz:prevalences} that allow a reader of our work to quickly and visually understand our data address these issues to some extent. They create an understanding of what is reported in the data and answer common questions about it. But it is not feasible to provide plots and tables on all aspects of a dataset within a publication.

\subsection*{Prototype}
\label{subsec:lyprox:motivation:prototype}

This is why we thought of providing a dashboard, allowing a user to create the visualizations and numbers themselves according to their interest w.r.t. the underlying data. The first implementation that provided such a dashboard was a Python-based \gls{gui} developed by Bertrand Pouymayou for the use on local hardware (e.g. a laptop). It could display the prevalence of involvement for all \glspl{lnl} while the patients could be stratified by a number of reported variables that were recorded (e.g. smoking status). Moreover, one could explore how frequently certain levels were involved together or on their own, e.g. without metastases in upstream \glspl{lnl}. A screenshot of this first prototype is shown in \cref{fig:lyprox:pouymayou_gui}.

\begin{figure}
    \centering
    \includegraphics[width=1.0\textwidth]{figures/pouymayou_gui.png}
    \caption[
        Prototype of a GUI to explore patterns of lymphatic progression
    ]{
        Python-based \gls{gui} as developed by Bertrand Pouymayou to interactively explore our dataset containing patterns of lymphatic progression. The sidebar on the right allows stratification w.r.t. patient and tumor information, e.g. smoking status, lateralization of the primary tumor or T-stage. In the center, patients can be selected based in primary tumor subsites. The boxes named e.g. \texttt{III\_ipsi} allowed to (de)select patients with metastases in the respective \gls{lnl}.
    }
    \label{fig:lyprox:pouymayou_gui}
\end{figure}

The prototype already allowed us to quickly check correlation and patterns without having to manually go through large tables. Distributing this version of the \gls{gui} such that anyone could obtain a copy of it, install and subsequently run it, however, would have faced difficulties. Making \acrlongpl{gui} platform-independent is usually very error-prone and requires care w.r.t. to the installation and set up process. Moreover, not all potential users are necessarily familiar with the tools required. Hence, we decided to provide the interface such that we set up and control the environment it runs in.

\subsection*{Online interface}
\label{subsec:lyprox:motivation:online}

Arguably the most broadly accessible and most easily usable type of user interface comes in the form of \faIcon{html5}~\acrshort{html} documents. In conjunction with \faIcon{css3-alt}~\glspl{css}, they allow the creation of arbitrary content that can be displayed and interacted with using any modern web browser, regardless of the platform used. Additionally, both \acrshort{html} and \acrshort{css} files can be distributed efficiently over the internet. The logic driving the interface one wants to share can then be defined in a \faIcon{js-square}~JavaScript file that is sent alongside the \acrshort{html} and \acrshort{css} documents and executed on the client/user side. Alternatively, one may also run a central remote server that runs the logic in an arbitrary backend and sends out corresponding \acrshort{html} and \acrshort{css} responses.

\end{document}
