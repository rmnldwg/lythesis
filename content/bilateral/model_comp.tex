\providecommand{\relativeRoot}{../..}
\documentclass[\relativeRoot/main.tex]{subfiles}
\graphicspath{{\subfix{./figures/}}}


\begin{document}

\section{Comparing Bilateral Models}
\label{sec:bilateral:model_comp}

Up to this point we have largely argued that the mixing parameter makes intuitive sense because of the thought experiment, where we moved the primary tumor from a clearly lateralized position closer and closer to the mid-sagittal plane, until it was perfectly symmetric w.r.t. that plane. However, we now need to actually test whether our arguments hold. For that, we compare three models:

\begin{itemize}
    \item Model $\mathcal{M}_\text{ag}$, which is agnostic to the tumor's extension $\text{e}$ over the mid-sagittal plane and treats the contralateral base spread in the same way for all patients.
    \item Model $\mathcal{M}_\alpha$ that uses the linear combination of the ipsilateral base probabilities and the contralateral ones for the patients without mid-plane extension to describe the spread for tumors which do extend over that plane. Note that if we set $\alpha = 0$, instead of learning it from data, we would recreate the agnostic model $\mathcal{M}_\text{ag}$.
    \item Model $\mathcal{M}_\text{full}$, going even further by defining a completely independent set of contralateral base probabilities for the patients whose tumor extends over the mid-sagittal plane.
\end{itemize}

We now want to know which of these three models best describes the data. Intuitively, one would argue that it must be $\mathcal{M}_\text{full}$, but this model is also more complex than the other two. A natural choice for a metric that incorporates both the accuracy of the model and a penalty for model complexity -- often also called \emph{Occam's razor} -- is the \emph{model evidence} \cite{aponte_introduction_2022}, which we introduced in \cref{sec:graph:model_comp}.

\subsection{Implementation}
\label{subsec:bilateral:model_comp:implementation}

\begin{tcolorbox}[
    title=\faIcon{recycle} Reproducibility,
    parbox=false,
    float
]
    All results of the section below are reproducible using a notebook we have published within the \repolink{lymph} repository at the time of submitting \cite{ludwig_hidden_2021}. The state of the repository at that time was just recently updated to make the notebook's reproduction a bit more user-friendly:

    \begin{enumerate}
        \item Clone the \repolink{lymph} repository and checkout the tagged version \href{https://github.com/rmnldwg/lymph/tree/0.1.1}{\faIcon{tags} 0.1.1}:
        \begin{itemize}[leftmargin=10mm]
            \setlength\itemsep{-0.5em}
            \item[\texttt{\$}] \verb|git clone https://github.com/rmnldwg/lymph|
            \item[\texttt{\$}] \verb|cd lymph|
            \item[\texttt{\$}] \verb|git checkout 0.1.1|
        \end{itemize}

        \item Set up a virtual environment that supports installing packages using \href{https://pypi.org/project/pip/}{\texttt{pip}}. I recommend \href{https://docs.python.org/3/library/venv.html}{\texttt{venv}}:
        \begin{itemize}[leftmargin=10mm]
            \setlength\itemsep{-0.5em}
            \item[\texttt{\$}] \verb|python3.8 -m venv .venv|
            \item[\texttt{\$}] \verb|source .venv/bin/activate|
            \item[\texttt{\$}] \verb|pip install -U pip setuptools wheel|
        \end{itemize}

        \item Install the necessary packages:
        \begin{itemize}[leftmargin=10mm]
            \setlength\itemsep{-0.5em}
            \item[\texttt{\$}] \verb|pip install -r requirements.txt|
        \end{itemize}

        \item Open the Jupyter notebook:
        \begin{itemize}[leftmargin=10mm]
            \setlength\itemsep{-0.5em}
            \item[\texttt{\$}] \verb|cd notebook|
            \item[\texttt{\$}] \verb|jupyter notebook results_and_plots.ipynb|
        \end{itemize}
    \end{enumerate}

    Now, it should be possible to execute the cells of the notebook in order. The inference precesses will likely run for a while, but should continuously report its progress. Subsequent cells will produce the figures shown in this chapter and store them inside the \faIcon{folder} \texttt{figures} folder relative to the notebook.
\end{tcolorbox}


\begin{tcolorbox}[title=\faIcon{database} Data, parbox=false]
    For the following experiments, we combined two datasets:

    \begin{itemize}[leftmargin=7mm]
        \item[\faIcon{hospital}] \textbf{USZ}: 287 patients with newly diagnosed \gls{opscc}, treated at the \acrlong{usz} between 2013 and 2019. This dataset has been described in great detail in \cref{chap:dataset_usz}.
        \item[\faIcon{hospital}] \textbf{CLB}: 263 patients with newly diagnosed \gls{opscc}, treated at the \acrlong{clb} between 2014 and 2018. A table containing this data was kindly provided to us by \textbf{Vincent Grégoire}, who was also the \gls{pi} of the study during which this data was extracted and for which it was analyzed \cite{bauwens_prevalence_2021}.
    \end{itemize}

    Both datasets are available in our repository \repolink{lyDATA} or in the online interface \href{https://lyprox.org/patients/dataset/}{\faIcon{external-link-alt} LyProX}.

    All available diagnostic and pathological modalities reported in the data were used in combination with literature values for their respective sensitivity and specificity \cite{de_bondt_detection_2007} to arrive at an estimate for the most likely true pattern of involvement for each patient.

    This maximum likelihood estimate was then fed into the model, assuming a sensitivity and specificity of 1. See \cref{chap:dataset_clb} and more specifically \cref{subsec:dataset_clb:results:implications} for an intuition into our motivations for this simplification.
\end{tcolorbox}
}

To compare the models $\mathcal{M}_\text{ag}$, $\mathcal{M}_\alpha$ and $\mathcal{M}_\text{full}$, we performed \gls{ti} with a ladder of 64 $\beta$ values with step sizes selected according to a fifth order power rule. For each of the steps in the ladder, we performed an ensemble sampling round using the \texttt{emcee} \cite{foreman-mackey_emcee_2013} Python package. The ensemble consisted of parallel ``walkers'' that allow sampling in parallel and mutually influence each other's proposals. We composed the ensemble of $20 \cdot k$ parallel ``walkers'', where $k$ is the number of dimension in the model's parameter space. We set the sampling algorithm to propose new samples according to a mixture of two procedures: with 80\% probability it selected a differential evolution move \cite{nelson_run_2013} and with 20\% probability a snooker move, also based on differential evolution \cite{ter_braak_differential_2008}. The reason for this choice was that in previous experiments, this combination of proposals yielded the fastest convergence of the chain. Every one of the 64 sampling rounds consisted of a burn-in phase lasting 1000 steps, followed by 250 steps of which every fifth step was kept for later analysis. This might seem like a relatively short chain, but since the change of the posterior we sampled from only changed very slightly from $\beta_j$ to $\beta_{j+1}$, fewer steps are required to reach convergence.

In the end, we kept $S = 50 \cdot 20 \cdot k$, where $k$ is the dimensionality of the model, samples for each of the 64 $\beta_j$. The dimensionality $k$ of the parameter spaces ranged from nine for the agnostic model $\mathcal{M}_\text{ag}$ over ten (mixing model $\mathcal{M}_\alpha$) to twelve in the case of the full model $\mathcal{M}_\text{full}$. Out of these $S$ samples we randomly drew $M = 1000$ per $\beta_j$ and used them to integrate $\mathcal{A}_\text{MC}(\beta)$ 1000 times over $\beta$, yielding 1000 estimates for the log-evidence $\ln{E}_l$ with $l \in [1, \ldots, M]$. Using this ensemble of estimates, we were able to compute both the mean and the standard deviation, giving us a simple measure of uncertainty for its value. We also computed the \gls{bic} from the samples drawn at $\beta_{64} = 1$, as introduced in \cref{subsec:graph:model_comp:bic}.

We restricted ourselves to the bilateral modelling of the \glspl{lnl} II, III and IV, because contralaterally we rarely observe involvement outside those levels, and it drastically speeds up the inference process, when fewer \glspl{lnl} are considered.

\subsection{Results}
\label{subsec:bilateral:model_comp:results}

\begin{figure}
    \def\svgwidth{1.0\textwidth}
    \input{figures/ipsi-overall.pdf_tex}
    \caption[
        Comparison of ipsilateral prevalences
    ]{
        Predicted prevalences (shaded histograms) and posterior beta distributions of observed prevalences (solid lines) for the ipsilateral levels II (blue), III (orange) and IV (green). These prevalences have each been plotted for early T-category patients (left column) and late T-category (right column) and for the three models $\mathcal{M}_\text{ag}$ (top row), $\mathcal{M}_\alpha$ (middle row) and for $\mathcal{M}_\text{full}$ (bottom row).
    }
    \label{fig:bilateral:model_comp:ipsi}
\end{figure}

First, we verify that all three models still describe the ipsilateral spread sufficiently well. We plotted the prevalence our trained models predict in the forms of histograms against the Beta-posterior (see \cref{box:dataset_clb:results:beta}) of the observed prevalence in the data (\cref{fig:bilateral:model_comp:ipsi}). These plots were created by computing the respective prevalence for samples drawn during the final 250 steps at the end of the \gls{ti} process of which every fifth step was discarded.

The shown differences between the model's predictions are negligible. For late T-categories (bottom row of \cref{fig:bilateral:model_comp:ipsi}) it seems as if the model that is agnostic to the tumor's extension over the mid-sagittal plane slightly overestimates the prevalence, while the other two models seem to match them better or underestimate them by a small amount. Overall the fit of all models ipsilaterally is very good and shows no indication that one model performs better than the other.

\begin{figure}
    \def\svgwidth{1.0\textwidth}
    \input{figures/contra-overall.pdf_tex}
    \caption[
        Comparison of contralateral prevalences
    ]{
        Predicted prevalences (shaded histograms) and posterior beta distributions of observed prevalences (solid lines) for the contralateral overall involvement (anything \emph{not} N0, on that side of the neck). Predicted and observed prevalences are stratified by T-category and mid-plane extension: Early T-category prevalences are shown in blue and orange (and labelled \texttt{early}), late T-category in green and red (labelled \texttt{late}). Without mid-sagittal extension (labelled \texttt{noext}) the plots are colored blue and green, whereas with the extension (label \texttt{ext}), they are colored orange and red. The three models $\mathcal{M}_\text{ag}$, $\mathcal{M}_\alpha$ and $\mathcal{M}_\text{full}$ are depicted in the left, middle and right panel respectively.
    }
    \label{fig:bilateral:model_comp:contra}
\end{figure}

On the contralateral side, however, this does not hold anymore. Here, we do not only stratify the prevalence by T-category, but also by midline extension. Naturally, this cannot be captured by the agnostic model $\mathcal{M}_\text{ag}$ since it lacks the ability of modelling this. What is of interest to us here is how the mixing model $\mathcal{M}_\alpha$ and the full model $\mathcal{M}_\text{full}$ compare against each other and whether their improvements in predicting contralateral spread are worth the additional complexity.

The overall prevalence of contralateral involvement is plotted in \cref{fig:bilateral:model_comp:contra} for the three different models and we have distinguished between four cases for each model: The prevalence of any contralateral involvement for patients with
\begin{enumerate*}[label={(\arabic*)}]
    \item early T-category and a clearly lateralized tumor (blue histograms and curves),
    \item early T-category with a tumor extending over the mid-sagittal plane (orange),
    \item late T-category with, a lateralized tumor (green) and finally
    \item where the tumor is both in late T-category and does extend over the mid-plane (red).
\end{enumerate*}

As discussed, the agnostic model $\mathcal{M}_\text{ag}$ (left panel in \cref{fig:bilateral:model_comp:contra}) cannot model midline extension, and hence it must predict the same prevalence for the same scenario, regardless of the patient's mid-sagittal extension. Its spread probability rates from the tumor to the contralateral \glspl{lnl} can only attempt to find an average of the respective observed prevalence.

\begin{figure}
    \centering
    \def\svgwidth{1.0\textwidth}
    \input{figures/contra-correlations-ext.pdf_tex}
    \caption[
        Correlations between involvement of the contralateral LNLs II and III
    ]{
        Histograms over predicted prevalences compared to Beta posteriors over observed prevalences respectively for three scenarios where the tumor extends over the mid-sagittal plane:
        \begin{enumerate*}[label={(\arabic*)}]
            \item The case $\lnot X_\text{II} \land X_\text{III}$ in blue,
            \item co-involved levels $X_\text{II} \land X_\text{III}$ in green and lastly,
            \item $X_\text{II} \land \lnot X_\text{III}$ in orange
        \end{enumerate*}.
        All scenarios are plotted for each of the three models in the respective row, and for early and late T-category in the corresponding column. 
    }
    \label{fig:bilateral:model_comp:contra-correlations-ext}
\end{figure}

Of the four investigated scenarios, three are described very well by our model. Interestingly, both the model using the mixing parameter $\alpha$ and the full model, which has in total six parameters to model the spread from the tumor to the contralateral \glspl{lnl}, overestimate the prevalence of contralateral, early T-category involvement when the tumor extends over the mid-sagittal line. This might be because, of the displayed cases, this is the rarest one, so it makes sense for the model to put less attention to it. On the other hand, we believe the reason also relates to how the model treats the mid-plane extension in general: If this binary \gls{rv} is observed to be true, the model assumes increased spread probabilities from the primary tumor to the contralateral \glspl{lnl} from the onset of the disease (i.e., from time-step 0). Realistically, however, this is probably not how a typical patient's disease evolves. As tumors grow over time, in many cases they will not cross the mid-sagittal plane right after they started to form, but only when they are sufficiently large, as indicated by our data as well: Of the early T-category tumors in the dataset, only 7\% (11 of 150) cross the mid-sagittal plane, while of the late T-category tumors 58\% (79 of 137) have grown over into the contralateral half of the patient. Consequently, the spread from them to the contralateral side increases during the tumor's growth. The current model cannot capture this and therefore assumes early T-category patient's contralateral spread to be larger than observed.

\begin{table}
    \centering
    \begin{tabular}{|l|l|l|l|l|l|}
        \hline
        \textbf{Model} & \textbf{-- log-evidence} & \textbf{BIC/2} & \textbf{-- max. llh} & \textbf{-- mean llh} \\
        \hline
        $\mathcal{M}_\text{ag}$ & 1118.2 $\pm$ 1.8 & 1116.7 & 1088.3 & 1092.3 \\
        $\mathcal{M}_\alpha$ & 1093.3 $\pm$ 1.9 & 1093.1 & 1061.5 & 1065.7 \\
        $\mathcal{M}_\text{full}$ & 1099.7 $\pm$ 2.0 & 1098.2 & 1060.4 & 1065.5 \\
        \hline
    \end{tabular}
    \caption[
        Metrics for assessing three bilateral models
    ]{
        Metrics computed via \gls{ti} for the three bilateral models introduced in \cref{sec:bilateral:model_comp}: The negative log-evidence $\ln{E}$ with the respective standard deviation in the first column, one half of the \gls{bic} in the second column, as well as the negative of the maximum and mean likelihood of the sampling procedures in columns three and four respectively. Note that the smaller the values given here, the better the performance of the respective model.
    }
    \label{table:bilateral:model_comp:evidences}
\end{table}

Despite this, the two models perform equally well regarding the overall contralateral spread. In combination with the unaffected capabilities to predict the ipsilateral prevalences, as shown in \cref{fig:bilateral:model_comp:ipsi}, this indicates that the assumptions underlying the introduction of the mixing parameter $\alpha$ are feasible.

\begin{figure}
    \centering
    % \def\svgwidth{1.0\textwidth}
    \input{figures/thermo_int.pdf_tex}
    \caption[Plotted accuracy over the course of a thermodynamic integration round]{
        Expectation of the log-likelihood under the power posterior (\cref{eq:ti:power_post}) plotted against 64 inverse temperature steps $\beta$ for the three models. The accuracy on the y-axis is plotted on a log-scale, while the $\beta$ values, which themselves represent a fifth order annealing schedule, were plotted on an x-axis where the ticks were spaced according to a seventh order power rule. This was done to nicely visualize both the differences in accuracy and to better depict at which $\beta$ values the accuracies begin to rise.
    }
    \label{fig:bilateral:model_comp:thermo_int}
\end{figure}

One risk of modelling the correlations between involvements of the contralateral \glspl{lnl} is that some predictions might suffer from the simplifying introduction of the mixing parameter $\alpha$. To convince ourselves that this simplification is not too restrictive, we have compared the three models in their predictions for scenarios where we correlated the involvement of \gls{lnl} II and III in \cref{fig:bilateral:model_comp:contra-correlations-ext}. What we find in these plots is that the full model $\mathcal{M}_\text{full}$ is indeed slightly better at predicting e.g. how often \gls{lnl} II is involved contralaterally without \gls{lnl} III in the case of late T-category patients whose tumors extend over the mid-sagittal plane. However, the improvement is minor, compared to how much better most predictions become when switching from the agnostic model $\mathcal{M}_\text{ag}$ in the top row to the mixing model in the second row of \cref{fig:bilateral:model_comp:contra-correlations-ext}.

Also, from a clinical perspective this is currently of less importance than estimating the overall contralateral involvement. The recent debate centers around whether to treat or to spare the contralateral side as a whole when performing elective \gls{rt} or elective bilateral \gls{nd} \cite{biau_selection_2019,al-mamgani_contralateral_2017}. Sparing individual \glspl{lnl} is not yet discussed. Therefore, a more complete model like $\mathcal{M}_\text{full}$, that can capture the mentioned correlations within the contralateral neck slightly better, are not worth the additional model complexity at this point.

A comparison of the log-evidence of the three models -- which we provide in \cref{table:bilateral:model_comp:evidences} -- supports our initial hypothesis that the mixing model $\mathcal{M}_\alpha$ performs well, even compared to the full model $\mathcal{M}_\text{full}$, while being significantly less complex. Besides \emph{decisive} evidence against the agnostic model (according to \cref{table:bayes_factor}), we also see that the accuracy $\mathcal{A}_\text{MC}(\beta=1)$ is almost the same for the mixing model $\mathcal{M}_\alpha$ and the full model $\mathcal{M}_\text{full}$. But the higher complexity from introducing two additional parameters is punished heavily both by the thermodynamically integrated log-evidence, and its approximation, the \gls{bic}. Therefore, from a Bayesian standpoint, we find \emph{decisive} evidence in favor of the mixing model $\mathcal{M}_\alpha$ with $\ln{K}_{\alpha\,\text{v}\,\text{full}} = 6.4$. Note that the \gls{bic} is a very good approximation of the log-evidence for all three models compared here.

\begin{figure}
    \centering
    \input{figures/contra-risks.pdf_tex}
    \caption[
        Risk for involvement in LNL II for different diagnoses
    ]{
        Histograms showing how the risk to find microscopic metastases in \gls{lnl} II contralaterally increases as the given diagnosis worsens: From left to right, the provided diagnosis to compute said risk was
        \begin{enumerate*}[label={(\arabic*)}]
            \item an unsuspicious \gls{ct} scan (clinically $N_0$) of all levels in both sides of the neck for an early T-category patient whose tumor is clearly lateralized (green shaded),
            \item the same as before, but for a late T-category patients (blue hatched),
            \item clinically $N_0$, late T-category but now with a tumor extending over the mid-sagittal plane (blue shaded),
            \item all \glspl{lnl} appear negative on \gls{ct}, except for a positive finding in the ipsilateral \gls{lnl} II, with a late T-category patient having a mid-sagittal extension (orange shaded), and lastly
            \item the same diagnosis, except that the ipsilateral \gls{lnl} is diagnosed with being metastatic as well (red hatched)
        \end{enumerate*}.
        The sensitivity and specificity were set to 81\% and 76\% respectively, following de Bondt et al. \cite{de_bondt_detection_2007}.
    }
    \label{fig:bilateral:model_comp:risks}
\end{figure}

\Cref{fig:bilateral:model_comp:thermo_int} displays how -- using \gls{ti} -- the expected log-likelihood under the power posterior evolves over the course of a \gls{ti} round for each of the three compared models. It shows that the final accuracy $\mathcal{A}_\text{MC}(\beta=1)$ of the agnostic model $\mathcal{M}_\text{ag}$ is lower than of the other two models, which owe that to their ability to incorporate the tumor's extension over the mid plane into the prediction. However, while the mixing model $\mathcal{M}_\alpha$ and the full model $\mathcal{M}_\text{full}$ achieve the same fit to the data, the full model's accuracy rises for later $\beta$ values, which results in a lower log-evidence and a higher complexity penalty (see \cref{eq:ti:acc_vs_kl}) compared to the simpler mixing model.

\subsection{Contralateral Risk Depending on Ipsilateral Diagnosis}
\label{subsec:bilateral:model_comp:risk}

To conclude this section, we show that the extended bilateral model that uses the mixing parameter $\alpha$ to model the contralateral spread in case of a tumor that extends over the mid-sagittal plane, can factor in ipsilateral diagnoses into its risk prediction for contralateral involvement. To that end, we have plotted in \cref{fig:bilateral:model_comp:risks} how the risk for microscopic involvement in the contralateral \gls{lnl} II increases for a worsening diagnosis. We have started with an early T-category patient with a clearly lateralized tumor whose \gls{ct} scan showed no detectable metastases in either side of the neck. For that case, the risk for micrometastases is very low with below 2\%. This risk increases just slightly if we were to observe the same diagnosis but for a late T-category patient. As expected, the risk jumps to over 7\% as soon as the tumor touches or grows over the mid-sagittal plane, where the patient is still of late T-category and clinically $N_0$. The contralateral risk for microscopic involvement in \gls{lnl} II increases further to around 9\% when a metastasis is detected via \gls{ct} in the ipsilateral \gls{lnl} II and once more to 10\% when level III on the same side shows involvement as well.

Note that all these computed risks are conditioned on the fact that the \gls{ct} scan observed a node-negative contralateral neck. Hence, these risks cannot easily be compared to the prevalences of involvement as reported by data. But they illustrate how our probabilistic \acrshort{hmm}-based model is capable of naturally factoring a number of clinically relevant variables into its risk prediction and how these may be used by clinicians as supporting estimates flowing into a more objective and quantitative decision on a treatment plan.

\end{document}
