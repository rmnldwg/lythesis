\providecommand{\relativeRoot}{../..}
\documentclass[\relativeRoot/main.tex]{subfiles}
\graphicspath{
    {\subfix{./figures/}}
}


\begin{document}

\chapter{Bilateral hidden Markov model}
\label{chap:bilateral}

In the previous chapter we have set up the formalism to deal with only one side of the neck. Implicitly, we have assumed that to be the ipsilateral side, i.e. the side of the sagittal plane where the primary tumor is located. This is because we assume lymphatic drainage to be a process that is somewhat symmetric w.r.t. the sagittal plane, which means there can only be limited lymph flow across this plane. But depending on the tumor's location and lateralization, drainage and hence metastatic spread to the contralateral lymphatic system of the neck may also occur. In current clinical practice, a bilateral neck dissection or irradiation is often prescribed when the tumor is close to the mid-sagittal plane. So, ideally, we would like to model the risk for involvement in both sides of the neck at the same time.

The formalism of \cref{chap:unilateral} can easily be applied to the contralateral side and given respective training data for the sampling process, it would learn the appropriate spread probabilities to and among the contralateral \glspl{lnl} just as it would learn the ones for the ipsilateral side. From clinical experience, the contralateral involvement is usually less severe than the ipsilateral one, and hence we would expect the contralateral spread to be less probable as well. This is also supported by \cref{table:dataset:prevalence} shown in \cref{chap:dataset}: 81\% (232 of 287) of patients were diagnosed with metastases in the ipsilateral \gls{lnl} II, while only 18\% (51 of 287) appeared to harbor metastases in the same level contralaterally. A similar pattern can also be observed of the other reported \glspl{lnl}.

However, combining two unilateral models naively would make the assumption that ipsi- and contralateral spread are independent, which is disputed by the data: \cref{table:dataset:contralateral} shows the incidence of contralateral metastases being much higher for patients with metastases in the ipsilateral \gls{lnl} III. E.g., patients with a clearly lateralized early T-stage tumor show contralateral metastases only in 4\% of the cases (4 out of 94), while this number rises to 18\% (8 out of 45) when level III on the ipsilateral side harbored metastases. Therefore, if we find a patient has advanced metastases in the contralateral neck nodes, the risk to find similarly or even more advanced disease in ipsilateral neck nodes should be higher than if the contralateral neck were healthy. In other words, the joint probability $P \left( \mathbf{X}^\text{i}, \mathbf{X}^\text{c} \mid \mathbf{Z}^\text{i}, \mathbf{Z}^\text{c} \right)$ does not simply factorize into an ipsi- and a contralateral term. The superscripts $\text{i}$ and $\text{c}$ indicate the respective side of the neck.

The following section will pick up the unilateral formalism, extend and modify it to come up with a less naive bilateral model.

\subfile{expand}
\subfile{parameter_symmetries}
\subfile{model_comp}

\end{document}