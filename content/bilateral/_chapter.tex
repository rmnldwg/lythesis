\providecommand{\relativeRoot}{../..}
\documentclass[\relativeRoot/main.tex]{subfiles}
\graphicspath{
	{\relativeRoot/figures/}
    {\subfix{./figures/}}
}


\begin{document}

\chapter{Extending the HMM to the Contralateral Neck}
\label{chap:bilateral}
\global\togglefalse{lyproxIsUsed}

In the previous chapter we have set up the formalism to deal with only one side of the neck. Implicitly, we have assumed that to be the ipsilateral side, i.e. the side of the sagittal plane where the primary tumor is located. Depending on the tumor's location and lateralization, drainage and hence metastatic spread to the contralateral lymphatic system of the neck also occurs. In current clinical practice, a bilateral neck dissection or irradiation is recommended in the vast majority of cases \cite{de_veij_mestdagh_incidence_2019}, although a growing number of researchers believe treatment may be safely limited to the ipsilateral side for many patients \cite{rackley_unilateral_2017,de_veij_mestdagh_spectct-guided_2020}. A probabilistic model like ours could support the selection of patients in whom the contralateral neck irradiation is reduced or even omitted, if it can be extended successfully to the contralateral side.

The formalism of \cref{chap:unilateral} can in principle easily be applied to the contralateral side and given respective training data for the sampling process, it would learn the appropriate spread probabilities to and among the contralateral \glspl{lnl} just as it learned the ones for the ipsilateral side. From clinical experience, contralateral involvement is less frequent than ipsilateral lymph node metastasis, which is also supported by the \gls{usz} data (see \cref{table:dataset_usz:prevalence}) shown in \cref{chap:dataset_usz}: 81\% (232 of 287) of patients were diagnosed with metastases in the ipsilateral \gls{lnl} II, whereas only 18\% (51 of 287) harbor metastases in the contralateral level II. A similar pattern can also be observed for the other \glspl{lnl}. Consequently, the base probability rates $\tilde{b}^\text{c}_v$ for spread of the primary tumor to the contralateral \glspl{lnl} are expected to be lower than for the ipsilateral side.

However, combining two unilateral models naively would make the assumption that ipsi- and contralateral spread are independent, which is disputed by the \gls{usz} data: \Cref{table:dataset_usz:contralateral} shows the incidence of contralateral metastases being significantly higher for patients with metastases in the ipsilateral \gls{lnl} III. E.g., patients with a clearly lateralized early T-category tumor, without metastasis in \gls{lnl} III ipsilaterally, show contralateral metastasis only in 4\% of the cases (4 out of 94). This number rises to 18\% (8 out of 45) when level III on the ipsilateral side harbors metastases. Therefore, if a patient has advanced metastatic disease in the ipsiateral neck nodes, the risk to find contralateral neck node metastasis should be higher than for N0 patients or if involvement is limited to level II, ipsilaterally. In other words, the joint probability $P \left( \mathbf{X}^\text{i}, \mathbf{X}^\text{c} \mid \mathbf{Z}^\text{i}, \mathbf{Z}^\text{c} \right)$ does not simply factorize into an ipsi- and a contralateral term. The superscripts $\text{i}$ and $\text{c}$ here indicate the ipsi- and contralateral side of the neck, repsectively.

The following section will start with the unilateral formalism, extend and modify it to develop a bilateral model that can capture the correlations between the involvement in the two sides of the neck.

\subfile{expand}
\subfile{parameter_symmetries}
\subfile{model_comp}

\end{document}