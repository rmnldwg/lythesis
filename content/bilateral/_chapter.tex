\providecommand{\relativeRoot}{../..}
\documentclass[\relativeRoot/main.tex]{subfiles}
\graphicspath{{\relativeRoot/figures/}}

\begin{document}

\chapter{Bilateral hidden Markov model}
\label{chap:bilateral}

In the previous chapter we have set up the formalism to deal with only one side of the neck. Implicitly, we have assumed that to be the ipsilateral side, i.e. the side of the sagittal plane where the primary tumor is located. This is because we assume lymphatic drainage to a process that is somewhat symmetric w.r.t. the sagittal plane, which means there can only be limited lymph flow across this plane. But depending on the tumor's location and lateralization, drainage and hence metastatic spread to the contralateral lymphatic system of the neck may also occur. In current clinical practice, a bilateral neck dissection or irradiation is often prescribed when the tumor is close to the mid-sagittal plane. So, ideally, we would like to model the risk for involvement in both sides of the neck at the same time.

The formalism of \cref{chap:unilateral} can easily be applied to the contralateral side and given respective training data for the sampling process, it would learn the appropriate spread probabilities to and among the contralateral \glspl{lnl} just as it would learn the ones for the ipsilateral side. From clinical experience, the contralateral involvement is usually less severe than the ipsilateral one, and hence we would expect the contralateral spread to be less probable as well. 

However, combining two such unilateral models naively would make the assumption that ipsi- and contralateral spread are independent, which seems unlikely: If we know a patient has advanced metastases in the contralateral neck nodes, the risk to find similarly or even more advanced disease in ipsilateral neck nodes should probably be higher than if the contralateral neck were healthy. In other words, we are now looking for the joint probability $P \left( \mathbf{X}^\text{i}, \mathbf{X}^\text{c} \mid \mathbf{Z}^\text{i}, \mathbf{Z}^\text{c} \right)$, where the superscripts $\text{i}$ and $\text{c}$ indicate the ipsi- and contralateral side respectively.

The following section will pick up the unilateral formalism, extend and modify it to come up with a less naive bilateral model.

\subfile{expand}
\subfile{parameter_symmetries}

\end{document}