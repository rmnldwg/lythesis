\providecommand{\relativeRoot}{../..}
\documentclass[\relativeRoot/main.tex]{subfiles}
\graphicspath{{\subfix{./figures/}}}


\begin{document}

\chapter{Discussion and Outlook}
\label{chap:discussion}

Over the course of this thesis we have worked on three related aspects to address the problem of poorly quantified risks of occult disease in \gls{hnscc}. During the first part of the thesis, in \cref{chap:dataset_usz,chap:dataset_clb} we have described two datasets of \gls{opscc} patients, one extracted by our group at the \gls{usz}, and one extracted by \citeauthorandlink{bauwens_prevalence_2021} at the \gls{clb} and provided to us through a collaboration. Both datasets detail lymphatic metastatic progression patterns of patients in unprecedented detail. This includes information on which \glspl{lnl} were involved simultaneously in a patient, which is a requirement for a more personal estimation of the risk for microscopic metastases. Because the correlation between involvements of different \glspl{lnl} is necessary to infer how the risk for occult disease changes when different clinical involvements are diagnosed.

Through our effort, the \gls{usz} and \gls{clb} datasets were also made available to the public in their raw format (\acrshort{csv} tables in the repository \repolink{lyDATA}). And as such may be used by other researchers to improve the prediction of microscopic involvement in \gls{hnscc}. Hence, our work on detailed progression pattern datasets is already a significant contribution to the task of \gls{ctv-n} definition that depends on accurate estimates of risks for occult disease.

Moreover, making the data freely accessible may motivate other researchers to share similar cohort information as well. It is reasonable to assume that underlying the many publications that report prevalences of \gls{lnl} involvement \cite{candela_patterns_1990,shah_patterns_1990,woolgar_histological_1999,woolgar_topography_2007,chao_determination_2002,vauterin_patterns_2006,razfar_incidence_2009,ho_patterns_2012,bauwens_prevalence_2021} is a dataset of detailed metastatic progression patterns. The \gls{gui} \href{https://lyprox.org}{\faIcon{external-link-alt}~LyProX} that we developed and explained in \cref{chap:lyprox} may add to this motivation, because it demonstrates the value of the extracted data visually. But mainly, it enables researchers to quickly investigate correlations between \gls{lnl} involvements and other clinicopathological reported variables. Its functionality at this point does not yet allow its user to obtain a quantitative estimate for the risk of metastases in a \gls{lnl}. But it can already inform clinicians about which levels are most frequently involved together or how severe the impact of risk factors on certain involvement patterns is. In addition, \href{https://lyprox.org}{\faIcon{external-link-alt}~LyProX} serves as another way to distribute the underlying datasets and hence improves the data's accessibility. Therefore, our \gls{gui} complements our contributions regarding lymphatic metastatic progression data towards more personalized \gls{ctv-n} definition.

In parallel to the two aspects mentioned above, we developed an intuitive \gls{hmm} for predicting regional metastasis (see \cref{chap:unilateral}). The model represents the presumed underlying mechanics of lymphatic spread and consists of relatively few parameters that are highly interpretable. In the \cref{chap:graph} we have extended the model to cover most \glspl{lnl} in the neck region, namely the levels I, II, III, IV, V, and VII. After that, in \cref{chap:bilateral} modelling the contralateral spread as well was incorporated into the model. Throughout the three \cref{chap:unilateral,chap:graph,chap:bilateral} we have showed how accurate the model's fit to the data is and in \cref{chap:complete} we demonstrated how it may be used to guide volume de-escalated radiotherapy in \gls{hnscc} patients.

While testing our \gls{hmm} with retrospective data is an important part of validating the model, at some point in the future, we would like to validate it in a clinical trial. Currently, there already exists an ongoing clinical phase II trial called \texttt{EVADER} \cite{bratman_-escalation_nodate}. This trial's objective is to investigate whether radiotherapy to some lymph node areas can be omitted safely, thereby reducing toxicity, without risking regional recurrence in \gls{opscc} patients. We would like to perform a comparable trial in the future, where we employ our model to support the decision on which \glspl{lnl} can be spared from irradiation because of their low risk to harbor microscopic metastasis. Such a trial could -- if successful -- convincingly show that our model does not underestimate the risk for occult disease and at the same time produce more data on lymphatic progression patterns.

In conclusion, after further validation using more data and/or via a clinical trial, our probabilistic model on lymphatic metastatic tumor progression may refine guidelines on \gls{ctv-n} definition or replace them entirely. Instead of choosing the nodal volumes for irradiation based on guidelines that list recommendations for typical scenarios, \glspl{lnl} may be in- or excluded from the \gls{ctv-n} based on the predictions of our model. This model, in turn, can incorporate every newly diagnosed patient's individual clinical involvement state together with clinicopathological factors, such as \acrshort{hpv} status. Further extensions to the probabilistic framework may even further tailor the predictions to the patient and thereby optimize the balance between tumor control and preserving quality of life for every patient individually.

\end{document}