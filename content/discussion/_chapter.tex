\providecommand{\relativeRoot}{../..}
\documentclass[\relativeRoot/main.tex]{subfiles}
\graphicspath{
	{\relativeRoot/figures/}
    {\subfix{./figures/}}
}


\begin{document}

\chapter{Discussion and Outlook}
\label{chap:discussion}
\globalreset

Over the course of this thesis we have worked on three related aspects to address the problem of insufficiently quantified risks of occult nodal disease in \gls{hnscc}. During the first part of the thesis, in \cref{chap:dataset_usz,chap:dataset_clb} we described two datasets of \gls{opscc} patients, one extracted by our group at the \gls{usz}, and one extracted by \citeauthorandlink{bauwens_prevalence_2021} at the \gls{clb} and provided to us through a collaboration. Both datasets detail lymphatic metastatic progression patterns in unprecedented detail. This includes information on which \glspl{lnl} were involved simultaneously in a patient. This is a requirement for a more personalized estimation of the risk for microscopic metastases, because the correlation between involvements of different \glspl{lnl} is necessary to infer how the risk for occult disease depends on the extent of clinical lymph node involvement.

Through our effort, the \gls{usz} and \gls{clb} datasets were also made available to the public in their raw format (\acrshort{csv} tables in the repository \repolink{lyDATA}). And may be used by other researchers to improve the prediction of microscopic involvement in \gls{hnscc}. Hence, our work on detailed progression pattern datasets is already a significant contribution to the task of improving \gls{ctv-n} definition that depends on accurate estimates of risks for occult disease.

Moreover, making the data freely accessible may motivate other researchers to share similar datasets. Likely, datasets underlying the many other publications on the prevalence of \gls{lnl} involvement \cite{candela_patterns_1990,shah_patterns_1990,woolgar_histological_1999,woolgar_topography_2007,chao_determination_2002,vauterin_patterns_2006,razfar_incidence_2009,ho_patterns_2012,bauwens_prevalence_2021} also contain more detailed information than what is summarized in the publications. The \gls{gui} \inlinelyproxlogo{} that we developed and explained in \cref{chap:lyprox} may add to the motivation to share this data, because it demonstrates the value of it visually. But mainly, it enables researchers to quickly investigate correlations between \gls{lnl} involvements and other clinicopathological reported variables. Its functionality at this point does not yet allow its user to obtain a quantitative estimate for the risk of occult metastases in a \gls{lnl}. But it can already inform clinicians about which levels are most frequently involved together or how strong the impact of other risk factors on certain involvement patterns is. \inlinelyproxlogo{} also serves as an additional way to distribute datasets and hence improves the data's accessibility. Therefore, our \gls{gui} complements our efforts to provide lymphatic metastatic progression data to enable more personalized \gls{ctv-n} definition.

In parallel to the two aspects mentioned above, we developed an intuitive \gls{hmm} for predicting occult lymph node metastasis (see \cref{chap:unilateral}). The model represents the presumed underlying mechanics of lymphatic spread and consists of relatively few parameters that are interpretable. In \cref{chap:graph}, we extended the model to cover all relevant \glspl{lnl} in the neck region for \gls{opscc}, namely the levels I, II, III, IV, V, and VII. Subsequently, modelling the contralateral spread was incorporated into the model in \cref{chap:bilateral}. Throughout the three \cref{chap:unilateral,chap:graph,chap:bilateral} we have showed how accurate the model's fit to the data is. Finally, in \cref{chap:complete} we demonstrated how it may be used to guide volume de-escalated radiotherapy in \gls{hnscc} patients.

While testing our \gls{hmm} with retrospective data is an important part of model validation, we want to start working on the design of a clinical trial on volume deescalated radiotherapy based on the model. Currently, there already exists an ongoing clinical phase II trial called \texttt{EVADER} \cite{bratman_-escalation_nodate}. This trial's objective is to investigate whether radiotherapy to selected \glspl{lnl} can be omitted safely in \gls{opscc} patients, thereby reducing toxicity without risking regional recurrence. We would like to perform a comparable trial in the future, where we employ our model to support the decision on which \glspl{lnl} can be omitted from elective irradiation because of their low risk to harbor microscopic metastasis. Such a trial could -- if successful -- convincingly show that our model does not underestimate the risk for occult disease and at the same time produce more data on lymphatic progression patterns.

Ultimately, our model may then contribute to updated guidelines on further personalized elective \gls{ctv-n} definition, which optimize the balance between locoregional tumor control and preserving the quality of life for every patient individually.

\end{document}