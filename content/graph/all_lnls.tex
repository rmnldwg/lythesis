\providecommand{\relativeRoot}{../..}
\documentclass[\relativeRoot/main.tex]{subfiles}
\graphicspath{
    {\subfix{./figures/}}
}


\begin{document}

\section{Considering additional LNLs V and VII}
\label{sec:graph:all_lnls}

\begin{tcolorbox}[title=\faIcon{users} Contributions, parbox=false]
    \faIcon{user}~\textbf{Laurence Bauwens}~et~al.\cite{bauwens_prevalence_2021} extracted the data described here and used it to publish a study on the prevalence of lymph node metastases in \gls{opscc}. The raw data was not provided with the publication, but \faIcon{user} \textbf{Vincent Grégoire}, the \gls{pi} on the mentioned publication, kindly sent it to us and allowed us to analyze it, as well as making it publicly available.

    \faIcon{user}~\textbf{Roman Ludwig}~(me) normalized the data into the same format as our dataset from \cref{chap:dataset_usz}, analyzed both the differences to our dataset, as well as this data's sensitivity and specificity, which has not been done by the authors of the original article, and made the data publicly available.
\end{tcolorbox}


Up to this point, we have introduced a model to infer ipsilateral lymphatic spread from data and shown how it works on reconstructed data -- which was the only data available at the time of writing \cref{chap:unilateral} -- that reported pathological findings on the \glspl{lnl} I, II, III and IV. Now, we have access to more detailed datasets that both include lymphatic patterns of progression for the levels I through V and also \gls{lnl} VII. Hence, we will ow extend the graph shown in \cref{fig:bn:graph} to include the additional \glspl{lnl} V and VII, making use of model comparisons similar to what we have illustrated in \cref{sec:graph:simple}.

Following \cite{pouymayou_bayesian_2019}, we have so far connected the \glspl{lnl} I through IV in a chain. The data presented in \cref{chap:dataset} seems to support at least the connections from \gls{lnl} II to III and further from there into \gls{lnl} IV (see e.g. \cref{fig:dataset:flowchart}). However, neither the data nor the anatomy, as displayed in the schematics in \cref{fig:intro:schematics_head} or -- more detailed -- in Lengelé et al. \cite{lengele_anatomical_2007}, strongly support neither a flow from \gls{lnl} I to II, nor the other way around. This will be the first connection to be investigated.

Looking back at the anatomy, in \cref{fig:intro:schematics_head} we can see

\end{document}
