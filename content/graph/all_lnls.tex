\providecommand{\relativeRoot}{../..}
\documentclass[\relativeRoot/main.tex]{subfiles}
\graphicspath{
    {\subfix{./figures/}}
}


\begin{document}

\section{Considering additional LNLs V and VII}
\label{sec:graph:all_lnls}

\begin{tcolorbox}[title=\faIcon{users} Contributions, parbox=false]
    \faIcon{user}~\textbf{Laurence Bauwens}~et~al.\cite{bauwens_prevalence_2021} extracted the data described here and used it to publish a study on the prevalence of lymph node metastases in \gls{opscc}. The raw data was not provided with the publication, but \faIcon{user} \textbf{Vincent Grégoire}, the \gls{pi} on the mentioned publication, kindly sent it to us and allowed us to analyze it, as well as making it publicly available.

    \faIcon{user}~\textbf{Roman Ludwig}~(me) normalized the data into the same format as our dataset from \cref{chap:dataset_usz}, analyzed both the differences to our dataset, as well as this data's sensitivity and specificity, which has not been done by the authors of the original article, and made the data publicly available.
\end{tcolorbox}


\begin{figure}
    \centering
    \def\svgwidth{0.4\textwidth}
    \input{figures/extended-graph.pdf_tex}
    \caption[
        Base graph of the extended ipsilateral model with possible additional edges
    ]{
        Minimal base graph connecting the extended ipsilateral network of \glspl{lnl}. Red arcs connect the tumor $T$ to the \glspl{lnl} $X_v$ for $v \in \{ \text{I}, \text{II}, \text{III}, \text{IV}, \text{V}, \text{VII}, \}$, while the levels themselves are connected to each other with the blue arcs. Green arcs depict possible additional connections and deciding whether they should be added to better describe the data is the aim of this section.
    }
    \label{fig:graph:all_lnls:extended_graph}
\end{figure}

Up to this point, we have introduced a model to infer ipsilateral lymphatic spread from data and shown how it works on reconstructed data -- which was the only data available at the time of writing \cref{chap:unilateral} -- that reported pathological findings on the \glspl{lnl} I, II, III and IV. Now, we have access to more detailed datasets that both include lymphatic patterns of progression for the levels I through V and also \gls{lnl} VII. Hence, we will now extend the graph shown in \cref{fig:bn:graph} to include the additional \glspl{lnl} V and VII, making use of model comparisons similar to what we have illustrated in \cref{sec:graph:simple}.

Following \cite{pouymayou_bayesian_2019}, we have so far connected the \glspl{lnl} I through IV in a chain. The data presented in \cref{chap:dataset} seems to support at least the connections from \gls{lnl} II to III and further from there into \gls{lnl} IV (see e.g. \cref{fig:dataset:flowchart}). However, the data does not strongly support a flow from \gls{lnl} I to II, or the other way around, although the anatomy suggests indeed the lymphatics flow from \gls{lnl} I to \gls{lnl} II \cite{lengele_anatomical_2007}. Lengelé et al. \cite{lengele_anatomical_2007} also describe the \emph{posterior accessory pathway} that originates in \gls{lnl} II and runs down through \gls{lnl} V.

The way we are going to perform inference on the combined data from the \gls{usz} and the \gls{clb}, further requires us to draw connections from the primary tumor in the oropharynx to all \glspl{lnl} we are considering now. This is because after computing the maximum likelihood estimate for the true involvement of each patient in the dataset based on their diagnoses and pathology reports, we set the sensitivity and specificity of this combined assessment to 1. By doing so we assume -- perhaps wrongly -- that all observations represent the ground truth. We can then look at patients that e.g. show \emph{only} involvement in \gls{lnl} VII, which leaves us with e.g. 5 patients. If there were no direct arc from the tumor to \gls{lnl} VII, this would be impossible and render the likelihood zero regardless of the choice of parameter values. Hence, our inference algorithm, which essentially selects those parameter values based on the respective likelihood, would break down.

Based on the data, as well as anatomical and practical considerations, we can now draw a \emph{base graph} that allows us to do inference and compare it to models that are based on graphs with additional connections between the \glspl{lnl}. We have sketched this graph in \cref{fig:graph:all_lnls:extended_graph} (considering only the red and blue edges). Note that this base graph is not set up such that it allows modelling cause and effect of the lymphatic spread, which is anyway not how our models work, as we have shown in \cref{sec:graph:simple}. Rather, it is the other way around: We fix those arcs to make our model more interpretable, as it allows e.g. a clinician or patient, getting informed by the model, to assign meaning to the parameters an abstract model has inferred from correlations in data.

In \cref{fig:graph:all_lnls:base_graph}, we have also drawn green edges between the \glspl{lnl}. These additional connections are the subject of this section's experiments, as we are assessing whether adding them individually improves the model's accuracy enough to justify the greater model complexity.

\end{document}
