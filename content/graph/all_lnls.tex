\providecommand{\relativeRoot}{../..}
\documentclass[\relativeRoot/main.tex]{subfiles}
\graphicspath{
    {\subfix{./figures/}}
}


\begin{document}

\section{Considering additional LNLs V and VII}
\label{sec:graph:all_lnls}

\begin{tcolorbox}[
    title=\faIcon{users} Contributions,
    parbox=false,
    float
]
    A \acrlong{gui} introduced in this chapter was conceptualized by \faIcon{user}~\textbf{Bertrand Pouymayou} \cite{pouymayou_analysis_2019}. He implemented a Python-based interface for local use. \faIcon{user}~\textbf{Roman Ludwig}~(me) then developed an online tool, freely accessible at \superhref{https://lyprox.org}{\faIcon{external-link-alt}~\texttt{https://lyprox.org}} based on this prototype using the web framework Django \cite{noauthor_django_2022} with substantially extended functionality.
\end{tcolorbox}


Up to this point, we have introduced a model to infer ipsilateral lymphatic spread from data and shown how it works on reconstructed data -- which was the only data available at the time of writing \cref{chap:unilateral} -- that reported pathological findings on the \glspl{lnl} I, II, III and IV. Now, we have access to more detailed datasets that both include lymphatic patterns of progression for the levels I through V and also \gls{lnl} VII. Hence, we will ow extend the graph shown in \cref{fig:bn:graph} to include the additional \glspl{lnl} V and VII, making use of model comparisons similar to what we have illustrated in \cref{sec:graph:simple}.

Following \cite{pouymayou_bayesian_2019}, we have so far connected the \glspl{lnl} I through IV in a chain. The data presented in \cref{chap:dataset} seems to support at least the connections from \gls{lnl} II to III and further from there into \gls{lnl} IV (see e.g. \cref{fig:dataset:flowchart}). However, neither the data nor the anatomy, as displayed in the schematics in \cref{fig:intro:schematics_head} or -- more detailed -- in Lengelé et al. \cite{lengele_anatomical_2007}, strongly support neither a flow from \gls{lnl} I to II, nor the other way around. This will be the first connection to be investigated.

Looking back at the anatomy, in \cref{fig:intro:schematics_head} we can see

\end{document}
