\providecommand{\relativeRoot}{../..}
\documentclass[\relativeRoot/main.tex]{subfiles}
\graphicspath{
	{\relativeRoot/figures/}
    {\subfix{./figures/}}
}


\begin{document}

Interestingly, one can even find that (exhaustive leave-$p$-out) cross-validation (averaged over all $p$) and \gls{ti} are fundamentally computing the same quantity, as shown by Fong et al. \cite{fong_marginal_2019}. We will briefly follow their reasoning for illustrative purposes: Let us write the marginal log-likelihood as a sum of log-conditionals
%
\begin{equation} \label{eq:evidence_as_sequence}
    \ln{\Probofgiven{\boldsymbol{\mathcal{D}}}{\mathcal{M}}} = \sum_{i=1}^N{ \ln{ \Probofgiven{d_i}{\boldsymbol{\mathcal{D}}_{1:i-1}} } }
\end{equation}
%
with $\boldsymbol{\mathcal{D}}_{1:i-1}$ being all data points from the first to the one before the $i$-th. Notably, this is generally true regardless of how we order the data. The terms the sum runs over here are essentially expected likelihoods under different posteriors over the parameters:
%
\begin{equation}
    \begin{aligned}
        \ln{ \Probofgiven{d_i}{\boldsymbol{\mathcal{D}}_{1:i-1}} } &= \int{ \ln{ \Probofgiven{d_i}{\theta} } \probofgiven{\theta}{\boldsymbol{\mathcal{D}}_{1:i-1}} d\theta} \\
        &= \mathbb{E}_{\probofgiven{\theta}{\boldsymbol{\mathcal{D}}_{1:i-1}}} \big[ \ln{ \Probofgiven{d_i}{\theta} } \big]
    \end{aligned}
\end{equation}
%

\end{document}
