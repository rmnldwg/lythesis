\providecommand{\relativeRoot}{../..}
\documentclass[\relativeRoot/main.tex]{subfiles}
\graphicspath{{\subfix{./figures/}}}


\begin{document}

\section[Discussion on Model-Based Guideline Updates]{Discussion on Model-Based\\Guideline Updates}
\label{sec:complete:dicussion}

In the previous \cref{sec:complete:ipsiIV,sec:complete:ipsiV,sec:complete:contraII,sec:complete:contraIIIandIV} we have shown what our \gls{hmm}-based model, which we have developed, extended and tested in the \cref{chap:unilateral,chap:graph,chap:bilateral}, predicts for the risk of occult disease in different \glspl{lnl} and given a range of diagnoses. The results indicate that the elective \gls{ctv-n} definition as recommended by current guidelines \cite{gregoire_delineation_2014,biau_selection_2019} may be reduced in some cases, especially in the contralateral neck.

However, we stress that the predictions by our model still have to be considered with caution. They were trained on two datasets (see \cref{chap:dataset_usz, chap:dataset_clb}) where we -- for each patient individually -- computed the maximum likelihood consensus diagnosis (as detailed in \cref{subsec:dataset_clb:methods:max_llh}) and subsequently assumed this consensus to be the ground truth. We argue that this resulted in a large dataset of realistic lymphatic metastatic spread patterns, and allows for an easy comparison of predicted and observed prevalences, which we have extensively used to check our model. But as especially the \gls{usz} data \cite{ludwig_detailed_2021} contains mostly clinical diagnostic data and not -- like the \gls{clb} dataset \cite{bauwens_prevalence_2021} -- pathology reports, it might have underestimated the true rates of involvement in the patient cohort treated at the \gls{usz}. Consequently, the risk predictions presented in the \cref{sec:complete:ipsiIV,sec:complete:ipsiV,sec:complete:contraII,sec:complete:contraIIIandIV} may be too optimistic and underestimate the true risk.

Also, the presented \gls{hmm} models involvement as a binary \acrfull{rv} in the probabilistically correct manner. But by doing so it assumes that the simulated \gls{ct} diagnosis, which our risk predictions were conditioned on, correctly identifies the presence of nodal disease with the probability $\Probofgiven{Z=1}{X=1} = s_N$ equal to a typical \gls{ct}'s sensitivity. However, in truth the sensitivity of a diagnosis based on \gls{ct} scans to detect microscopic disease is probably far lower (see \cref{sec:future:trinary} for a potential solution to this), which, in turn, would mean the true risks may be higher than predicted.

Nonetheless, in this chapter, we have demonstrated how our model may be used in the future to augment prevalence-based guidelines by incorporating personalized patient diagnosis. It provides a quantitative assessment of risk for nodal microscopic disease that may prove valuable for clinicians tasked with the decision of what to include in the elective nodal \gls{ctv}.

\end{document}
