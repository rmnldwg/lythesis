\providecommand{\relativeRoot}{../..}
\documentclass[\relativeRoot/main.tex]{subfiles}
\graphicspath{{\subfix{./figures/}}}


\begin{document}

\section[Risk for Involvement in the Ipsilateral LNL IV]{Risk for Involvement\\in the Ipsilateral LNL IV}
\label{sec:complete:ipsi_IV}

\begin{figure}
    \centering
    \def\svgwidth{1.0\textwidth}
    \input{figures/ipsiIV-risks.pdf_tex}
    \caption[]{
        Histograms over risk for involvement in the ipsilateral \gls{lnl} II, given specific scenarios:
        \begin{enumerate*}
            \item[(green, shaded)] Early T-category patient with clinically N0 neck.
            \item[(blue, outlined)] Also early T-category, but with metastasis in the upstream \gls{lnl} II detected using \gls{ct}, for which we defined the specificity of $s_P = 76\%$, and sensitivity of $s_N = 81\%$. All other \glspl{lnl} show no suspicious lymph nodes.
            \item[(orange, shaded)] Late T-category with clinically N0 neck.
            \item[(red, shaded)] Late T-category with involvement in \gls{lnl} II, as detected by \gls{ct}.
            \item[(red, hatched)] Late T-category where involvement has been found in the levels II and III. The \gls{lnl} of interest to us, level IV, still shows no metastatic lmyph node on \gls{ct}
        \end{enumerate*}.
    }
\end{figure}

We begin with inspecting what risks our model predicts for the ipsilateral \gls{lnl} IV, given different diagnoses. The most recent guidelines by \citeauthorandlink{biau_selection_2019} recommend for all cases, even N0 patients, the elective irradiation of at least the ipsilateral levels II, III and IVa, regardless of T-category. According to the prediction of our model, however, the risk for occult disease in \gls{lnl} IV, given different \gls{ct} scan based diagnoses (specificity $s_P = 76\%$, sensitivity $s_N = 81\%$), suggests that the risk is only high ($\geq 5\%$) in case of late T-category and when both \gls{lnl} II and III ipsilaterally showed macroscopic metastases.

If we look at the spread probabilities, the model has inferred that the primary tumor spreads to \gls{lnl} IV with a probability of $\tilde{b}_\text{IV}^\text{i} = 0.96 \pm 0.26 \%$ per time-step, which is comparatively small. Spread from \gls{lnl} III to IV shows a rate of $\tilde{t}_{\text{III}\rightarrow\text{IV}} = 16.37 \pm 2.8 \%$. This means that as soon as level III shows likely involvement, the risk in \gls{lnl} IV rises quickly, too. But as long as the upstream \gls{lnl} III appears healthy, the model predicts a low probability of involvement in level IV.

\end{document}
