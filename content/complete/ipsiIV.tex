\providecommand{\relativeRoot}{../..}
\documentclass[\relativeRoot/main.tex]{subfiles}
\graphicspath{{\subfix{./figures/}}}


\begin{document}

\section[Risk for Involvement in the Ipsilateral LNL IV]{Risk for Involvement\\in the Ipsilateral LNL IV}
\label{sec:complete:ipsi_IV}

\begin{figure}
    \centering
    \def\svgwidth{1.0\textwidth}
    \input{figures/ipsiIV-risks.pdf_tex}
    \caption[]{
        Histograms over risk for involvement in the ipsilateral \gls{lnl} II, given specific scenarios:
        \begin{enumerate*}
            \item[(green, shaded)] Early T-category patient with clinically N0 neck.
            \item[(blue, outlined)] Also early T-category, but with metastasis in the upstream \gls{lnl} II detected using \gls{ct}, for which we defined the specificity of $s_P = 76\%$, and sensitivity of $s_N = 81\%$. All other \glspl{lnl} show no suspicious lymph nodes.
            \item[(orange, shaded)] Late T-category with clinically N0 neck.
            \item[(red, shaded)] Late T-category with involvement in \gls{lnl} II, as detected by \gls{ct}.
            \item[(red, hatched)] Late T-category where involvement has been found in the levels II and III. The \gls{lnl} of interest to us, level IV, still shows no metastatic lmyph node on \gls{ct}
        \end{enumerate*}.
    }
\end{figure}

We being with inspecting what risks our model predicts for the ipsilateral \gls{lnl} IV, given different diagnoses.

\end{document}
