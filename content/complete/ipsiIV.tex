\providecommand{\relativeRoot}{../..}
\documentclass[\relativeRoot/main.tex]{subfiles}
\graphicspath{
	{\relativeRoot/figures/}
    {\subfix{./figures/}}
}


\begin{document}

\section[Risk for Involvement in the Ipsilateral LNL IV]{Risk for Involvement\\in the Ipsilateral LNL IV}
\label{sec:complete:ipsiIV}

\begin{figure}
    \centering
    \def\svgwidth{1.0\textwidth}
    \input{figures/ipsiIV-risks.pdf_tex}
    \caption[
        Prediction by the complete model on ipsi LNL IV involvement
    ]{
        Distributions over predicted risks of nodal involvement of the ipsilateral \gls{lnl} IV in the form of histograms. The legend in the top right corner indicates T-category (\texttt{early} meaning T1 and T2, whereas \texttt{late} refers to T3 and T4) and the extent of clinically diagnosed nodal disease: \texttt{N0} in the legend indicates that based on a \gls{ct} scan with a specificity of $s_P = 76\%$ and sensitivity of $s_N = 81\%$, no macroscopic metastasis was found in any of the six \glspl{lnl} on either side of the neck. \texttt{CT finding in LNL II} suggests that in the ipsilateral level II suspicious nodes have been seen on the \gls{ct} scan, but all other \glspl{lnl} were still clinically negative. Here, the clinical involvement of the level III directly upstream is most influential for the risk in level IV.
    }
    \label{fig:complete:ipsiIV}
\end{figure}

We begin with inspecting what risks our model predicts for the ipsilateral \gls{lnl} IV, given different diagnoses. The most recent guidelines by \citeauthorandlink{biau_selection_2019} recommend for all cases, even N0 patients, the elective irradiation of at least the ipsilateral levels II, III and IVa, regardless of T-category. According to the prediction of our model, however, the risk for occult disease in \gls{lnl} IV, given different \gls{ct} scan based diagnoses (specificity $s_P = 76\%$, sensitivity $s_N = 81\%$), suggests that the risk is only high ($\geq 5\%$) in case of late T-category and when both \gls{lnl} II and III ipsilaterally showed macroscopic metastases (see \cref{fig:complete:ipsiV}).

If we look at the spread probabilities, the model has inferred that the primary tumor spreads to \gls{lnl} IV with a probability of $\tilde{b}_\text{IV}^\text{i} = 0.96 \pm 0.26 \%$ per time-step, which is comparatively small. Spread from \gls{lnl} III to IV shows a rate of $\tilde{t}_{\text{III}\rightarrow\text{IV}} = 16.37 \pm 2.8 \%$. This means that as soon as level III shows clinical involvement, the risk in \gls{lnl} IV rises quickly, too. But as long as the upstream \gls{lnl} III appears healthy, the model predicts a low probability of involvement in level IV.

Note that this risk, given a diagnosis, should not be compared with or mistaken for the prevalence of involvement in the ipsilateral level IV. In this case, the combined prevalence as observed in the datasets from the \gls{usz} and the \gls{clb} shows 10\% involvement in \gls{lnl} IV ipsilaterally (\superhref{https://re.lyprox.org/dffca0}{\faIcon{external-link-alt}~reproduce with LyProX}). However, this includes patients with macroscopic metastasis in level IV. To extract a comparable prevalence, we would need data of patients whose \gls{ct} scan showed no involvement, but a pathological examination following a neck dissection detected occult disease.

\end{document}
