\providecommand{\relativeRoot}{../..}
\documentclass[\relativeRoot/main.tex]{subfiles}
\graphicspath{
	{\relativeRoot/figures/}
    {\subfix{./figures/}}
}


\begin{document}

\section[Ipsilateral LNL V's Risk for Involvement]{Ipsilateral LNL V's Risk for Involvement}
\label{sec:complete:ipsiV}

\begin{figure}
    \centering
    \def\svgwidth{1.0\textwidth}
    \input{figures/ipsiV-risks.pdf_tex}
    \caption[
        Computed risks of involvement for ipsilateral LNL V
    ]{
        The histograms display the predicted risk for occult disease in \gls{lnl} V ipsilaterally, given successively increasing clinical involvement. They all display the risk for late T-category tumors, but start with a clinically N0 neck (green). The macroscopic metastases detected by a \gls{ct} scan (specificity $s_P = 76\%$, sensitivity $s_N = 81\%$) are located in the ipsilateral level II only (orange), II and III (red), and lastly in the levels II, III and IV (black). Even in the last case, the predicted risk does not reach 5\%. Again, the prediction assumes no finding in \gls{lnl} V, based on the \gls{ct} images.
    }
    \label{fig:complete:ipsiV}
\end{figure}

Guidelines for the elective definition of \gls{ctv-n} recommend including \gls{lnl} V ipsilaterally, as soon as the clinical N-category is N2a or higher \cite{biau_selection_2019}, meaning that either one large or multiple smaller nodal metastases are present \cite{brierley_tnm_2017}. In \cref{fig:complete:ipsiV}, we have plotted the risk for microscopic disease in the ipsilateral level V, given that the primary tumor is of late T-category and that other ipsilateral levels showed involvement to varying degrees on \gls{ct} imaging.

We found that the risk for involvement in \gls{lnl} V is generally low. Only when the ipsilateral neck is involved extensively (e.g., the \glspl{lnl} II, III and IV) does the computed risk rise to around 3.5\%. This finding does qualitatively agree with the guidelines: The elective irradiation of \gls{lnl} V is at the earliest indicated, when extensive nodal involvement is present.

\end{document}
