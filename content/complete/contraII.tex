\providecommand{\relativeRoot}{../..}
\documentclass[\relativeRoot/main.tex]{subfiles}
\graphicspath{{\subfix{./figures/}}}


\begin{document}

\section{Involvement Risk for Contralateral LNL II}
\label{sec:complete:contraII}

\begin{figure}
    \centering
    \def\svgwidth{1.0\textwidth}
    \input{figures/contraII-risks.pdf_tex}
    \caption[
        Complete model's risk predictions for contralateral LNL II
    ]{
        Histograms over risk for involvement in the contralateral \gls{lnl} II, given specific scenarios. In the legend, ``early'' (T1 and T2) and ``late'' (T3 and T4) refer to the T-category, ``no ext'' and ``ext'' indicate whether the tumor extended over the mid-sagittal plane. Lastly, ``N0'' means that based on a \gls{ct} scan (specificity $s_P = 76\%$, sensitivity $s_N = 81\%$), no nodal involvement was detected, whereas e.g. ``CT shows ipsi LNL II involved'' indicates that the \gls{ct} scan found suspicious lymph nodes in level II. However, all other \glspl{lnl} where still diagnosed to be healthy. The histograms indicate that the strongest predictor of contralateral involvement in \gls{lnl} II is the tumor's lateralization, i.e., whether it extends over the mid-plane.
    }
    \label{fig:complete:contraII}
\end{figure}

Looking at the guidelines' recommendations for the \gls{ctv-n} definition in the contralateral neck, we find that \gls{lnl} II (as well as III and IVa) should be included \cite{biau_selection_2019}. An exception are \glspl{scc} in the tonsil, where drainage appears to be more lateralized and data showed low contralateral neck failures when only ipsilateral irradiation was performed \cite{huang_re-evaluation_2017}.

Our model predicted the risk of contralateral level II involvement for a range of worsening scenarios, which are plotted in \cref{fig:complete:contraII}. It predicts that the tumor's extension over the mid-sagittal plane is the greatest risk factor for contralateral nodal disease. Only when the tumor is not lateralized anymore does the risk for occult disease in \gls{lnl} II contralaterally rise above 4\%. Note that these risks are still conditioned on the fact that a \gls{ct} scan did not find any suspicious lymph nodes in the contralateral level II.

As we have shown already in \cref{chap:bilateral}, our model is capable of correlating worsening ipsilateral diagnoses with increased risk for contralateral disease. As such, finding metastasis e.g. in the \glspl{lnl} II and/or III ipsilaterally increases the risk for nodal disease in the contralateral level II significantly.

Based on these predictions and the data we have provided our model with, recommendations for the definition of contralateral \gls{ctv-n} would be as follows: Ipsilateral irradiation only is indicated as long as the tumor is clearly lateralized, regardless of T-category and extent of nodal disease ipsilaterally. However, as soon as the primary tumor touches or extends over the mid-sagittal plane, again regardless of T-category or ipsilateral nodal involvement, inclusion of at least \gls{lnl} II into the contralateral \gls{ctv-n} is indicated.

\end{document}
