\providecommand{\relativeRoot}{..}
\documentclass[\relativeRoot/main.tex]{subfiles}


\begin{document}
    
\begin{abstract}
    \thispagestyle{plain}
    \Glspl{hnscc} are often treated using radiotherapy. Because this type of cancer frequently spreads to regional lymph nodes, parts of the lymphatic system are irradiated alongside the primary tumor. Macroscopic nodal metastasis, visible on imaging modalities such as \acrlong{ct}, is always included in the target volume of radiotherapy. Additionally, clinicians electively irradiate \glspl{lnl} when their estimated risk to harbor microscopic disease exceeds a given threshold. Currently, this risk estimate is based on the overall prevalence of metastatic involvement in a given \gls{lnl} as reported in the literature. However, few published studies on lymphatic metastatic progression patterns investigate how the probability of involvement of \glspl{lnl} depends on the involvement of other levels and clinicopathological factors such as T-category, which is of crucial importance to create more personalized elective target volumes. This work makes three related contributions to address this problem:

    \begin{enumerate}[label={(\arabic*)}]
        \item We extracted a dataset of patients diagnosed with oropharyngeal \acrshort{scc} at the \gls{usz}. This dataset details the observed involvement per \gls{lnl} and diagnostic modality together with tumor characteristics including subsite, T-category, lateralization and \acrshort{hpv} status. It is thereby the most detailed dataset on patterns of lymphatic progression to date.
        \item In addition to sharing the full dataset freely with the research community, we developed an intuitive web page named \href{https://lyprox.org}{\faIcon{external-link-alt}~LyProX}. The interface on this web page allows researchers to visually and interactively explore the data extracted at the \gls{usz}, as well as another dataset that was kindly provided to us by Vincent Grégoire in response to our efforts.
        \item Lastly, we have developed a \gls{hmm} that predicts the risk of microscopic disease, given an individual patient's clinical diagnosis. It builds on a previous probabilistic model using \glspl{bn} for the involvement of ipsilateral \glspl{lnl} I, II, III, and IV, but extends it to naturally include T-category as diagnostic variable. Using Bayesian model comparison, the \gls{hmm} framework is extended to include the \glspl{lnl} I, II, III, IV, V, and VII in both the ipsi- and the contralateral side of the neck, accounting for the primary tumor's extension over the mid-sagittal plane as risk factor for contralateral nodal involvement.
    \end{enumerate}

    Ultimately, we train the \gls{hmm} with the available data and demonstrate how the resulting predictions may be used to quantitatively predict the personalized risk of occult disease. We find that our model supports a reduction of electively irradiated nodal volumes, especially in the contralateral neck. As such, our model may support the design of future clinical trials on volume-deescalated \acrlong{rt} of \gls{hnscc}.
\end{abstract}

% \chapter*{Dedication}
% \thispagestyle{empty}
% To mum and dad

% \chapter*{Declaration}
% \thispagestyle{empty}
% I declare that..

\chapter*{Acknowledgements}
\thispagestyle{empty}
First, I want to thank my Professor and supervisor Jan Unkelbach for his relentless support and feedback throughout my PhD. Being able to discuss new ideas or occurring problems with him at almost any time was invaluable for this research project.

I am also grateful to Bertrand Pouymayou, a medical physics Postdoc in our group, who laid the groundwork on this topic and introduced me to the mathematical details of the model he had developed. His perspectives on different aspects of our modelling approach helped me gain a deeper understanding of formalism and its implications.

% Panagiotis Balermpas

This work was supported by the \gls{crpp}\\\emph{Artificial Intelligence in Oncological Imaging}.

\chapter*{Contributions}
\section*{Publications Included in this Dissertation}
\begin{itemize}
    \item \Cref{chap:dataset_usz}: \fullcite{ludwig_dataset_2022}
    \item \Cref{chap:dataset_usz}: \fullcite{ludwig_detailed_2022}
    \item \Cref{chap:unilateral}: \fullcite{ludwig_hidden_2021}
\end{itemize}

\section*{Abstracts Submitted to Conferences}
\begin{itemize}
    \item \fullcite{ludwig_predicting_2022}
    \item \fullcite{ludwig_personalized_2022}
    \item \fullcite{ludwig_dynamic_2021}
    \item \fullcite{ludwig_hidden_2020}
\end{itemize}

\section*{Not Included in this Dissertation}

During my PhD, I have worked on projects related to adaptive fractionation at the MR-Linac. Publications and abstracts related to that topic are listed below, but are not included in this thesis, as they are not related to the main topic of personalizing elective clinical target volume definition in \gls{hnscc}.

\subsection*{Publications}
\begin{itemize}
    \item \fullcite{mayinger_benefit_2021}
    \item \fullcite{perez_haas_adaptive_2022}. \emph{Submitted to Physics in Medicine and Biology}.
\end{itemize}

\subsection*{Abstracts}
\begin{itemize}
    \item \fullcite{mayinger_benefit_2022}
    \item \fullcite{perez_haas_adaptive_2022-1}
    \item \fullcite{perez_haas_adaptive_2021}
\end{itemize}

\tableofcontents
\thispagestyle{plain}

\end{document}