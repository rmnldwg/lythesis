\providecommand{\relativeRoot}{../..}
\documentclass[\relativeRoot/main.tex]{subfiles}
\graphicspath{
    {\subfix{./figures/}}
}

\begin{document}

\section{Discussion}
\label{sec:dataset_clb:discussion}

In this chapter we have described a dataset extracted by \citeauthorandlink{bauwens_prevalence_2021} at the \gls{clb} in detail. Its most valuable aspect is the reporting of pathological findings for each resected \gls{lnl}. We have compared its cohort of \gls{opscc} patients to the earlier cohort we collected at the \gls{usz} \cite{ludwig_detailed_2022}, especially w.r.t. nodal involvement patterns. We have found small differences between the two dataset's lymphatic metastatic progression patterns, which are probably to a large degree linked to differences in patient selection between the two datasets.

For most patients in the \gls{clb} cohort, data was available for both the clinical diagnosis, and the pathologically assessed ground truth of nodal involvement. Because of this, we were able to estimate the sensitivity and specificity of the \gls{clb}'s diagnosis processes. We found that, except for \gls{lnl} II ipsi- and contralaterally, the estimated values deviated substantially from literature values. We reasoned that this is because fundamental differences in what imaging modalities and a pathologist observe, but also in the uneven distribution of metastases over the \glspl{lnl} in the case of \gls{opscc}.

Later on in this thesis, we will use the \gls{usz} dataset combined with the \gls{clb} data in \cref{chap:graph,chap:bilateral,chap:complete} to train the probabilistic model introduced in \cref{chap:unilateral}. Because both the data from the \gls{usz} and from the \gls{clb} were available to us only after we had already developed the formalism shown in \cref{chap:unilateral}, these datasets were not yet used for the results in \cref{chap:unilateral}.

For the \cref{chap:graph,chap:bilateral,chap:complete}, we will combine all reported modalities in each of the two datasets using the maximum likelihood consensus from \cref{subsec:dataset_clb:methods:max_llh}. Then, we will assume the sensitivity and specificity of that consensus diagnosis to be $s_N = s_P = 1$. This is not a strictly probabilistic treatment of multiple available modalities (that will be introduced fomrally in \cref{subsec:unilateral:formalism:multimodality}), and it will likely lead to slightly inaccurate estimates for the risk prediction of nodal disease. However, it allows us to test if our model is able to learn the observed patterns of lymphatic progression by directly comparing model predictions to synthetic ``ground truth'' prevalences.

\end{document}
