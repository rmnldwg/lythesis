\providecommand{\relativeRoot}{../..}
\documentclass[\relativeRoot/main.tex]{subfiles}
\graphicspath{
    {\subfix{./figures/}}
}

\begin{document}

\chapter[
    CLB Dataset on Lymph Node Involvement in OPSCC
]{
    Dataset on Lymph Node Involvement in \acrlong{opscc} Extracted at the \gls{clb}
}
\chaptermark{CLB Dataset on Involvement in OPSCC}
\label{chap:dataset_clb}

Besides the \gls{usz} dataset introduced in the previous \cref{chap:dataset_usz}, data from another patient cohort, extracted at the \gls{clb} in France, was used in this thesis. It was kindly provided to us by Vincent Grégoire and is based on the publication by \citeauthorandlink{bauwens_prevalence_2021}.

The \gls{clb} dataset, like our \gls{usz} data, reports involvement information per \gls{lnl} for \gls{opscc} patients. But unlike the \gls{usz} cohort, all patients at the \gls{clb} were surgically treated and resected \glspl{lnl} were examined by a pathologist. Hence, in addition to a clinical diagnosis, the \gls{clb} dataset contains histopathological information on the metastatic state of each resected \gls{lnl}. This represents an invaluable addition to our \gls{usz} dataset, as the pathology information contained in the \gls{clb} data can basically be considered to be the ground truth.

This chapter will detail how this data was acquired, what information is contained in it and how it compares to the previously discussed \gls{usz} data.

\begin{tcolorbox}[title=\faIcon{users} Contributions, parbox=false]
    \faIcon{user}~\textbf{Laurence Bauwens}~et~al.\cite{bauwens_prevalence_2021} extracted the data described here and used it to publish a study on the prevalence of lymph node metastases in \gls{opscc}. The raw data was not provided with the publication, but \faIcon{user} \textbf{Vincent Grégoire}, the \gls{pi} on the mentioned publication, kindly sent it to us and allowed us to analyze it, as well as making it publicly available.

    \faIcon{user}~\textbf{Roman Ludwig}~(me) normalized the data into the same format as our dataset from \cref{chap:dataset_usz}, analyzed both the differences to our dataset, as well as this data's sensitivity and specificity, which has not been done by the authors of the original article, and made the data publicly available.
\end{tcolorbox}

\begin{tcolorbox}[
    title=\faIcon{database} Data,
    parbox=false,
    float
]
    The full dataset is available as a \acrshort{csv}-file via the Data-in-Brief article linked to this publication \cite{ludwig_dataset_2022} and in the repository \repolink{lyDATA} in a folder named \faIcon{folder}~\texttt{2021-usz-oropharynx}. In addition, the dataset has been archived and given a persistent identifier: \url{https://doi.org/10.5281/zenodo.6024778}.

    It can also be downloaded from our \gls{gui} \inlinelyproxlogo{}, where it is listed under the tab ``Datasets'', alongside our data. On that website, one may also explore the dataset interactively in the ``Data Viewer'' tab, which is described in the next section.
\end{tcolorbox}

\subfile{methods}
\subfile{results}
\subfile{discussion}

\end{document}
