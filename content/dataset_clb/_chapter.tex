\providecommand{\relativeRoot}{../..}
\documentclass[\relativeRoot/main.tex]{subfiles}
\graphicspath{
	{\relativeRoot/figures/}
    {\subfix{./figures/}}
}

\begin{document}

\chapter[
    CLB Dataset on Lymph Node Involvement in OPSCC
]{
    Dataset on\\Lymph Node Involvement\\in Oropharyngeal SCC\\Extracted at the CLB
}
\chaptermark{CLB Dataset on Involvement in OPSCC}
\label{chap:dataset_clb}
\globalreset

In addition to the \gls{usz} dataset introduced in the previous \cref{chap:dataset_usz}, data from another patient cohort, extracted at the \gls{clb} in France, was used in this thesis. It was kindly provided to us by Vincent Grégoire and is based on the publication by \citeauthorandlink{bauwens_prevalence_2021}.

The \gls{clb} dataset, like our \gls{usz} data, reports involvement information per \gls{lnl} for \gls{opscc} patients. But unlike the \gls{usz} cohort, all patients at the \gls{clb} were surgically treated and resected \glspl{lnl} were examined by a pathologist. Hence, in addition to a clinical diagnosis, the \gls{clb} dataset contains histopathological information on the metastatic state of each resected \gls{lnl}. This represents an invaluable addition to our \gls{usz} dataset, as pathology information is the gold standard of assessing nodal involvement. Hence, the reports contained in the \gls{clb} data will be treated by us as if they were the ground truth.

This chapter will detail how this data was acquired, what information is contained in it and how it compares to the previously discussed \gls{usz} data.

\begin{tcolorbox}[
    title=\faIcon{users} Contributions,
    parbox=false,
    float
]
    A \acrlong{gui} introduced in this chapter was conceptualized by \faIcon{user}~\textbf{Bertrand Pouymayou} \cite{pouymayou_analysis_2019}. He implemented a Python-based interface for local use. \faIcon{user}~\textbf{Roman Ludwig}~(me) then developed an online tool, freely accessible at \superhref{https://lyprox.org}{\faIcon{external-link-alt}~\texttt{https://lyprox.org}} based on this prototype using the web framework Django \cite{noauthor_django_2022} with substantially extended functionality.
\end{tcolorbox}

\begin{tcolorbox}[title=\faIcon{database} Data, parbox=false]
    For the following experiments, we combined two datasets:

    \begin{itemize}[leftmargin=5.5mm]
        \item[\faIcon{hospital}] \textbf{USZ}: 287 patients with newly diagnosed \gls{opscc}, treated at the \acrlong{usz} between 2013 and 2019. This dataset has been described in great detail in \cref{chap:dataset}.
        \item[\faIcon{hospital}] \textbf{CLB}: 263 patients with newly diagnosed \gls{opscc}, treated at the \acrlong{clb} between 2014 and 2018. A table containing this data was kindly provided to us by \textbf{Vincent Grégoire}, who was also the \gls{pi} of the study during which this data was extracted and for which it was analyzed \cite{bauwens_prevalence_2021}.
    \end{itemize}

    All available diagnostic and pathological modalities reported in the data were used in combination with literature values for their respective sensitivity and specificity \cite{de_bondt_detection_2007} to arrive at an estimate for the most likely true pattern of involvement for each patient.

    This maximum likelihood estimate was then fed into the model, assuming a sensitivity and specificity of 1. See \cref{chap:sens_spec_analysis} for an intuition into our motivations for this simplification.
\end{tcolorbox}


\subfile{methods}
\subfile{results}
\subfile{discussion}

\end{document}
