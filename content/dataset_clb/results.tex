\providecommand{\relativeRoot}{../..}
\documentclass[\relativeRoot/main.tex]{subfiles}
\graphicspath{
	{\relativeRoot/figures/}
    {\subfix{./figures/}}
}

\begin{document}

\section{Results}
\label{sec:dataset_clb:results}

\subsection*{Involvement Patterns and Differences}
\label{subsec:dataset_clb:results:patterns}

\begin{figure}
    \centering
    \input{figures/tstage_comp.pdf_tex}
    \caption[
        Comparison of T-category distribution in \gls{usz} and \gls{clb} data
    ]{
        Distribution over T-categories for both the patients in the \gls{usz} cohort (blue) and the \gls{clb} cohort (orange). The \gls{usz} dataset shows more patients with late T-category tumors, compared to the \gls{clb} data that even reports T0 cases.
    }
    \label{fig:dataset_clb:tstage_comp}
\end{figure}

To be able to compare some aspects of the \gls{clb} dataset to the data we extracted at the \gls{usz}, in \cref{fig:dataset_clb:statistics} we recreated the same figure as in \cref{sec:dataset_usz:results}. As mentioned in the last section, we also used another way to combine the diagnostic consensus and the pathology findings in the \gls{clb} data, compared to the logical \texttt{OR} we used to combine the modalities in our \gls{usz} dataset: The maximum likelihood consensus from \cref{subsec:dataset_clb:methods:max_llh}.

The patient cohort in the \gls{clb} dataset shows -- as expected -- similar patterns of \gls{lnl} involvement as the \gls{usz} dataset. Generally, the prevalence of involvement is slightly lower, e.g. in the ipsilateral \gls{lnl} II the \gls{usz} data reported 78\% overall and the \gls{clb} cohort shows only 68\%. Contralateral involvement in level II is also lower with 16\% in the \gls{usz} data and 10\% in the \gls{clb} set. Noticeably are also small differences in the prevalences of involvement for \glspl{lnl} V and VII: At the \gls{usz} level V was involved in 5\% and level VII in 10\%, but the \gls{clb} data shows 7\% for \gls{lnl} V and 3\% for level VII.

Contralaterally, the prevalence of involvement strongly increases when the primary tumor extends over the mid-sagittal plane: E.g., \gls{lnl} II in the contralateral neck showed metastasis in 4\% of the \gls{clb} patients, when the tumor was clearly lateralized. When the primary tumor extended over the mid-plane, \gls{lnl} II involvement contralaterally rose to 28\%. In the \gls{usz} dataset this figure increased from 9\% to 32\%, which is similar if again slightly higher. This can also be seen in the second row if \cref{fig:dataset_clb:statistics}, if compared to the same statistics for our \gls{usz} dataset in \cref{fig:dataset_usz:statistics}.

In the third row of \cref{fig:dataset_clb:statistics} we can see that the \gls{clb} data by \citeauthorandlink{bauwens_prevalence_2021} shows significantly less involvement in the case of \gls{hpv}-negative patients, both for early and late T-category. The patient cohort extracted at the \gls{usz}, however, only showed some differences in the prevalence of ipsilateral involvement for early T-category patients when comparing \gls{hpv}$+$ to \gls{hpv}$-$ patients. In case of late T-category, the \gls{usz} cohort showed no meaningful differences in the ipsilateral involvement patterns between \gls{hpv}+ and \gls{hpv}- patients.

The observed differences between these two datasets reporting per-level involvement of more than 250 \gls{opscc} patients each might stem from a combination of factors:

\begin{enumerate}
    \item Demographic differences between the two locations. The general population between France and Switzerland probably do not differ too much. However, different health care systems and practices regarding patient referral, as well as specializations by institutions may have an impact on the patient cohort treated at a specific clinic.
    \item Differences in patient selection. Our data contains all \gls{opscc} patients treated in a certain time frame, while the cohort from our colleagues in France contained only those treated with \acrlong{nd}. This may introduce a bias against advanced disease when those are referred to definitive chemo-radiotherapy. For example, it contains relatively more early T-category patients (67\% or 176 out of 263), while in our data we observe a roughly equal amount of early (52\% or 150 out of 287) and late T-category patients. We have plotted the differences in the distribution over T-categories in \cref{fig:dataset_clb:tstage_comp}.
    \item Different ways of combining multiple diagnoses into one estimate of the likely ``true'' involvement. For the \gls{usz} data we combined them using the logical OR, meaning we considered an \gls{lnl} to be metastatic as soon as one diagnostic modality found a suspicious node in that level. For the \gls{clb} cohort, again, we used the maximum likelihood method introduced in \cref{subsec:dataset_clb:methods:max_llh}.
    \item Not strictly comparable observations. The \gls{usz} data reports involvement from a number of diagnostic modalities separately, where for each one we defined criteria that decided when an \gls{lnl} was considered to be metastatic. The French dataset on the other hand reports the clinical diagnosis as a consensus decision by clinicians, and the pathology reports represent the gold standard for investigating the true presence of involvement, if they are available.
\end{enumerate}

Nonetheless, we still believe the two datasets to be comparable and that we may combine them to form a larger cohort as a basis for our efforts of modelling lymphatic progression. 

\begin{figure}
    \centering
    \def\svgwidth{1.0\textwidth}
    \input{figures/clb_statistics_crop.pdf_tex}
    \caption[
        Involvement statistics for the CLB cohort of OPSCC patients
    ]{
        Statistics as in \cref{fig:dataset_usz:statistics}, but for the \gls{clb} dataset:
        \begin{enumerate*}[label={(\alph*)}]
            \item Contralateral and
            \item ipsilateral prevalence of \gls{lnl} involvement for the whole patient cohort and stratified according to early (T1/T2) versus late (T3/T4) T-category
        \end{enumerate*}.
        Contralateral \gls{lnl} involvement stratified according to
        \begin{enumerate*}[label={(\alph*)},start=3]
            \item mid-sagittal plane extension and
            \item involvement of ipsilateral level III
        \end{enumerate*}.
        Ipsilateral \gls{lnl} involvement stratified according to HPV status for
        \begin{enumerate*}[label={(\alph*)},start=5]
            \item T1/T2 tumors and
            \item T3/T4 tumors
        \end{enumerate*}.
        The clinical diagnosis was combined with pathology reports using the max. likelihood consensus from \cref{subsec:dataset_clb:methods:max_llh}.
    }
    \label{fig:dataset_clb:statistics}
\end{figure}

\subsection*{Sensitivity and Specificity}
\label{subsec:dataset_clb:results:sens_spec}

\begin{table}
    \begin{subtable}{0.45\textwidth}
\centering
\caption{contralateral LNL Ib}
\begin{tabular}{|l|rr|}
\hline
\diagbox{truth}{observed} &  False &  True  \\

\hline
False &     25 &      0 \\
\hline
\end{tabular}
\end{subtable}
    \hfill
    \begin{subtable}{0.45\textwidth}
\centering
\caption{ipsilateral LNL Ib}
\begin{tabular}{|l|rr|}
\hline
\diagbox{path.}{clinical} &  False &  True  \\

\hline
False &    143 &      7 \\
True  &     13 &      5 \\
\hline
\end{tabular}
\end{subtable} \\ [2mm]
    \begin{subtable}{0.45\textwidth}
\centering
\caption{contralateral LNL II}
\begin{tabular}{|l|rr|}
\hline
\diagbox{path.}{clinical} &  False &  True  \\

\hline
False &     56 &      7 \\
True  &      6 &     17 \\
\hline
\end{tabular}
\end{subtable}
    \hfill
    \begin{subtable}{0.45\textwidth}
\centering
\caption{ipsilateral LNL II}
\begin{tabular}{|l|rr|}
\hline
\diagbox{truth}{observed} &  False &  True  \\

\hline
False &     62 &     20 \\
True  &     24 &    154 \\
\hline
\end{tabular}
\end{subtable} \\ [2mm]
    \begin{subtable}{0.45\textwidth}
\centering
\caption{contralateral LNL III}
\begin{tabular}{|l|rr|}
\hline
\diagbox{truth}{observed} &  False &  True  \\

\hline
False &     78 &      1 \\
True  &      5 &      2 \\
\hline
\end{tabular}
\end{subtable}
    \hfill
    \begin{subtable}{0.45\textwidth}
\centering
\caption{ipsilateral LNL III}
\begin{tabular}{|l|rr|}
\hline
\diagbox{path.}{clinical} &  False &  True  \\

\hline
False &    173 &     12 \\
True  &     40 &     35 \\
\hline
\end{tabular}
\end{subtable} \\ [2mm]
    \begin{subtable}{0.45\textwidth}
\centering
\caption{contralateral LNL IV}
\begin{tabular}{|l|rr|}
\hline
\diagbox{path.}{clinical} &  False &  True  \\

\hline
False &     80 &      1 \\
True  &      2 &      1 \\
\hline
\end{tabular}
\end{subtable}
    \hfill
    \begin{subtable}{0.45\textwidth}
\centering
\caption{ipsilateral LNL IV}
\begin{tabular}{|l|rr|}
\hline
\diagbox{truth}{observed} &  False &  True  \\

\hline
False &    226 &     12 \\
True  &     15 &      6 \\
\hline
\end{tabular}
\end{subtable} \\ [2mm]
    \begin{subtable}{0.45\textwidth}
\centering
\caption{contralateral LNL V}
\begin{tabular}{|l|rr|}
\hline
\diagbox{truth}{observed} &  False &  True  \\

\hline
False &     36 &      0 \\
True  &      1 &      0 \\
\hline
\end{tabular}
\end{subtable}
    \hfill
    \begin{subtable}{0.45\textwidth}
\centering
\caption{ipsilateral LNL V}
\begin{tabular}{|l|rr|}
\hline
\diagbox{truth}{observed} &  False &  True  \\

\hline
False &    121 &      3 \\
True  &     13 &      4 \\
\hline
\end{tabular}
\end{subtable}
    \caption[]{
        Confusion matrices between the diagnostic consensus and the pathology result in the \gls{clb} data report. The left column shows these matrices for the contralateral side, the right column for the ipsilateral \glspl{lnl}. The levels for which we computed confusion matrices were \gls{lnl} Ib (first row), II (second row), III (third row), IV (fourth row) and V (fifth row).
    }
    \label{table:dataset_clb:confusion}
\end{table}

Since for the \gls{clb} patient cohort both the clinical diagnosis of most, and the pathological examination result for many patient's \glspl{lnl} is available, we can estimate the sensitivity and specificity of the diagnostic procedure at the \gls{clb}. We have plotted Beta posteriors over the specificity (left column) and sensitivity (right column) in \cref{fig:dataset_clb:sens_spec} for each \gls{lnl} where enough data was available to compute all four necessary values (true positives (TP), false positives (FP), true negatives (TN) and false negatives (FN)).

\begin{tcolorbox}[
    title=\faIcon{lightbulb} Beta posterior,
    parbox=false,
    float
] \label{box:dataset_clb:results:beta}%
    The Beta posterior we used here comes from estimating the probability of a Binomial distribution using Bayesian inference: For example, the sensitivity $s_N$ is defined as the probability to observe a true positive event, meaning the diagnostic modality correctly identifies e.g. an \gls{lnl} as being metastatic. Hence, the number of true positives TP when P metastatic \glspl{lnl} are given can be described by a binomial distribution:
    %
    \begin{equation}
        \text{TP} \sim \operatorname{Bin}(\text{P}, s_N)
    \end{equation}
    %
    If we now assume that the sensitivity is Beta distributed $s_N \sim \operatorname{Beta}(\alpha, \beta)$, we can take advantage of the fact that the Beta distribution can act as a \emph{conjugate prior} to a Binomial likelihood. This means that the posterior distribution is of the same shape as the prior. Hence, using Bayesian inference, we can model the distribution over the sensitivity as an updated Beta distribution:
    %
    \begin{equation}
        \begin{aligned}
            \probofgiven{s_N}{\text{TP},\text{P}} &= \operatorname{Beta}(\alpha^\star, \beta^\star) \\
            &= \operatorname{Beta}(\alpha + \text{TP}, \beta + \text{FP})
        \end{aligned}
    \end{equation}
    %
    For the prior hyperparameters we have chosen $\alpha=1$ and $\beta=1$, amounting to a uniform prior over the sensitivity. In this case the maximum a posteriori estimate for $s_N$ coincides with its maximum likelihood estimate, but additionally, we naturally obtain a measure of uncertainty with it via the width of the distribution.

\end{tcolorbox}

It is immediately striking that -- except for level II -- the observed sensitivity is far below what \citeauthorandlink{de_bondt_detection_2007} reported for the three diagnostic modalities \gls{ct}, \gls{mri} and \gls{pet}, while the sensitivity is always higher. For \gls{lnl} II the literature values match our results, both for the ipsilateral side (green curve), and the contralateral neck (blue curve). However, as the prevalence for involvement in a given level decreases, so does the chance of diagnosing present disease.

We believe there are two reasons for this discrepancy between the literature values and the actually achieved sensitivities and specificities for the different \glspl{lnl}:

\begin{enumerate}
    \item The fundamental difference in what is actually observed by the imaging modalities commonly used to diagnose lymph nodes versus what a pathologist detects when investigating a resected tissue sample. Imaging techniques observe macroscopic morphological changes in lymph node, as a result of a growing tumor mass. Such macroscopic changes for example include enlarged lymph nodes, spherical shape and areas of central necrosis. As mentioned in \cref{sec:intro:diagnosis}, observing microscopic disease directly, would require imaging modalities to resolve individual tumor cells, which is currently beyond their capabilities.
    \item The distribution of occult metastases is not uniform over the \acrlongpl{lnl}. In the majority of \gls{opscc} cases, metastasis starts in the ipsilateral \gls{lnl} II. This means that the state of lymphatic metastatic progression is almost always more advanced in \gls{lnl} II than e.g. in \gls{lnl} IV. Consequently, it is more likely that metastases in level II have advanced sufficiently to be detectable by imaging, than it is for disease in \gls{lnl} IV to have reached that state.
\end{enumerate}

\begin{figure}
    \centering
    \def\svgwidth{1.0\textwidth}
    \input{figures/sens_spec_plot.pdf_tex}
    \caption[
        Sensitivity and specificity per LNL in the CLB dataset
    ]{
        Posterior Beta distributions over the specificities (left column) and sensitivities (right column) per reported \gls{lnl} (respective rows) in the \gls{clb} dataset. To compute the sensitivities and specificities, we compared the clinical diagnosis with what was pathologically reported after a \acrlong{nd} was performed. The vertical lines represent estimates for these values from the literature \cite{de_bondt_detection_2007,kyzas_18f-fluorodeoxyglucose_2008} for \gls{ct} (orange dashed line), \gls{mri} (red dash-dotted line) and \gls{pet} (black dotted line).
    }
    \label{fig:dataset_clb:sens_spec}
\end{figure}

\end{document}
