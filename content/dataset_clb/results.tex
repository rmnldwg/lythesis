\providecommand{\relativeRoot}{../..}
\documentclass[\relativeRoot/main.tex]{subfiles}
\graphicspath{
    {\subfix{./figures/}}
}

\begin{document}

\section{Results}
\label{sec:dataset_clb:results}

\subsection*{Involvement patterns and differences}
\label{subsec:dataset_clb:results:patterns}

To be able to compare some aspects of this dataset to what we extracted at our institution, in \cref{fig:dataset_clb:statistics} we recreated the same figure as in \cref{sec:dataset_usz:results}, just using the data we received from the \acrlong{clb}. As mentioned in the last section, we also used another way of combining the diagnostic consensus and the pathology findings, compared to what we used to combine the modalities in our dataset: The maximum likelihood consensus from \cref{subsec:dataset_clb:methods:max_llh}.

At first glance, this cohort does not seem to differ fundamentally from ours, although the general prevalence of involvement appears slightly lower, e.g. in the ipsilateral \gls{lnl} II where our data reported 78\% overall and this cohort shows only 68\%. Contralateral involvement in this level is also lower with 16\% in our data and 10\% in theirs. Noticeably, the prevalences of involvement for \glspl{lnl} V and VII seem switched: In our case it was 5\% and 10\% for these levels respectively, but the \gls{clb} data shows 7\% and 3\%.

In the second row, we find that both for our and for their data a tumor extending over the mid-sagittal plane or involvement of ipsilateral \gls{lnl} III are similarly correlated to an increased frequency of contralateral involvement.

In the third row of \cref{fig:dataset_clb:statistics} the data by \citeauthorandlink{bauwens_prevalence_2021} shows significantly less involvement in the case of \gls{hpv}-negative patients. The patient cohort extracted at the \gls{usz} even showed no difference in the prevalence of ipsilateral involvement when comparing late T-stage patients that are \gls{hpv}$+$ or \gls{hpv}$-$.

The observed differences between these two datasets reporting per-level involvement of more than 250 \gls{opscc} patients each might stem from a combination of factors:

\begin{enumerate}
    \item Demographic differences between the two locations/countries.
    \item Differences in patient sampling. Our data contains all \gls{opscc} patients treated in a certain time frame, while the cohort from our colleagues in France contained only those treated with \acrlong{nd}. This may introduce a bias against advanced disease when those are referred to definitive chemo-radiotherapy. For example, it contains relatively more early T-stage patients (67\% or 176 out of 263), while in our data we observe a roughly equal amount of early (52\% or 150 out of 287) and late T-stage patients.
    \item Different ways of combining multiple diagnoses into one estimate of the likely ``true'' involvement. For the \gls{usz} data we combined them using the logical OR, meaning we considered an \gls{lnl} to be metastatic as soon as one diagnostic modality found a suspicious node in that level. For the \gls{clb} cohort, again, we used the maximum likelihood method introduced in \cref{subsec:dataset_clb:methods:max_llh}.
    \item Not strictly comparable observations. The \gls{usz} data reports involvement from a number of diagnostic modalities separately, where for each one we defined criteria that decided when an \gls{lnl} was considered to be metastatic. The French dataset on the other hand reports the clinical diagnosis as a consensus decision by clinicians, and the pathology reports basically represent the ground truth, if available.
\end{enumerate}

Nonetheless, we still believe the two datasets to be comparable and that we may combine them to form a larger cohort as a basis for our efforts of modelling lymphatic progression. 

\begin{figure}
    \centering
    \def\svgwidth{1.0\textwidth}
    \input{figures/statistics_crop.pdf_tex}
    \caption[
        Statistics that characterize the cohort of OPSCC patients extracted at the CLB
    ]{
        Statistics as in \cref{fig:dataset_usz:statistics}, but for the \gls{clb} dataset:
        \begin{enumerate*}[label={(\alph*)}]
            \item Contralateral and
            \item ipsilateral prevalence of \gls{lnl} involvement for the whole patient cohort and stratified according to early (T1/T2) versus late (T3/T4) T-category
        \end{enumerate*}.
        Contralateral \gls{lnl} involvement stratified according to
        \begin{enumerate*}[label={(\alph*)},start=3]
            \item midsagittal plan extension and
            \item involvement of ipsilateral level III
        \end{enumerate*}.
        Ipsilateral \gls{lnl} involvement stratified according to HPV status for
        \begin{enumerate*}[label={(\alph*)},start=5]
            \item T1/T2 tumors and
            \item T3/T4 tumors
        \end{enumerate*}.
        Note that the clinical diagnosis was combined with the pathology reports using the max. likelihood consensus (see \cref{subsec:dataset_clb:methods:max_llh}).
    }
    \label{fig:dataset_clb:statistics}
\end{figure}

\subsection*{Sensitivity and specificity}
\label{subsec:dataset_clb:results:sens_spec}

\begin{figure}
    \centering
    \def\svgwidth{1.0\textwidth}
    \input{figures/sens_spec_plot.pdf_tex}
    \caption[
        Sensitivity and specificity per LNL in the CLB dataset
    ]{
        Posterior Beta distributions over the specificities (left column) and sensitivities (right column) per reported \gls{lnl} (respective rows) in the \gls{clb} dataset. To compute the sensitivities and specificities, we compared the clinical diagnosis with what was pathologically reported after a \gls{nd} was performed. The vertical lines represent estimates for these values from the literature \cite{de_bondt_detection_2007}.
    }
    \label{fig:dataset_clb:sens_spec}
\end{figure}

Since for this patient cohort both the clinical diagnosis of most, as well as the pathological examination results for many patient's \glspl{lnl} is available, we can estimate the sensitivity and specificity of the diagnosis at the \gls{clb}. We have plotted Beta posteriors over the specificity (left column) and sensitivity (right column) in \cref{fig:dataset_clb:sens_spec} for each \gls{lnl} where enough data was available to compute all four necessary values (true positives, false positives, true negatives and false negatives).

It is immediately striking that -- except for level II -- the observed sensitivity is far below what \citeauthorandlink{de_bondt_detection_2007} reported for the three diagnostic modalities \gls{ct}, \gls{mri} and \gls{pet}, while the sensitivity is always above. For \gls{lnl} II the literature values seem to be fairly representative, both for the ipsilateral side (green curve), and the contralateral neck (blue curve). However, as the prevalence for involvement in a given level decreases, so does the chance of diagnosing present disease.

We believe there is a profound reason for this discrepancy between the achieved sensitivities and specificities for the different \glspl{lnl}: The fundamental difference in what is actually observed by the imaging modalities commonly used to diagnose lymph nodes versus what a pathologist detects when investigating a resected tissue sample. In the former case, what imaging is in principle capable to detect are macroscopic changes in the tissue as exerted by the presence of a large enough tumor mass (e.g. swelling of an \gls{lnl}, necrosis, etc). The latter, however, is able to see individual cancer cells directly under a microscope.

In other words, e.g. a \gls{ct} scanner cannot -- with resolutions as they are currently achievable -- detect microscopic disease in an \gls{lnl}, as long as it has not yet altered the local anatomy or tissue characteristics to a large enough extent. If it had, however, the disease would by definition not be ``microscopic'' anymore, since we could then detect it with said imaging modality, albeit still not with certainty, since we might mistake a lymph node that is swollen from an infection to be a malignancy.

\end{document}
