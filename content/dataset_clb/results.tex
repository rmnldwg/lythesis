\providecommand{\relativeRoot}{../..}
\documentclass[\relativeRoot/main.tex]{subfiles}
\graphicspath{
    {\subfix{./figures/}}
}

\begin{document}

\section{Results}
\label{sec:dataset_clb:results}

\subsection*{Involvement Patterns and Differences}
\label{subsec:dataset_clb:results:patterns}

\begin{figure}
    \centering
    \input{figures/tstage_comp.pdf_tex}
    \caption[
        Comparison of T-stage distribution in \gls{usz} and \gls{clb} data
    ]{
        Distribution over T-stages for both the patients in the \gls{usz} cohort (blue) and the \gls{clb} cohort (orange). The \gls{usz} dataset shows more patients with late T-stage tumors, compared to the \gls{clb} data that even reports T0 cases.
    }
    \label{fig:dataset_clb:tstage_comp}
\end{figure}

To be able to compare some aspects of the \gls{clb} dataset to the data we extracted at the \gls{usz}, in \cref{fig:dataset_clb:statistics} we recreated the same figure as in \cref{sec:dataset_usz:results}. As mentioned in the last section, we also used another way to combine the diagnostic consensus and the pathology findings in the \gls{clb} data, compared to the logical \texttt{OR} we used to combine the modalities in our \gls{usz} dataset: The maximum likelihood consensus from \cref{subsec:dataset_clb:methods:max_llh}.

The patient cohort in the \gls{clb} dataset shows -- as expected -- similar patterns of \gls{lnl} involvement as the \gls{usz} dataset. Generally, the prevalence of involvement is slightly lower, e.g. in the ipsilateral \gls{lnl} II the \gls{usz} data reported 78\% overall and the \gls{clb} cohort shows only 68\%. Contralateral involvement in level II is also lower with 16\% in the \gls{usz} data and 10\% in the \gls{clb} set. Noticeably are also small differences in the prevalences of involvement for \glspl{lnl} V and VII: At the \gls{usz} level V was involved in 5\% and level VII in 10\%, but the \gls{clb} data shows 7\% for \gls{lnl} V and 3\% for level VII.

Contralaterally, the prevalence of involvement strongly increases when the primary tumor extends over the mid-sagittal plane: E.g., \gls{lnl} II in the contralateral neck showed metastasis in 4\% of the \gls{clb} patients, when the tumor was clearly lateralized. When the primary tumor extended over the mid-plane, \gls{lnl} II involvement contralaterally rose to 28\%. In the \gls{usz} dataset this figure increased from 9\% to 32\%, which is similar if again slightly higher. This can also be seen in the second row if \cref{fig:dataset_clb:statistics}, if compared to the same statistics for our \gls{usz} dataset in \cref{fig:dataset_usz:statistics}.

In the third row of \cref{fig:dataset_clb:statistics} we can see that the \gls{clb} data by \citeauthorandlink{bauwens_prevalence_2021} shows significantly less involvement in the case of \gls{hpv}-negative patients, both for early and late T-stage. The patient cohort extracted at the \gls{usz}, however, only showed some differences in the prevalence of ipsilateral involvement for early T-stage patients when comparing patients that are \gls{hpv}$+$ or \gls{hpv}$-$. In case of late T-stage, the \gls{usz} cohort showed no meaningful difference in the ipsilateral involvement patterns of \gls{hpv}+ or \gls{hpv}- patients.

The observed differences between these two datasets reporting per-level involvement of more than 250 \gls{opscc} patients each might stem from a combination of factors:

\begin{enumerate}
    \item Demographic differences between the two locations. The general population between France and Switzerland probably do not differ too much. However, different health care systems and practices regarding patient referral, as well as specializations by institutions may have an impact on the patient demographics treated at a specific clinic.
    \item Differences in patient selection. Our data contains all \gls{opscc} patients treated in a certain time frame, while the cohort from our colleagues in France contained only those treated with \acrlong{nd}. This may introduce a bias against advanced disease when those are referred to definitive chemo-radiotherapy. For example, it contains relatively more early T-stage patients (67\% or 176 out of 263), while in our data we observe a roughly equal amount of early (52\% or 150 out of 287) and late T-stage patients. We have plotted the differences in the distribution over T-stages in \cref{fig:dataset_clb:tstage_comp}.
    \item Different ways of combining multiple diagnoses into one estimate of the likely ``true'' involvement. For the \gls{usz} data we combined them using the logical OR, meaning we considered an \gls{lnl} to be metastatic as soon as one diagnostic modality found a suspicious node in that level. For the \gls{clb} cohort, again, we used the maximum likelihood method introduced in \cref{subsec:dataset_clb:methods:max_llh}.
    \item Not strictly comparable observations. The \gls{usz} data reports involvement from a number of diagnostic modalities separately, where for each one we defined criteria that decided when an \gls{lnl} was considered to be metastatic. The French dataset on the other hand reports the clinical diagnosis as a consensus decision by clinicians, and the pathology reports basically represent the ground truth, if available.
\end{enumerate}

Nonetheless, we still believe the two datasets to be comparable and that we may combine them to form a larger cohort as a basis for our efforts of modelling lymphatic progression. 

\begin{figure}
    \centering
    \def\svgwidth{1.0\textwidth}
    \input{figures/statistics_crop.pdf_tex}
    \caption[
        Statistics that characterize the cohort of OPSCC patients extracted at the CLB
    ]{
        Statistics as in \cref{fig:dataset_usz:statistics}, but for the \gls{clb} dataset:
        \begin{enumerate*}[label={(\alph*)}]
            \item Contralateral and
            \item ipsilateral prevalence of \gls{lnl} involvement for the whole patient cohort and stratified according to early (T1/T2) versus late (T3/T4) T-category
        \end{enumerate*}.
        Contralateral \gls{lnl} involvement stratified according to
        \begin{enumerate*}[label={(\alph*)},start=3]
            \item mid-sagittal plane extension and
            \item involvement of ipsilateral level III
        \end{enumerate*}.
        Ipsilateral \gls{lnl} involvement stratified according to HPV status for
        \begin{enumerate*}[label={(\alph*)},start=5]
            \item T1/T2 tumors and
            \item T3/T4 tumors
        \end{enumerate*}.
        Note that the clinical diagnosis was combined with the pathology reports using the max. likelihood consensus (see \cref{subsec:dataset_clb:methods:max_llh}).
    }
    \label{fig:dataset_clb:statistics}
\end{figure}

\subsection*{Sensitivity and Specificity}
\label{subsec:dataset_clb:results:sens_spec}

\begin{figure}
    \centering
    \def\svgwidth{1.0\textwidth}
    \input{figures/sens_spec_plot.pdf_tex}
    \caption[
        Sensitivity and specificity per LNL in the CLB dataset
    ]{
        Posterior Beta distributions over the specificities (left column) and sensitivities (right column) per reported \gls{lnl} (respective rows) in the \gls{clb} dataset. To compute the sensitivities and specificities, we compared the clinical diagnosis with what was pathologically reported after a \gls{nd} was performed. The vertical lines represent estimates for these values from the literature \cite{de_bondt_detection_2007,kyzas_18f-fluorodeoxyglucose_2008}.
    }
    \label{fig:dataset_clb:sens_spec}
\end{figure}

Since for this patient cohort both the clinical diagnosis of most, as well as the pathological examination results for many patient's \glspl{lnl} is available, we can estimate the sensitivity and specificity of the diagnosis at the \gls{clb}. We have plotted Beta posteriors over the specificity (left column) and sensitivity (right column) in \cref{fig:dataset_clb:sens_spec} for each \gls{lnl} where enough data was available to compute all four necessary values (true positives, false positives, true negatives and false negatives).

It is immediately striking that -- except for level II -- the observed sensitivity is far below what \citeauthorandlink{de_bondt_detection_2007} reported for the three diagnostic modalities \gls{ct}, \gls{mri} and \gls{pet}, while the sensitivity is always above. For \gls{lnl} II the literature values seem to be fairly representative, both for the ipsilateral side (green curve), and the contralateral neck (blue curve). However, as the prevalence for involvement in a given level decreases, so does the chance of diagnosing present disease.

We believe there is a profound reason for this discrepancy between the achieved sensitivities and specificities for the different \glspl{lnl}: The fundamental difference in what is actually observed by the imaging modalities commonly used to diagnose lymph nodes versus what a pathologist detects when investigating a resected tissue sample. In the former case, what imaging is in principle capable to detect are macroscopic changes in the tissue as exerted by the presence of a large enough tumor mass (e.g. swelling of an \gls{lnl}, necrosis, etc). The latter, however, is able to see individual cancer cells directly under a microscope.

In other words, e.g. a \gls{ct} scanner cannot -- with resolutions as they are currently achievable -- detect microscopic disease in an \gls{lnl}, as long as it has not yet altered the local anatomy or tissue characteristics to a large enough extent. If it had, however, the disease would by definition not be ``microscopic'' anymore, since we could then detect it with said imaging modality, albeit still not with certainty, since we might mistake a lymph node that is swollen from an infection to be a malignancy.

\end{document}
