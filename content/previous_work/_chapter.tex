\providecommand{\relativeRoot}{../..}
\documentclass[\relativeRoot/main.tex]{subfiles}


\begin{document}

\chapter{Previous Work}
\label{chap:previous_work}

In this chapter we will present what has been done -- both by the overall research community and in our research group at the \gls{uzh} and \gls{usz} -- prior to the work we present in this thesis. The three following \cref{sec:previous_work:sanguineti,sec:previous_work:prototype_gui,sec:previous_work:bayesian_network} will briefly introduce the three aspects we have focused on in our effort to make the \acrfull{ctv-n} definition more personalized.

First, in \cref{sec:previous_work:sanguineti}, we show what data is available that reports regional lymph node involvement in \gls{hnscc} and why the majority of it is insufficient to draw conclusions from this data about the lymphatic metastatic progression. The section will also explain how \citeauthorandlink{pouymayou_analysis_2019} began to work towards more detailed available data. Secondly, in \cref{sec:previous_work:prototype_gui}, and also based on the work by \citeauthorandlink{pouymayou_analysis_2019}, we show the origin of the idea to visualize the complex data of lymphatic metastatic patterns. Lastly, \cref{sec:previous_work:bayesian_network} discusses an approach to model the process of metastasis via the lymph system, notably the work by \citeauthorandlink{pouymayou_bayesian_2019}, which is also the basis for the later modelling work in this thesis.

\subfile{sanguineti}
\subfile{prototype_gui}
\subfile{bayesian_network}
\subfile{limitations}

\end{document}