\providecommand{\relativeRoot}{../..}
\documentclass[\relativeRoot/main.tex]{subfiles}


\begin{document}

\chapter[Previous Work on Personalizing CTV-N Definition]{Previous Work on Personalizing Elective CTV Definition}
\chaptermark{Previous Work on CTV-N Definition}
\label{chap:previous_work}

In this chapter we will summarize what has been done prior to the work we present in this thesis both by the overall research community and in our research group at the \gls{uzh} and \gls{usz}. The three following \cref{sec:previous_work:sanguineti,sec:previous_work:prototype_gui,sec:previous_work:bayesian_network} will briefly establish where the work of this thesis began w.r.t. the three aspects outlined in the abstract that we have focused on in our effort to make the \gls{ctv-n} definition more personalized.

First, in \cref{sec:previous_work:sanguineti}, we show what data was available on regional lymph node involvement in \gls{hnscc} and why the majority of it is insufficient to further personalize \gls{ctv-n} definition. \Cref{sec:previous_work:prototype_gui} will explain how \citeauthorandlink{pouymayou_analysis_2019} began to work towards making more detailed data available by visualizing the complex data of lymphatic metastatic patterns. And lastly, \cref{sec:previous_work:bayesian_network} introduces the work about modelling lymphatic spread in \gls{hnscc} using \glspl{bn} by \citeauthorandlink{pouymayou_bayesian_2019}, which is the starting point for the statistical modelling work in this thesis.

\subfile{sanguineti}
\subfile{prototype_gui}
\subfile{bayesian_network}

\end{document}