\providecommand{\relativeRoot}{../..}
\documentclass[\relativeRoot/main.tex]{subfiles}
\graphicspath{{\subfix{./figures/}}}


\begin{document}

\section[Bayesian Network Model for Lymphatic Spread]{A Bayesian Network Model\\for Lymphatic Metastatic Progression}
\label{sec:previous_work:bayesian_network}

To model lymphatic metastatic spread in \glspl{hnscc} probabilistically, the authors \citeauthorandlink{pouymayou_bayesian_2019} started by representing the state of an \gls{lnl} using a binary \gls{rv}. It may take on the value $X=0$ (or the boolean value \texttt{FALSE}) to indicate a healthy node level. If it harbors metastatic disease, its state would correspondingly be described with the value $X=1$ (or \texttt{TRUE}). This state indicates if there is truly disease present in an \gls{lnl}, including the presence of occult metastases. However, in practice we do not have access to that truth, which is why these \glspl{rv} representing the state of an \gls{lnl} are termed \emph{hidden} \glspl{rv}.

What we do have access to, is an array of diagnostic modalities with which we may be able to infer the true state of involvement. Most used for diagnosis are imaging techniques such as \gls{pet}, \gls{ct} and \gls{mri}, but palpation or \gls{fna} are also used. The diagnosis, too, can be modelled as a binary \gls{rv} -- this time an \emph{observed} one -- taking on the values $Z=0$ (\texttt{FALSE}) for a clinically healthy \gls{lnl} and $Z=1$ (\texttt{TRUE}) if the examination resulted in a finding in the corresponding level.

For notational convenience, we collect the hidden and observed \glspl{rv} in a random vector each:
%
\begin{equation} \label{eq:bn:variables}
    \begin{aligned}
        \text{hidden}& & \mathbf{X} &= ( X_v ) \rightarrow \{ 0,1 \}^V \\
        \text{observed}& & \mathbf{Z} &= ( Z_v ) \rightarrow \left\{ 0, 1 \right\}^V
    \end{aligned}
\end{equation}
%
where $V$ is the number of \glspl{lnl} $v \in \{ 1,2,\ldots,V \}$ in the graph. The conditional probabilities that link the hidden state to the observations can be written as follows:
%
\begin{equation}
    \begin{aligned}
        \Probofgiven{Z_v = z_v}{X_v = x_v} &= \left( z_v + (-1)^{z_v} \cdot s_P \right)(1 - x_v) \\ 
        &+ \left( (1 - z_v) + (-1)^{1 - z_v} \cdot s_N \right) x_v
    \end{aligned}
\end{equation}
%
with $s_N$ and $s_P$ being the sensitivity and specificity of the used diagnostic method. Below we give the conditional probability table of $\Probofgiven{Z}{X}$ for all combination of $X$ and $Z$:

\begin{center}
    \begin{tabular}{|c|cc|}
        \hline
        \diagbox{$Z$}{$X$} & 0 & 1 \\
        \hline
        0 & $s_P$ & $1 - s_N$ \\
        1 & $1 - s_P$ & $s_N$ \\
        \hline
    \end{tabular}
\end{center}

Spread of the tumor through the lymphatic network is represented in this model by directed arcs to and between \glspl{lnl} as illustrated in \cref{fig:bn:graph}. We introduce an additional vertex to the graph representing the primary tumor. Directed arcs from the primary tumor to an \gls{lnl} represent direct spread of tumor cells from the primary tumor to the \gls{lnl}. These arcs are associated with parameters $b_v$ that we call base probabilities, and which indicate the probability that the tumor spreads directly to \gls{lnl} $v$. When \gls{lnl} $s$ receives efferent lymphatics from \gls{lnl} $r$, this too is represented by a directed arc from \gls{lnl} $r$ to $s$, and $r=\pa{(s)}$ which is called a parent node of $s$. These arcs are associated with a transition probability $t_{rs}$ from $r$ to $s$. The resulting \gls{dag} is shown in \cref{fig:bn:graph}, comprising ipsilateral levels I, II, II, through IV, and will be used in \cref{chap:unilateral}, too.

\begin{figure}
    \centering
    \def\svgwidth{0.35\textwidth}
    \input{figures/graph.pdf_tex}
    \caption[DAG representing the Bayesian network as described by Pouymayou et al.]{Bayesian network for modelling lymphatic metastatic spread as described by (Pouymayou et al. \cite{pouymayou_bayesian_2019}). It consists of the primary tumor $T$ (red circle), hidden binary variables $X_v$ for the involvement of \gls{lnl} $v$ (blue outlined circles) and observed (or diagnostic) variables $Z_v^\mathcal{O}$ (orange, filled squares), where $\mathcal{O}$ denotes the diagnostic modality, since there are potentially many observations per hidden variable. Annotated arcs depict the direction of lymphatic flow where the parameter next to it ($b$ in red and $t$ in blue) represents the probability of metastatic spread. Arrows connecting the \glspl{lnl} to their diagnoses via sensitivity and specificity in orange.}
    \label{fig:bn:graph}
\end{figure}

The parameters $b_v$ and $t_{rs}$ associated with the directed arcs represent conditional probabilities, i.e. $b_v$ answers the following question: Given that all parent nodes are healthy, how likely is it that the primary tumor spreads to node v? $t_{rs}$ on the other hand, can answer the question: Assuming no efferent spread from the primary tumor and given that all parent nodes except $r$ are healthy, what is the likelihood of spread to node $s$? The conditional probability for involvement of \gls{lnl} $s$ given the state of its parent node $r = \pa{(s)}$ is then given by
%
\begin{equation} \label{eq:bn_cp}
    P_\text{BN} \left( X_s = x_s \mid X_r = x_r, b_s, t_{rs} \right) = x_s + (-1)^{x_s} (1 - b_s) (1 - t_{rs})^{x_r}
\end{equation}
%
We note here that this parametrization assumes the \acrlong{ici}, thereby allowing us to describe the model using only a few interpretable parameters. Dropping this assumption, a \gls{bn} can also be defined using \gls{cpt} that have columns for every possible combinations of parent states. However, with the increase of the number of parent nodes (causes) in the graph, the number of parameters in the respective \gls{cpt} would grow exponentially.

For the graph in \cref{fig:bn:graph} we can use \cref{eq:bn_cp} to write down the parametrized \gls{cpt} explicitly in the following manner:
%
\begin{equation} \label{eq:bn_cpt}
    \begin{aligned}
        P_\text{BN} \left( X_v = 0 \mid X_{\pa(v)} = 0 \right) &= 1 - b_v \\
        P_\text{BN} \left( X_v = 1 \mid X_{\pa(v)} = 0 \right) &= b_v \\
        P_\text{BN} \left( X_v = 0 \mid X_{\pa(v)} = 1 \right) &= (1 - b_v) \left(1 - t_{\pa(v)v}\right) \\
        P_\text{BN} \left( X_v = 1 \mid X_{\pa(v)} = 1 \right) &= 1 - (1 - b_v) \left(1 - t_{\pa(v)v}\right) 
    \end{aligned}
\end{equation}
%
In case of a more general network, in which some \glspl{lnl} receive efferent lymphatics from multiple other \glspl{lnl}, \cref{eq:bn_cpt} can be generalized and the conditional probability of the hidden state becomes
%
\begin{equation}
    \begin{multlined}[0.7\textwidth]
        P_\text{BN} \left( X_v = x_v \mid \big\{ X_r = x_r \,,\, t_{rv} \big\}_{r \in \pa(v)} , b_v \right) \\
        = x_v + (-1)^{x_v} (1 - b_v) \prod_{r \in \pa(v)}{\left( 1 - t_{rv} \right)}^{x_r}
    \end{multlined}
\end{equation}
%
We can now connect the probability of observing certain $Z_v^k$ given hidden involvement with the conditional probabilities above. Then the likelihood of observing a cohort of patients $\mathcal{Z} = \begin{Bmatrix} z_{nv}^k \mid n \leq N, v \leq V, k \in \mathcal{O} \end{Bmatrix}$ given a set of parameters $\theta = \begin{Bmatrix} b_v, t_{\pa{(v)}v} \mid v \leq V \end{Bmatrix}$ can be written as
%
\begin{equation} \label{eq:bn:likelihood}
    \begin{multlined}[0.85\textwidth]
        P_\text{BN}\left( \mathcal{Z} \mid \theta \right) =\\
        \prod_{n=1}^N{ \sum_{\mathbf{x} \in \{ 0,1 \}^V }{\prod_{v=1}^V{ \prod_{k \in \mathcal{O}}{ P_\text{BN} \left( z_{nv}^k \mid x_v \right) P_\text{BN} \left(  x_v \mid \{ x_{\pa{(v)}v} \}, \{ t_{pa{(v)}v} \}, b_v  \right) } } } }
    \end{multlined}
\end{equation}
%
where we marginalized over all hidden variables $X$. Here we have assumed that each patient's diagnosis $\mathbf{z} = \left( z_1 \quad z_2 \quad \cdots \quad z_V \right)$ is complete, meaning that we have a diagnosis for each \gls{lnl}. The likelihood can then be used to infer the model parameters via maximum likelihood inference or sampling.

\subsection*{Limitations of the Bayesian Network Model}
\label{subsec:previous_work:limitations}

While \glspl{bn} can model the probabilistic relationship between involvement in different levels, they lack an explicit way to describe the evolution of the tumor over time. The concept of \glspl{dbn} has been developed to introduce the notion of time into probabilistic models. \Glspl{dbn} are generalizations of \glspl{hmm} and formally similar to what we will introduce now. The metastatic spread in the lymphatic system is a dynamic system and by modeling it with a formalism that can capture this, we obtain a more intuitive model of the problem and a framework that can incorporate T-category into estimating the risk of \gls{lnl} involvement. We can do this because tumors go through the stages T1 to T4 sequentially, meaning that -- for a given tumor -- T-category is a surrogate of time.

\end{document}