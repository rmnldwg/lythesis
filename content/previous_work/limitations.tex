\providecommand{\relativeRoot}{../..}
\documentclass[\relativeRoot/main.tex]{subfiles}


\begin{document}

\section{Limitations}
\label{sec:previous_work:limitations}

While \glspl{bn} can model the probabilistic relationship between involvement in different levels, they lack an explicit way to describe the evolution of the tumor over time. The concept of \glspl{dbn} has been developed to introduce the notion of time into probabilistic models. \Glspl{dbn} are generalizations of \glspl{hmm} and formally similar to what we will introduce now. The metastatic spread in the lymphatic system is a dynamic system and by modeling it with a formalism that can capture this, we obtain a more intuitive model of the problem and a framework that can incorporate T-category into estimating the risk of \gls{lnl} involvement. We can do this because tumors go through the stages T1 to T4 sequentially, meaning that -- for a given tumor -- T-category is a surrogate of time.

\end{document}