\providecommand{\relativeRoot}{../..}
\documentclass[\relativeRoot/main.tex]{subfiles}
\graphicspath{{\relativeRoot/figures/}}

\begin{document}

\section{Abstract}
\label{sec:dataset:asbtract}

\subsection*{Purpose/Objective}

Whereas the prevalence of \gls{lnl} involvement in \gls{hnscc} has been reported, the details of lymphatic progression patterns are insufficiently quantified. In this study, we investigate how the risk of metastases in each \gls{lnl} depends on the involvement of upstream \glspl{lnl}, T-category, \gls{hpv} status and other risk factors.

\subsection*{Results}

We retrospectively analyzed patients with newly diagnosed \gls{opscc} treated at a single institution, resulting in a dataset of 287 patients. For all patients, involvement of \glspl{lnl} I-VII was recorded individually based on available diagnostic modalities (\gls{pet}, \gls{mri}, \gls{ct}, \gls{fna}) together with clinicopathological factors. To analyze the dataset, a web-based \gls{gui} was developed, which allows querying the number of patients with a certain combination of co-involved \glspl{lnl} and tumor characteristics.

\subsection*{Results}

The full dataset and \gls{gui} is part of the publication. Selected findings are: Ipsilateral level IV was involved in 27\% of patients with level II and III involvement, but only in 2\% of patients with level II but not III involvement. Prevalence of involvement of ipsilateral levels II, III, IV, V was 79\%, 34\%, 7\%, 3\% for early T-category patients (T1/T2) and 85\%, 50\%, 17\%, 9\% for late T-category (T3/T4), quantifying increasing involvement with T-category. Contralateral levels II, III, IV were involved in 41\%, 19\%, 4\% and 12\%, 3\%, 2\% for tumors with and without midline extension, respectively. T-stage dependence of \gls{lnl} involvement was more pronounced in \gls{hpv} negative than positive tumors, but overall involvement was similar. Ipsilateral level VII was involved in 14\% and 6\% of patients with primary tumors in the tonsil and the base of tongue, respectively.

\subsection*{Conclusions}

Detailed quantification of \gls{lnl} involvement in \gls{hnscc} depending on involvement of upstream \glspl{lnl} and clinicopathological factors may allow for further personalization of \gls{ctv-n} definition in the future.

\end{document}