\providecommand{\relativeRoot}{../..}
\documentclass[\relativeRoot/main.tex]{subfiles}
\graphicspath{{\relativeRoot/figures/}}

\begin{document}

\section{Introduction}
\label{sec:dataset:introduction}

\Gls{hnscc} spread through the lymphatic system of the neck and form metastases in regional lymph nodes. Therefore, the target volume in radiotherapy of \gls{hnscc} patients includes, in addition to the primary tumor, parts of the lymph drainage volume \cite{biau_selection_2019}, \cite{grosu_target_2015}. The nodal gross tumor volume \gls{gtv-n} contains detectable macroscopic lymph node metastases, while the elective clinical target volume \gls{ctv-n} contains parts of lymph drainage volume that is at risk of harboring microscopic tumor, i.e. occult metastases that are not yet visible with current imaging techniques.

\Gls{gtv-n} definition is primarily performed through imaging techniques (\gls{pet}-\gls{ct}/\gls{mri}, \gls{mri} or \gls{ct}) as well as \gls{fna}. Imaging criteria for lymph node metastases include size, round rather than oval shape, central necrosis, and FDG uptake as summarized by Biau et al \cite{biau_selection_2019}. Goel et al. gives an overview over clinical practice in \gls{pet}/\gls{ct} for the management of \gls{hnscc} \cite{goel_clinical_2017}. However, all imaging techniques have finite sensitivity and specificity \cite{park_diagnostic_2020}, \cite{jensen_imaging_2021}, \cite{rohde_18f-fluoro-deoxy-glucose-positron_2014}, i.e. they fail to detect small metastases or may incorrectly identify suspicious lymph nodes as tumor.

For standardized reporting of the location of lymph node metastases as well as delineation of the \gls{ctv-n}, the lymph drainage system of the neck is divided into anatomically defined regions called \gls{lnl} \cite{gregoire_delineation_2014}. \gls{ctv-n} definition amounts to the decision which \glspl{lnl} to include into the elective \gls{ctv-n} and is based on international consensus guidelines. Such guidelines were first published by Grégoire et al in 2000 and have been updated in 2006, 2014 and 2019 \cite{biau_selection_2019}, \cite{gregoire_delineation_2014}, \cite{gregoire_selection_2000}, \cite{gregoire_proposal_2006}. Current recommendations for the selection of lymph node levels in \gls{opscc} can be found in Table 2 of the guidelines published in 2019 by Biau et al. \cite{biau_selection_2019}. Current guidelines are primarily based on the prevalence of \gls{lnl} involvement for a given primary tumor location, i.e. the percentage of patients diagnosed with metastases in each level. It is recommended that the elective \gls{ctv-n} includes all \glspl{lnl} that are involved in 10–15\% of patients or more. Patients are primarily stratified by primary tumor location. For example, tumors of the soft palate, the posterior pharyngeal wall and the base of tongue show lymph node metastases on both sides via crossing lymph vessels. For this reason, even for lateralized tumors of these localizations, bilateral neck treatment is recommended. However, the lymphatic drainage of the tonsil is mainly unilateral, therefore an ipsilateral irradiation is recommended for lateralized low T-category (T1/T2) tumors (at least up to lymph node stage N2a). Volume-reduced elective nodal irradiation has been or is being investigated in several trials \cite{bratman_cctg_2020}, \cite{al-mamgani_contralateral_2017}.

While the general patterns of lymph drainage in the neck is understood and prevalence of \gls{lnl} involvement has been reported in the literature \cite{gregoire_selection_2000}, \cite{candela_patterns_1990}, \cite{iyizoba-ebozue_retropharyngeal_2020}, \cite{bauwens_prevalence_2021}, the details of progression patterns in \gls{opscc} are poorly quantified. How much does the risk of level IV involvement increase depending on whether levels II and III harbors macroscopic metastases? How much does the risk of involvement increase for late versus early T-category? Are progression patterns different for \gls{hpv} positive versus \gls{hpv} negative tumors? Answering these questions quantitatively may allow for further personalizing \gls{ctv-n} definition based on an individual patient's clinical presentation at the time of diagnosis.

The basis for better quantification of \gls{lnl} involvement are detailed datasets of \gls{hnscc} patients for whom involvement is reported per individual \gls{lnl} together with tumor and patient characteristics. For example, to answer the question of how much the risk in level IV increases depending on the involvement of upstream levels II and III, it is insufficient to only report the prevalence of \gls{lnl} involvement in levels II, III, and IV. Instead, the observed frequency of certain involvement combinations must be known, e.g. how often levels II, III and IV are involved simultaneously, versus how often only the levels II and III are involved without level IV. The contributions of this work are:

\begin{itemize}
    \item We provide a dataset of lymphatic progression patterns in 287 \gls{opscc} patients treated at our institution in whom involvement of \glspl{lnl} together with tumor characteristics are reported on a patient-individual basis.
    \item To visualize and explore the complex dataset, a graphical user interface is provided that allows the user to query the number of patients who were diagnosed with a specific combination of simultaneously involved \glspl{lnl} and tumor characteristics.
\end{itemize}

We hope that this work provides the basis for collecting large multicenter datasets of lymphatic progression patterns, which can then inform future guidelines on further personalized \gls{ctv-n} definition.

\end{document}