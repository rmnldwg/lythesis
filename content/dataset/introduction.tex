\providecommand{\relativeRoot}{../..}
\documentclass[\relativeRoot/main.tex]{subfiles}
\graphicspath{{\relativeRoot/figures/}}

\begin{document}

\subsection{Introduction}
\label{subsec:dataset:paper:introduction}

\Gls{hnscc} spread through the lymphatic system of the neck and form metastases in regional lymph nodes. Therefore, the target volume in radiotherapy of \gls{hnscc} patients includes, in addition to the primary tumor, parts of the lymph drainage volume \cite{biau_selection_2019}, \cite{grosu_target_2015}. The nodal gross tumor volume \gls{gtv-n} contains detectable macroscopic lymph node metastases, while the elective clinical target volume \gls{ctv-n} contains parts of lymph drainage volume that is at risk of harboring microscopic tumor, i.e.\ occult metastases that are not yet visible with current imaging techniques.

\Gls{gtv-n} definition is primarily performed through imaging techniques (\gls{pet}-\gls{ct}/\gls{mri}, \gls{mri} or \gls{ct}) as well as fine needle punctures (\gls{fna}). Imaging criteria for lymph node metastases include size, round rather than oval shape, central necrosis, and FDG uptake as summarized by Biau et al \cite{biau_selection_2019}. Goel et al. gives an overview over clinical practice in \gls{pet}/\gls{ct} for the management of \gls{hnscc} [3]. However, all imaging techniques have finite sensitivity and specificity [4], [5], [6], i.e. they fail to detect small metastases or may incorrectly identify suspicious lymph nodes as tumor.

\end{document}