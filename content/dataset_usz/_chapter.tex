\providecommand{\relativeRoot}{../..}
\documentclass[\relativeRoot/main.tex]{subfiles}
\graphicspath{
    {\subfix{./figures/}}
}

\begin{document}

\chapter[
    USZ Dataset on Lymph Node Involvement in OPSCC
]{
    Dataset on Lymph Node Involvement in \acrlong{opscc} Extracted at the \gls{usz}
}
\chaptermark{USZ Dataset on Involvement in OPSCC}
\label{chap:dataset_usz}

\begin{tcolorbox}[
    title=\faIcon{users} Contributions,
    parbox=false,
    float
]
    A \acrlong{gui} introduced in this chapter was conceptualized by \faIcon{user}~\textbf{Bertrand Pouymayou} \cite{pouymayou_analysis_2019}. He implemented a Python-based interface for local use. \faIcon{user}~\textbf{Roman Ludwig}~(me) then developed an online tool, freely accessible at \superhref{https://lyprox.org}{\faIcon{external-link-alt}~\texttt{https://lyprox.org}} based on this prototype using the web framework Django \cite{noauthor_django_2022} with substantially extended functionality.
\end{tcolorbox}

\clearpage

One critical aspect of our effort to model and predict the lymphatic tumor progression is the data we use to train the model. As previously explained, our model essentially consumes tables with rows of patients and columns involvement by \gls{lnl}. Data in this relatively simple format has been extracted in the past to create studies like \cite{candela_patterns_1990} or \cite{shah_patterns_1990}. However, the authors then used the data to compute statistics of it -- e.g. the prevalence of involvement -- but stopped short of publishing that data in its raw format. From these statistics it is -- with one exception \cite{sanguineti_defining_2009} -- usually not possible to reconstruct the correlations between the involvement of \glspl{lnl}.

With almost no usable data, of course, our methodology for modelling lymphatic progression cannot be tested or applied. So, we decided to start at the \gls{usz} to extract all patterns of lymphatic progression in patients with newly diagnosed \gls{opscc} between 2013 and 2019. We then not only used that data for inference on it, but also published it freely, hoping that other researchers might find it useful and that it may even motivate them to share their data in a similar fashion in the future.

\subfile{abstract}
\subfile{introduction}
\subfile{methods}
\subfile{results}
\subfile{discussion}

\end{document}
