\providecommand{\relativeRoot}{../..}
\documentclass[\relativeRoot/main.tex]{subfiles}


\begin{document}

\section{Material \& Methods}
\label{sec:dataset_usz:material}

\subsection*{Data curation}

We included patients diagnosed with \gls{opscc} (primary diagnosis) between 2013 and 2019 and treated at the department of radiation oncology and/or head and neck surgery of the \gls{usz}. Patients with prior radiotherapy or surgery to the neck were excluded, resulting in a dataset of 287 patients. Specific subsites of oropharyngeal cancer included the base of tongue, the tonsils as well as the oropharyngeal side of the vallecula and the posterior or lateral wall of the oropharynx. Patient information consisted of the date of birth, gender, the date of the 1st histological confirmation of the tumor, the performed treatment (surgery with neck dissection prior to RT/RCHT vs. surgery only vs. definitive radio(chemo)therapy), risk factors such as nicotine abuse and \gls{hpv}-status (p16 pos/neg), the TNM-classification (UICC 7th edition until 2017, 8th edition since 2017), the position of the primary tumor (left/right neck) as well as positive vs. negative mid-sagittal plane extension. Further details are described in the accompanying data-in-brief article \cite{ludwig_dataset_2021}.

The analysis of the lymphatic spread included levels Ia, Ib, IIa, IIb, III, IV, V, VII and was performed separately for the diagnostic imaging modalities available for a patient (FDG \gls{pet}-\gls{ct}, FDG \gls{pet}-\gls{mri}, \gls{mri}, \gls{ct}) as well as \gls{fna} and radiotherapy planning CT if available. This was performed by 2 experienced radiation oncologists by reviewing radiology and pathology reports together with the diagnostic images. Criteria for considering a lymph node as malignant followed the description in Biau et al \cite{biau_selection_2019} and are described in detail in the data-in-brief article \cite{ludwig_dataset_2021}.

\begin{tcolorbox}[title=\faIcon{database} Data, parbox=false]
    For the following experiments, we combined two datasets:

    \begin{itemize}[leftmargin=5.5mm]
        \item[\faIcon{hospital}] \textbf{USZ}: 287 patients with newly diagnosed \gls{opscc}, treated at the \acrlong{usz} between 2013 and 2019. This dataset has been described in great detail in \cref{chap:dataset}.
        \item[\faIcon{hospital}] \textbf{CLB}: 263 patients with newly diagnosed \gls{opscc}, treated at the \acrlong{clb} between 2014 and 2018. A table containing this data was kindly provided to us by \textbf{Vincent Grégoire}, who was also the \gls{pi} of the study during which this data was extracted and for which it was analyzed \cite{bauwens_prevalence_2021}.
    \end{itemize}

    All available diagnostic and pathological modalities reported in the data were used in combination with literature values for their respective sensitivity and specificity \cite{de_bondt_detection_2007} to arrive at an estimate for the most likely true pattern of involvement for each patient.

    This maximum likelihood estimate was then fed into the model, assuming a sensitivity and specificity of 1. See \cref{chap:sens_spec_analysis} for an intuition into our motivations for this simplification.
\end{tcolorbox}


\subsection*{Graphical User Interface}

We developed an online \gls{gui} based on the Django framework \cite{noauthor_django_2021} and provide it to explore the dataset. It allows the user to conveniently determine the number of patients that show a particular combination of co-involved \glspl{lnl} and tumor characteristics. The \gls{gui} is available at \url{https://2021-oropharynx.lyprox.org}; its source code under MIT license is available on GitHub at \url{https://github.com/rmnldwg/lyprox}. Documentation is provided within the \gls{gui}; a video demonstrating the use of the \gls{gui} is available \href{https://www.sciencedirect.com/science/article/pii/S0167814022000615#s0100}{online}.

\end{document}