\providecommand{\relativeRoot}{../..}
\documentclass[\relativeRoot/main.tex]{subfiles}

\begin{document}

\section{Discussion}
\label{sec:dataset_usz:discussion}

The purpose of this study was to provide detailed per-level-quantifications of cervical lymph node involvement for oropharyngeal carcinoma on a patient-individual basis, depending on T-category, involvement of other nodal levels, and various clinicopathological factors such as smoking and \gls{hpv} status. To the best of our knowledge, this is the only study providing such detailed quantitative information considering multimodal diagnostic modalities, which distinguishes this study from previous publications on the overall prevalence of \gls{lnl} involvement for oropharyngeal cancer. Furthermore, an elaborate user-friendly \gls{gui} is provided to visualize and explore the dataset and study the dependence of \gls{lnl} involvement as a function of the above parameters. While only selected information can be presented here in the form of tables and figures, the \gls{gui} can be used to access the full information contained in the dataset and study the influence of other factors such as primary tumor subsite or smoking status.

The main limitation of this dataset is that pathological involvement for the surgically treated patients was not available because neck dissection was performed en bloc. In addition, as a single-institution dataset, the number of patients is limited. However, dataset and \gls{gui} are made publicly available. The dataset can in the future be pooled with other datasets without loss of information, and the software platform and \gls{gui} developed may serve as the basis for collecting large multi-institutional datasets.

\subsection*{Comparison to Previous Works}

\subsubsection*{Prevalence of \gls{lnl} Involvement}

Overall patterns and prevalence of \gls{lnl} involvement in our study (\cref{table:dataset_usz:prevalence}) is consistent with previous studies \cite{candela_patterns_1990,gregoire_selection_2000,iyizoba-ebozue_retropharyngeal_2020}. Our study contained a relatively low number of N0 patients (16\%) compared to previous reports, which may be explained by differences in patient selection and diagnostic modalities used. Our study includes all patients treated at our institution between 2013 and 2019 irrespective of primary treatment. Hence, our patient cohort may be considered relatively unbiased. Studies reporting on surgically treated patients may introduce bias towards lower prevalence of \gls{lnl} involvement when patients with advanced disease are referred to definitive chemoradiotherapy.

\subsubsection*{Dependence on Upstream Levels}

The question on how the probability of metastases in a \gls{lnl} depends on the involvement of upstream levels is poorly reported in the literature. To our knowledge, Sanguinetti et al \cite{sanguineti_defining_2009} is the only publication reporting on this question for early T-category surgically treated \gls{opscc}. For example, out of 42 patients with ipsilateral level III involved, 12 patients (29\%) had also level IV involved \footnote{These numbers are reconstructed from the data reported but are not directly contained in the publication.
}, which is similar to our findings (28 out of 113 patients, 25\%, for all patients combined). In agreement with previous studies, our dataset confirms that skip metastases in levels III and IV occur but are rare (\cref{table:dataset_usz:upstream}). Furthermore, we observed no case of level I or V involvement without metastases in level II, which is also confirming previous publications. Further data collection and analysis in that direction could potentially lead to treatment-de-escalation strategies by not irradiating down-stream \glspl{lnl} in the absence of metastases in up-stream levels, e.g. by identifying patients in whom level IV may be excluded from the \gls{ctv-n}.

\subsubsection*{Contralateral Spread}

A prominent example of treatment de-escalation is the sparing of the elective contralateral irradiation or the pN0, negative, neck: Chronowski et al. \cite{chronowski_unilateral_2012} provided data of 102 patients with tonsillar carcinoma treated with unilateral radiotherapy, of which only 2\% experienced contralateral recurrence. Very similar data, with contralateral recurrence rates of only 2-3.5\% were reported from the Princess Margaret Hospital \cite{huang_re-evaluation_2017,osullivan_benefits_2001}. Moreover, similarly to the results of the large meta-analysis of Al-Mamgani \cite{al-mamgani_contralateral_2017}, we could demonstrate that the probability of contralateral involvement also strongly depends on T-category and midline extension. Concerning omission of radiotherapy to the pN0 neck, a recent prospective trial, with most of the patients included suffering from oropharyngeal cancer, could demonstrate excellent tumor control rates of 97\% in the unirradiated neck \cite{contreras_eliminating_2019}. These results show that \gls{ctv-n} reduction is possible for selected patients. However, according to current guidelines, unilateral radiotherapy is recommended in specific circumstances. In our study, incidence of contralateral involvement was 20\% and the data suggests that for many patients with favorable characteristics (early T-category, no midline extension, limited ipsilateral involvement), the risk of contralateral metastases is low (\cref{table:dataset_usz:contralateral}). If supported by further multi-institutional data, this could identify additional patients in whom the contralateral neck may be safely excluded from the \gls{ctv-n}, either completely or in part. E.g., radiotherapy could be limited to level II in some patients when level II still bears a relevant risk of occult metastases but the risk in levels III and IV is minimal.

\subsubsection*{\Gls{hpv} Status}

Consistent with the findings of Bauwens et al \cite{bauwens_prevalence_2021} and the general clinical observation that \gls{hpv}-positive tumors seem to metastasize early despite small primaries, our data suggests that the dependence of lymph node involvement on T-category is less pronounced in \gls{hpv}-positive tumors (\cref{table:dataset_usz:upstream}). Beyond that, our data does not provide evidence that lymphatic progression patterns differ between \gls{hpv}-positive versus negative tumors (consistent with Bauwens et al \cite{bauwens_prevalence_2021}).

\subsection*{From Macroscopic Progression Patterns\\to Microscopic Involvement}

In this work, we consider lymphatic progression patterns assessed through imaging. Hence, the distribution of macroscopic lymph node metastases is studied. For defining the elective \gls{ctv} or the extent of surgical resection, we are instead interested in the conditional probability of microscopic involvement in a \gls{lnl} given that no macroscopic metastases are seen in that level. This risk depends on two aspects: (1) The sensitivity and specificity of diagnostic imaging, i.e. the probability of not diagnosing lymph node metastases that are present; and (2) The probability that tumor cells have spread to a \gls{lnl}, given the observed state of tumor progression for that patient. The latter can be obtained from datasets of metastatic progression patterns in a cohort of patients as presented in this paper. Statistical methods that combine both aspects to calculate risk of microscopic involvement have been developed for ipsilateral levels I-IV \cite{pouymayou_bayesian_2019,ludwig_hidden_2021}. Future work will extend these statistical models to contralateral spread and levels V and VII, informed by the data presented here. However, this is not part of the current paper.

\subsection*{Summary and Prospect}

Detailed datasets of lymphatic progression patterns, meaning reporting the combination of simultaneously involved \glspl{lnl} together with tumor characteristics on a patient-individual basis, allows for better quantification of \gls{lnl} involvement. This may in turn allow for further personalization of elective \gls{ctv-n} definition based on the individual patient's state of tumor progression. In this paper we publish such a dataset together with a graphical user interface to explore the dataset. The software tools are made publicly available for others to study our dataset in detail and to contribute data for building larger multi-institutional datasets. Large datasets, possibly containing thousands of patients, together with the statistical methods for analysis, may eventually inform future clinical trials and guidelines on nodal \gls{ctv} definition in head \& neck cancer. Potential applications are to omit irradiation of ipsilateral level IV in selected patients, or to identify additional patients in whom contralateral neck irradiation can be omitted or limited to level II.

\end{document}