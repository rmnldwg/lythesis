\providecommand{\relativeRoot}{../..}
\documentclass[\relativeRoot/main.tex]{subfiles}
\graphicspath{\relativeRoot/figures/}


\begin{document}

\section{Continuous Variables as Risk Factors}
\label{sec:future:continuous}

Every observed quantity that went into the presented models so far was a binary (observations of \gls{lnl} involvement, extension over mid-plane) or categorical variable (T-category). However, especially the variable of T-category is -- according to its definition \cite{brierley_tnm_2017} -- largely a categorization of tumor size, which is a continuous variable. To date, no dataset reporting this quantity is available to us. But it is often recorded when \acrlong{rt} is chosen as treatment, as the primary tumor is carefully delineated on a 3D scan prior to treatment planning. From this \gls{gtv} delineation in the treatment planning software it is usually straightforward to compute e.g. the volume of the tumor.

Our assumption when introducing the marginalization over diagnosis times in \cref{subsec:unilateral:formalism:tstage} was that tumor size at the time of diagnosis correlates with the time a tumor was able to spread lymphatically, and that T-category is a surrogate of tumor size. Based on this, we should also be able to directly use the continuous variable of tumor size and define a conditional probability over the diagnosis time $\probofgiven{t}{V_\text{PT}, \gamma}$, given the volume of a primary tumor $V_\text{PT}$ and some learnable parameters $\gamma$. These parameters may control how the volume $V_\text{PT}$ affects the distribution over the diagnosis time $t$.

Another possibly interesting continuous variable to include in the model may be results from \gls{fdg-pet}. In this imaging technique, the radiopharmaceutical \gls{fdg} is injected into a patient and shortly after, a \acrshort{pet}-\acrshort{ct} scan is taken. \Gls{fdg} follows a similar metabolic pathway to normal glucose and since tumor cells show an increase in glucose uptake, they in particular accumulate \gls{fdg}. The \acrshort{pet}-\acrshort{ct} scan can then detect, quantify and localize the metabolism of \gls{fdg} as it decays radioactively inside cells with a high glucose uptake. The so-called \gls{suv}, a normalized value of tumor uptake, is the most commonly used quantitative output of \gls{fdg-pet}. One could now theorize that tumors and metastases with a high uptake of \gls{fdg} also proliferate faster and hence cause faster metastasis overall. Under this assumption, if the \gls{suv} was recorded per \gls{lnl}, we could use it as a continuous variable that correlates with the respective \gls{lnl}'s probability rate of metastatic spread. However, more data and further investigation is necessary to test this hypothesis before beginning to incorporate such a variable into our predictive model.

\end{document}
