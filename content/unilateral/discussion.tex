\providecommand{\relativeRoot}{../..}
\documentclass[\relativeRoot/main.tex]{subfiles}
\graphicspath{{\subfix{./figures/}}}

\begin{document}

\section{Discussion}
\label{sec:unilateral:discussion}

In this chapter, we presented a probabilistic model based on \gls{hmm} for predicting the lymphatic progression of \gls{hnscc} through a patient's \glspl{lnl}. The model allows for estimating risk of microscopic \gls{lnl} involvement, given two patient-specific diagnostic observations:
\begin{enumerate*}[label={(\arabic*)}]
    \item imaging information on the location of macroscopic metastases, and
    \item T-category
\end{enumerate*}. The first aspect has been addressed in a previous publication, and we showed that the predictions of the new \gls{hmm}-based model agree with the previously published \gls{bn}-based model when given the same training data. The \gls{hmm}-based model adds the capability to include T-category into the assessment of \gls{lnl} involvement risk by modeling the transitions between different states of nodal progression over discrete time-steps. This assumes that for a given tumor T-category is a surrogate of time and that primary tumor growth and metastatic spread occur alongside and are hence correlated. Late T-category tumors are on average diagnosed in a later phase of their disease than early T-category tumors, patients are consequently more likely to be in a more advanced state of nodal progression, which in turn increases the risk of microscopic involvement of \gls{lnl} -- an intuition that can be quantified by the presented model. Also, the model assumes the pathways of lymphatic spread to stay the same throughout the evolution of the disease, which is probably not true for all patients, especially when presenting with very advanced tumor stages.

To the best of our knowledge, it has not been investigated how much time passes between tumor formation and diagnosis and how this varies with T-category. Although, this may initially appear as a problem, it is surprisingly not relevant -- although interesting -- how much time passes in the real world between two time-steps in the model. The model does not even assume that this time per time-step remains constant. It could, for example, become progressively shorter for later time-steps, accounting for the fact that a more advanced tumor also spreads faster. The time-prior's exact shape however is harder to determine. This distribution gives, by definition, the probability to diagnose a patient after $t$ time-steps, given their T-category. However, it can be shown empirically, that support and exact shape of the time-prior have no or limited impact on the model predictions.

There have been two other studies \cite{benson_markov_2006,jung_development_2016} from the same group that looked into modelling lymphatic metastatic progression in head and neck cancer using Markov models. The authors in those works, too, express that the length of a time-step is abstract and not necessary for modelling. The first study \cite{benson_markov_2006} differs from the work presented here in that it models an \gls{lnl}'s state not as binary, but as a categorical variable taking on values between 0 and 4, indicating different states of involvement. With the increase of an \gls{lnl}'s state, the probability of spread to the next \gls{lnl} increases too. This is an interesting idea that could potentially be incorporated into our methodology as well. A shortcoming of their approach is that they assume all \glspl{lnl} to have the same probabilities of evolving and metastasizing, and they are not learned, but arbitrarily fixed. Also, T-category enters the model only via the number of time-steps the model is run for and the state a patient is ultimately in is modelled as observable, not hidden. The second work \cite{jung_development_2016} models T-category explicitly as a random variable and the involvement of all \glspl{lnl} along a chain up to a certain \gls{lnl} as binary. It is not modelled as hidden and the probabilities for progressing to the next T-category are constant, as well as the probabilities for the involvement to spread further down the chain.

The methodology presented here may be used to inform future guidelines on elective nodal \gls{ctv} definition or the extent of surgical resection. However, to do so, learning of the model parameters must be based on larger training datasets of lymphatic progression patterns than the one we reconstructed from \cite{sanguineti_defining_2009}. At the time of writing this work \cite{ludwig_hidden_2021}, there was a lack of available training data in the form necessary for the model, which requires a table with rows of patients and columns of patient information containing T-category, whether each individual \gls{lnl} was involved, and possibly additional risk factors that potentially have impact on nodal progression. Such data is routinely acquired in clinical practice and can be anonymized for sharing without substantial hurdles regarding patient data confidentiality. However, it is rarely published. Many studies only report prevalence of \gls{lnl} involvement \cite{lindberg_distribution_1972,woolgar_histological_1999,candela_patterns_1990,vauterin_patterns_2006,ho_patterns_2012,shah_patterns_1990,razfar_incidence_2009,woolgar_topography_2007,chung_pattern_2016} but omit detailed individual reports on the patterns of involvement, i.e. which \glspl{lnl} were simultaneously involved. Although prevalence data can be incorporated into our model as a special case of incomplete observations (\cref{subsec:unilateral:formalism:incomplete_diagnose}), it is not helpful for addressing the question of how the location of macroscopic metastases impacts the risk of microscopic disease in other \gls{lnl}. At the university hospital Zurich, we have collected and curated such a dataset to consolidate risk predictions for ipsilateral levels I-IV and to further extend the model \cite{ludwig_detailed_2021,ludwig_dataset_2021}, as described in \cref{chap:dataset_usz}.

Larger data sets allow us to extend the model to include:
\begin{enumerate*}[label={(\arabic*)}]
    \item additional \glspl{lnl} such as levels V and VII. This corresponds to extending the graph and thereby the set of parameters. Since these levels are more rarely involved, larger datasets for training are required (see \cref{chap:graph}).
    \item other tumor locations in the head and neck region such as hypopharynx, larynx, and oral cavity. Intuitively one may expect that different primary tumor locations mainly mean different base probability rates $\tilde{b}$ while the transition probability rates $\tilde{t}$ remain similar, since they depend on lymphatic drainage between levels rather than the primary tumor location. However, only larger datasets will answer this question. Multiple tumor sites can also be incorporated into our graph-based approach with relative ease.
    \item contralateral spread accounting for patient-specific observations such as midline extension of the primary tumor. Here too, one may expect the transition probability rates to remain similar between ipsilateral and contralateral side while the contralateral base probabilities are lower depending on the lateralization of the primary tumor.
    \item Beyond changing the graph structure and its parameters, we would also like to include other risk factors such as \gls{hpv} status, age, alcohol and nicotine abuse etc. into the model at some point in the future.
    \item Apart from \gls{hnscc}, the methodology presented here may also be applied to calculate probabilities of lymphatic spread in other cancer sites such as breast or advanced stage prostate cancer.
\end{enumerate*}

In conclusion, this chapter presents an interpretable probabilistic model to describe lymphatic tumor progression over time, which incorporates both the anatomy of the lymphatic drainage system and clinical data on lymph node involvement. It extends previous work on estimating the risk of microscopic involvement in \acrlongpl{lnl} by incorporating T-category as an additional risk factor. When provided with larger and more diverse datasets, the model may support clinicians in making \gls{ctv-n} definition more objective and personalized.

\end{document}
