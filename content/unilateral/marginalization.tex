\providecommand{\relativeRoot}{../..}
\documentclass[\relativeRoot/main.tex]{subfiles}


\begin{document}

\section{Marginalization}
\label{sec:unilateral:marginalization}

To calculate the likelihood function, we have to calculate the probability of a given diagnostic observation. To that end, we first calculate the probability of observing a given diagnosis $\mathbf{z} = \boldsymbol{\zeta}_j$ at a fixed time-step $t$. As depicted in \cref{fig:unilateral:hmm_paths}, we must consider every possible evolution of a patient's disease that leads to the observed diagnosis. Mathematically, this means that we need to marginalize over all such paths. And here is where the HMM-formalism comes in very useful, because this marginalization happens automatically when we multiply the transition matrix with itself and eventually with the observation matrix:
%
\begin{equation} \label{eq:unilateral:marginalization:cond_time}
    P \left( \mathbf{Z} = \boldsymbol{\zeta}_j \mid t \right) = \left[ \boldsymbol{\pi}^\top \cdot (\mathbf{A})^t \cdot \mathbf{B} \right]_j
\end{equation}
%
where the $\boldsymbol{\pi}$ is the column vector for the healthy starting state. $\mathbf{A}$ is multiplied with itself $t$ times and thereby produces a matrix that describes the transition probability from the healthy state to all possible states $\mathbf{x}[t]$ in exactly $t$ time-steps marginalized over the actual pathway of the patient’s disease. The index $[\ldots]_j$ here means that from the resulting (row-)vector of probabilities we take the component that corresponds to the diagnose $\mathbf{z} = \boldsymbol{\zeta}_j$.

So, essentially, \cref{eq:unilateral:marginalization:cond_time} first computes the probability vector of all possible true hidden states, given a time step $t$
%
\begin{equation} \label{eq:unilateral:marginalization:risk_cond_time}
    P \left( \mathbf{X} = \boldsymbol{\xi}_i \mid t \right) = \left[ \boldsymbol{\pi}^\top \cdot (\mathbf{A})^t \right]_i
\end{equation}
%
and then multiplies it with the respective observation probability vector, which is a column of the $\mathbf{B}$ matrix, to finally marginalize over all possible true hidden states -- effectively a sum over $i$ in \cref{eq:unilateral:marginalization:risk_cond_time} -- at the time $t$ of diagnosis.

The problem that the number of time-steps until diagnosis is unknown cannot be solved in such an elegant fashion. Therefore, we must resort to brute force marginalization and introduce a prior $p(t)$, which is a discrete distribution over a finite number of time-steps. It describes the prior probability that a patient's cancer is diagnosed at a particular time-step $t$. To get the probability of a diagnosis $\mathbf{z}$ we must compute
%
\begin{equation} \label{eq:hmm_marginalize}
    P\left( \mathbf{Z} = \boldsymbol{\zeta}_j \right) = \sum_{t = 0}^{t_\text{max}}{p(t) \cdot P\left( \mathbf{Z} = \boldsymbol{\zeta}_j \mid t \right)} = \left[ \sum_{t = 0}^{t_\text{max}}{p(t) \cdot \boldsymbol{\pi}^\top \cdot (\mathbf{A})^t \cdot \mathbf{B}} \right]_j
\end{equation}
%
While the choice of the time-prior may seem unclear at this point, its role for including T-stage into this model will be discussed in \cref{sec:unilateral:tstage}.

\end{document}