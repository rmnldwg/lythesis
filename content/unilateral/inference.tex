\providecommand{\relativeRoot}{../..}
\documentclass[\relativeRoot/main.tex]{subfiles}


\begin{document}

\section{Inference of model parameters}
\label{sec:unilateral:inference}

In the formalism of the last sections, the $P_{HMM}$ depends implicitly through $P_{BN}$ on parameters $\theta = \left\{ \Tilde{b}_v , \Tilde{t}_{pv} \mid v \leq V , p \in \pa(v) \right\}$, which – as mentioned – are now probability rates and have therefore a slightly different interpretation. Due to the marginalization over time-steps in \cref{eq:hmm_marginalize} the likelihood function additionally depends on the choice and parametrization of the prior $p(t)$. The parameters are to be inferred from a dataset of lymphatic progression patterns in a cohort of patients. We still assume that for each patient we record for every \gls{lnl} $v$ whether it is involved according to only one diagnostic modality. In other words, for each patient we observe one of the $2^V$ possible diagnoses. As mentioned before, we will expand this to multiple diagnostic modalities furhter down in \cref{subsec:hmm_unilateral_implementation}.

Formally, we can then express the dataset $\boldsymbol{\mathcal{Z}}$ of $N$ patients as vector $\mathbf{f}$ of the number of patients $f_i$ for which the diagnosis corresponds to the observational state $\boldsymbol{\zeta}_i$. The likelihood $P \left( \boldsymbol{\mathcal{Z}} \mid \theta \right)$ of observing this dataset, given a particular choice of parameters, is then given by
%
\begin{equation}
    P \left( \boldsymbol{\mathcal{Z}} \mid \theta \right) = \prod_{i=1}^{2^V}{P \left( \boldsymbol{\zeta}_i \mid \theta \right)^{f_i}}
\end{equation}
%
with the probability $P \left( \boldsymbol{\zeta}_i \mid \theta \right)$ specified by \cref{eq:hmm_marginalize}. The product runs formally over all possible observational states. In reality, $f_i$ will likely be zero for a number of rare or implausible states that are not in the dataset. Note that $\sum_{i}{f_i} = N$.

By Bayes' rule, the posterior distribution of those parameters is 
%
\begin{equation} \label{eq:unilateral:inference:bayes_theorem}
    P \left( \theta \mid \boldsymbol{\mathcal{Z}} \right) = \frac{P \left( \boldsymbol{\mathcal{Z}} \mid \theta \right) P\left( \theta \right)}{\int{P \left( \boldsymbol{\mathcal{Z}} \mid \theta' \right) P \left( \theta' \right) \,d\theta'}}
\end{equation}
%
where $P(\theta)$ is the prior over these parameters. Since they are exclusively probability rates, they must all come from the interval $[0,1] \in \mathbb{R}$. In this work we will choose the most uninformative prior
%
\begin{equation}
    p(\theta) = 
    \begin{cases}
        1 & \text{if} \ \ \theta_r \in \left[ 0,1 \right]; \forall r \leq E \\
        0 & \text{otherwise}
    \end{cases}
\end{equation}
%
where $E$ is the number of edges in the \gls{dag} we use to represent the lymphatic system. While it is easy to compute the likelihood, it is not feasible to efficiently calculate the normalization constant in the denominator of \cref{eq:unilateral:inference:bayes_theorem}. Hence, we will use \gls{mcmc} sampling methods to estimate the parameters $\theta$ and their uncertainty.

\end{document}