\providecommand{\relativeRoot}{../..}
\documentclass[\relativeRoot/main.tex]{subfiles}


\begin{document}

\section{Formulating lymphatic progression as HMM}
\label{sec:unilateral:formulation}

We consider discrete time-steps $t \in \{ 0, 1, 2, \ldots, t_\text{max} \}$. We will start by defining the hidden random variable for the state of the \gls{hmm} at time $t$ to be
%
\begin{equation}
    \mathbf{X}[t] = \left( X_v[t] \right)
\end{equation}
%
which represents the patient’s state of \gls{lnl} involvement as in the \gls{bn}, but for each time-step we have an instance of it. For the diagnosis $\mathbf{Z}$ on the other hand, we do not need to differentiate between different times, since in practice we will only ever see one diagnosis. This is illustrated in \cref{fig:hmm_schema}. The reason for this is that, if we diagnose a patient with cancer, treatment starts timely and we no longer observe the natural progression of the disease. From a modeling standpoint however, this is a problem that we will address later.

A hidden Markov model is fully described by the starting state $\mathbf{X}[0] \coloneqq \boldsymbol{\pi}$ and the two conditional probability functions that govern the progression from a state $X[t]$ at time $t$ to a state $X[t+1]$ at the following time-step
%
\begin{equation}
    P_{HMM}\left( \mathbf{X}[t+1] \mid \mathbf{X}[t] \right)
\end{equation}
%
and the probability of a diagnostic observation given the true state of the patient
%
\begin{equation}
    P_{HMM}\left( \mathbf{Z} \mid \mathbf{X}[t] \right)
\end{equation}
%
Since both our state space and our observation space are discrete and finite, it is possible to enumerate all possible states and observations and collect them in a table or matrix. This so-called \emph{transition matrix} would then be
%
\begin{equation}
    \mathbf{A} = \left( A_{ij} \right) = \left( P_{HMM} \left( \mathbf{X}[t+1] = \boldsymbol{\xi}_i \mid \mathbf{X}[t] = \boldsymbol{\xi}_j \right) \right)
\end{equation}
%
and the \emph{observation matrix}
%
\begin{equation}
    \mathbf{B} = \left( B_{ij} \right) = \left( P_{HMM} \left( \mathbf{Z} = \boldsymbol{\zeta}_i \mid \mathbf{X}[t] = \boldsymbol{\xi}_j \right) \right)
\end{equation}
%
Here, $\boldsymbol{\xi}_i$ and $\boldsymbol{\zeta}_j$ are no new variables, but just $\mathbf{x}$ and $\mathbf{z}$ renamed and reordered. The indices $i$ and $j$ for one of the possible states or observations for the entire patient, not for an individual \gls{lnl}. In total, there are $S = |\{ 0,1 \}|^V$ different states and the same number of different possible observations per diagnostic modality. We order the hidden states from
%
\begin{equation}
    \boldsymbol{\xi}_1 = 
    \begin{pmatrix}
        0 & 0 & 0 & 0
    \end{pmatrix}
\end{equation}
%
to
%
\begin{equation} \label{eq:unilateral:formulation:obs_matrix}
    \boldsymbol{\xi}_{16} = 
    \begin{pmatrix}
        1 & 1 & 1 & 1
    \end{pmatrix}
\end{equation}
%
in this case of $V = 4$. The exact ordering does not matter, it is just a convenience for the notation. our ordering of the states can be seen in the axes of \cref{fig:unilateral:trans_matrix}. In analogy, we order the observations $\boldsymbol{\zeta}_j$ from 1 to $2^V$. Note that for now we will not consider multiple diagnostic modalities and how to combine them. We will get back to that topic in \cref{subsec:hmm_unilateral_implementation}.

\begin{figure}
    \centering
    \def\svgwidth{0.8\textwidth}
    \input{figures/transition_matrix.pdf_tex}
    \caption{Transition matrix. All gray pixels in this image correspond to
    entries in the matrix being zero. The colored pixels take on values $\in [0, 1]$
    which are here overlaid in \%. The exact values stem from the mean of the
    learned parameters in \cref{sec:unilateral:application}. The
    exact shape of the gray ``mask'' depends on how one orders the states.}
    \label{fig:unilateral:trans_matrix}
\end{figure}

In our case, the starting state corresponds to a primary tumor being present but all \glspl{lnl} are still in the healthy state. The observation matrix $\mathbf{B}$ is specified via sensitivity and specificity as described in \cref{eq:unilateral:formulation:obs_matrix}. The main task is to infer the transition matrix $\mathbf{A}$. Usually, it is inferred from a series of observations and there exist efficient algorithms for that, e.g. the sum-product algorithm, which is particularly efficient in chains. Unfortunately, these algorithms cannot be applied for our problem for two profound reasons:

\begin{enumerate}
    \item We only have a single observation instead of a consecutive series of observations. 
    \item It is unclear how many time-steps it took from the starting state to the one observation we have at the time of diagnosis.
\end{enumerate}

In the remainder of this section, we will detail the \gls{hmm} step-by-step, starting with the parameterization of the transition matrix $\mathbf{A}$ in \cref{sec:unilateral:parametrization}. Afterwards, in \cref{sec:unilateral:marginalization}, I will tackle the aforementioned problems, followed up by explaining how we perform inference on this model (\cref{sec:unilateral:inference}), incorporate information about a patient’s T-stage (\cref{sec:unilateral:tstage}) and assess the risk of \gls{lnl} involvement in a new patient (\cref{sec:unilateral:risk_assessment}). Lastly, we will introduce a way to incorporate incomplete observations in \cref{sec:unilateral:incomplete_diagnose}.

\end{document}