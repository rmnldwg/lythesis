\begin{tcolorbox}[
    title=\faIcon{recycle} Reproducibility,
    parbox=false,
    float
]
    All results of the section below are reproducible using a notebook we have published within the \repolink{lymph} repository at the time of submitting \cite{ludwig_hidden_2021}. The state of the repository at that time was just recently updated to make the notebook's reproduction a bit more user-friendly:

    \begin{enumerate}
        \item Clone the \repolink{lymph} repository and checkout the tagged version \superhref{https://github.com/rmnldwg/lymph/tree/0.1.1}{\faIcon{tags} 0.1.1}:
        \begin{itemize}[leftmargin=10mm]
            \setlength\itemsep{-0.5em}
            \item[\texttt{\$}] \verb|git clone https://github.com/rmnldwg/lymph|
            \item[\texttt{\$}] \verb|cd lymph|
            \item[\texttt{\$}] \verb|git checkout 0.1.1|
        \end{itemize}

        \item Set up a virtual environment that supports installing packages using \superhref{https://pypi.org/project/pip/}{\texttt{pip}}. I recommend \superhref{https://docs.python.org/3/library/venv.html}{\texttt{venv}}:
        \begin{itemize}[leftmargin=10mm]
            \setlength\itemsep{-0.5em}
            \item[\texttt{\$}] \verb|python3.8 -m venv .venv|
            \item[\texttt{\$}] \verb|source .venv/bin/activate|
            \item[\texttt{\$}] \verb|pip install -U pip setuptools wheel|
        \end{itemize}

        \item Install the necessary packages:
        \begin{itemize}[leftmargin=10mm]
            \setlength\itemsep{-0.5em}
            \item[\texttt{\$}] \verb|pip install -r requirements.txt|
        \end{itemize}

        \item Open the Jupyter notebook:
        \begin{itemize}[leftmargin=10mm]
            \setlength\itemsep{-0.5em}
            \item[\texttt{\$}] \verb|cd notebook|
            \item[\texttt{\$}] \verb|jupyter notebook results_and_plots.ipynb|
        \end{itemize}
    \end{enumerate}

    Now, it should be possible to execute the cells of the notebook in order. The inference precesses will likely run for a while, but should continuously report its progress. Subsequent cells will produce the figures shown in this chapter and store them inside the \faIcon{folder} \texttt{figures} folder relative to the notebook.
\end{tcolorbox}
