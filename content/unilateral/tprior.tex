\providecommand{\relativeRoot}{../..}
\documentclass[\relativeRoot/main.tex]{subfiles}
\graphicspath{{\subfix{./figures/}}}


\begin{document}

\section[Impact of Shape and Length of the Time-Prior]{Impact of Shape and Length\\of the Time-Prior}
\label{sec:unilateral:tprior}

It turns out that length and shape of $p(t)$ have almost no effect on the risk predictions as long as we are not concerned with different T-categories. So, if we learn our parameters from a dataset that only contains T1 patients and then compute risks for T1 patients only, the result will not differ almost regardless of the time-prior that was used for learning and risk assessment. Only too few time-steps may pose a problem, since then the system might not be able to spread to all \glspl{lnl} via all pathways. And too many time-steps could introduce numerical problems, because the learned probability rates $\tilde{b}_v$ and $\tilde{t}_{\pa(v)v}$  become smaller for longer time-priors.

\begin{figure}
    \centering
    \def\svgwidth{0.7\textwidth}
    \input{figures/simple_rate_decay.pdf_tex}
    \caption[Analytic solutions for the decay of spread probability rates]{Solutions to \cref{eq:unilateral:tprior:analytical_result} for the base probability rate $\tilde{b}_1$ given a $p^{\star}$ of 0.4 and $T$ increasing from 1 to 15.}
    \label{fig:unilateral:tprior:rate_decay_analytic}
\end{figure}

To understand the impact of the number of time-steps $T$ on the results, we looked at a simple analytical model: Assume that there is a system with only one \gls{lnl} that the primary tumor can spread to that is empirically involved with probability $p^{\star} = 0.4$. For this situation, we can now derive how the base probability rate $\tilde{b}_v$ changes for a uniform time-prior
%
\begin{equation}
    p(t) = \frac{1}{T} \qquad \text{for} \quad t \in \{ 1, 2, \ldots, T \}
\end{equation}
%
if we vary the total number of time-steps $T$. We can write $p^{\star}$ as
%
\begin{equation}
    \begin{aligned}
        p^{\star} &= \frac{1}{T} \sum_{t=1}^T{
            \begin{pmatrix}
                1 & 0
            \end{pmatrix}
            \cdot
            \begin{bmatrix}
                (1 - \tilde{b}_1) & \tilde{b}_1 \\
                0 & 1
            \end{bmatrix}^t
            \cdot
            \begin{pmatrix}
                0 \\
                1
            \end{pmatrix}
        } \\
        &= \frac{1}{T} \sum_{t=1}^T{
            \begin{pmatrix}
                1 & 0
            \end{pmatrix}
            \cdot
            \begin{bmatrix}
                (1 - \tilde{b}_1)^t & 1 - (1 - \tilde{b}_1)^t \\
                0 & 1
            \end{bmatrix}
            \cdot
            \begin{pmatrix}
                0 \\
                1
            \end{pmatrix}
        } \\
        &= \frac{1}{T} \sum_{t=1}^T{ \left[ 1 - (1 - \tilde{b}_1)^t \right] }
        = 1 - \frac{1}{T} \sum_{t=1}^T{(1 - \tilde{b}_1)^t}
    \end{aligned}
\end{equation}
%
The right-hand side essentially contains the partial sum of the geometric series and can easily be computed to yield
%
\begin{equation} \label{eq:unilateral:tprior:analytical_result}
    p^{\star} = 1 - \frac{\left(1 - \tilde{b}_1\right) \left( 1 - (1 - \tilde{b}_1)^T \right)}{\tilde{b}_1 T}
\end{equation}
%
\begin{figure}
    \centering
    \def\svgwidth{1.0\textwidth}
    \input{figures/rate_decay_theory_vs_sampled.pdf_tex}
    \caption[
        Decay of spread probabilities for different time-step supports
    ]{
        Decay of base probability rates as a function of the number of time-steps the time-prior has. Circles depict the results from learning the same dataset with different time-priors while solid lines show the analytical result starting with a $p^{\star}$ corresponding to the prevalence of involvement of \gls{lnl} I and II respectively.
    }
    \label{fig:unilateral:tprior:rate_decay_sampled}
\end{figure}

It is not possible to analytically solve for $\tilde{b}_1$ in the case of arbitrary $T$, but numerical solutions are very easy to find and are plotted in \cref{fig:unilateral:tprior:rate_decay_analytic}. This confirms the intuition, that the base and transition probability rates become smaller when the total time over which the tumor spreads is divided into more but shorter time-steps.

Now we compare this idealized result to the decay of the probability rates for the full system. To that end, the model with \glspl{lnl} I-IV was trained as in the same way as for the figures in the section above, but with differently long uniform time-priors instead of a Binomial prior. So, the probability for every time-step is $p(t)=1/T$ for all $t \geq 1$, but zero for the starting state $\pi$. \cref{fig:unilateral:tprior:rate_decay_sampled} shows the expected value of the parameters as a function of $T$. It is important to stress again that the risk predicted by the models using all those different-length uniform time-priors was the same for $T \geq 2$. For the one-step model with $T=1$ the risk prediction of a \gls{lnl} does not depend on the diagnose. For example, we expect the risk in level III is higher, when level II is involved, due to the spread from \gls{lnl} II to III. With such a short time-prior, however, the model cannot capture this, and all risk predictions will just yield the prevalence of involvement. This effect can be seen in \cref{fig:unilateral:tprior:multi_length_risk} and shows what has been stated earlier: The support of the time-prior has little effect on the model's predictions, as long as it is sufficient to capture the spread through the lymphatic system.

The theoretical result in \cref{eq:unilateral:tprior:analytical_result} is applicable to the parameter $\tilde{b}_1$, and approximately to $\tilde{b}_2$ since involvement of level II is driven by direct infiltration from the primary tumor rather than transition from level I. For levels III and IV, the theory is not applicable as they have two relevant parent nodes. The solid lines in \cref{fig:unilateral:tprior:rate_decay_sampled} show agreement of the theoretical result with the sampling based training of the full model (circles), where the probabilities $p^{\star}$ were set to 6.1\% and 63.9\%, corresponding to the prevalence of level I and II involvement in the dataset, respectively.

This again shows that, while looking at one T-category only, the time-prior's parameters overdetermine the system. For any choice of $T$, the base and transition probability rates can be adjusted such that the Hidden Markov Model is equivalent to the Bayesian network's performance. Only if we want to distinguish between patients of different T-category the \gls{hmm} can outperform the \gls{bn}.

\begin{figure}
    \centering
    \def\svgwidth{0.7\textwidth}
    \input{figures/multi_length_risk.pdf_tex}
    \caption[A risk prediction for varying lengths of time priors]{Prediction for the risk of involvement in \gls{lnl} III, given that no \gls{lnl} was observed to be involved. Computed by training our \gls{hmm} with 15 different-length uniform time-priors (transparent red) as well as the \gls{bn} model (blue line). The one outlier is the \gls{hmm} with a time-prior covering only one time-step. It is centered on the prevalence of involvement for \gls{lnl} III, regardless of the given diagnose.}
    \label{fig:unilateral:tprior:multi_length_risk}
\end{figure}

\end{document}
