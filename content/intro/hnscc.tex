\providecommand{\relativeRoot}{../..}
\documentclass[\relativeRoot/main.tex]{subfiles}
\graphicspath{
    {\subfix{./figures/}}
}


\begin{document}

\section{Head and Neck Squamous Cell Carcinoma}
\label{sec:intro:hnscc}

All body surfaces and cavities are covered in a type of tissue called \emph{epithelium}. For example, both the skin and the lungs are lined with epithelial cells. Of those cells, three different basic types can be distinguished:
\begin{enumerate*}[label={(\arabic*)}]
    \item \textbf{Squamous} (from \emph{squama}, Latin for ``scale'') cells are flat and thin, while 
    \item \textbf{cuboidal} cells are approximately as thick as they are wide and lastly
    \item \textbf{columnar} epithelial cells are column-like and hence much taller than wide
\end{enumerate*}.
They often have a hexagonal shape when viewed from above (meaning perpendicular to the surface they cover) and are close-packed with little to no intercellular space \cite{marieb_human_1995}.

Malignancies that develop in squamous cells of the epithelium are called \gls{scc}. As many parts of the human body are covered with such cells, there exists a respective variety of types and locations it can arise from. E.g., it is common in the lung, skin or the vagina. But it may also arise in the \emph{mucosa} inside the mouth and upper respiratory tract. In that case, the malignancy is termed \gls{hnscc}.

\gls{hnscc} is the most common type of head and neck cancer, and the sixth most common type of cancer worldwide with an incidence of almost 900,000 new cases in 2018, of which 450,000 resulted in deaths. \cite{johnson_head_2020,ferlay_estimating_2019,bray_global_2018}.

Air pollutants, tobacco and alcohol abuse have been linked to an increased risk for \gls{hnscc} 

\end{document}
