\providecommand{\relativeRoot}{../..}
\documentclass[\relativeRoot/main.tex]{subfiles}
\graphicspath{
    {\subfix{./figures/}}
}


\begin{document}

\section{Management of HNSCC}
\label{sec:intro:management}

The treatment of \gls{hnscc} often follows a multimodal approach employing surgery, \gls{rt} and systemic treatments such as chemotherapy. Using surgery or \acrlong{rt} as a single modality -- targeting the primary tumor and e.g. the sentinel lymph node electively -- is only considered for patients with early T-category (T1 or T2) and a clinically negative neck (N0). In these cases cure rates of 70\% to 90\% can be achieved \cite{pfister_head_2014}. For later T-categories and/or locally advanced disease, larger parts of the lymph system are often resected or irradiated in case of involvement or even electively when the estimated risk of occult disease exceeds a certain threshold, for example 15\% or 20\% \cite{weiss_use_1997,pitman_rationale_2000,pillsbury_iii_rationale_1997}. In \acrlong{rt}, this means one defines an \gls{ctv-n} during treatment planning, which will then be irradiated alongside the GTV-T containing the visible primary tumor mass, visible lymph node metastases (GTV-N) and the primary tumor CTV-T accounting for tumor infiltration of adjacent tissues.

The fundamental problem regarding the elective resection or \gls{ctv-n} definition in head and neck cancer now becomes the accurate estimation of the risk for occult disease in the \glspl{lnl} of a given patient.

For this task, that clinicians face on a daily basis, numerous guidelines have been proposed and consequently established as clinical consensus in large parts of the world \cite{gregoire_ct-based_2003,gregoire_delineation_2014,gregoire_delineation_2018,gregoire_proposal_2006,gregoire_selection_2000,biau_selection_2019,eisbruch_intensity-modulated_2002,ferlito_elective_2009,vorwerk_guidelines_2011,chao_determination_2002}. They describe in great detail which \glspl{lnl} should be included in the elective \gls{ctv-n} depending on the \gls{tnm} classification and primary tumor localization a patient presents with. The basis for those guidelines were mostly studies reporting the prevalence of \gls{lnl} involvement for a given primary tumor location \cite{candela_patterns_1990,shah_patterns_1990,woolgar_histological_1999,woolgar_topography_2007,chao_determination_2002,vauterin_patterns_2006,razfar_incidence_2009,ho_patterns_2012}.

\end{document}
