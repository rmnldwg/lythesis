\providecommand{\relativeRoot}{../..}
\documentclass[\relativeRoot/main.tex]{subfiles}
\graphicspath{
	{\relativeRoot/figures/}
    {\subfix{./figures/}}
}


\begin{document}

\section{Aim of this Work}
\label{sec:intro:aim}

What is missing from these guidelines is a more personalized approach. Current guidelines base their recommendations on the overall prevalences of lymph node involvement in \glspl{lnl}, as seen in large cohorts. This neglects correlations between the frequency of metastases in different levels of the neck. For instance the prevalence of lymph node metastasis is often reported to around 10\% in the ipsilateral \gls{lnl} IV for primary tumors in the oropharynx, and is currently irradiated for almost all patients. However, one expects the probability of occult disease in the ipsilateral \gls{lnl} IV to be low, given a clinically N0 neck. However, in case that levels II and III show visible metastasis on a \gls{ct} scan, the probability for involvement in level IV may be much higher. Estimating this probability could be refined even further by including other parts of a patient's diagnosis: T-category, exact tumor subsite, \gls{hpv} status, and possibly more that cannot be covered in heuristic guidelines.

For the purpose of a more individualized \gls{ctv-n} definition, \citeauthorandlink{pouymayou_bayesian_2019} developed a probabilistic model based on \glspl{bn}. It aims to predict the risk of occult metastasis given any possible combination of the location of visible metastases, thereby providing a method to formulate more detailed guidelines. We will explain this work's core concept separately in \cref{sec:previous_work:bayesian_network}.

The goal of this thesis is to build upon and improve the model by \citeauthorandlink{pouymayou_bayesian_2019}. We take the inspiring idea of modelling the lymphatic metastatic progression in \gls{hnscc}, and extend it to include more patient information and primary tumor characteristics, so that we can eventually predict the risk of occult disease in all \glspl{lnl} for the complete, individual diagnosis new patients may present themselves with.

Employing such a model in clinical practice may ultimately improve the quality of life for \gls{hnscc} patients, as it could aid clinicians in safely reducing the extent of elective nodal treatment and thereby reduce the probability of side effects \cite{batth_practical_2014}.

\end{document}
