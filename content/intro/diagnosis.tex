\providecommand{\relativeRoot}{../..}
\documentclass[\relativeRoot/main.tex]{subfiles}
\graphicspath{
    {\subfix{./figures/}}
}


\begin{document}

\section{Diagnosis and Staging of HNSCC}
\label{sec:intro:diagnosis}

The diagnosis of \glspl{hnscc} are always confirmed by a histopathological examination of a biopsy taken from the primary tumor. When an \gls{scc} is suspected in the head and neck region, tests for an \gls{hpv} infection are often performed. Most commonly, the \gls{hpv} status is established by testing for the cell cycle protein $\text{p16}^\text{INK4a}$, which is not indicative of an \gls{hpv} infection directly. Rather, it is a protein that is over-expressed in cancers caused by the \acrlong{hpv} \cite{johnson_head_2020}.

Most cancers are staged using the \gls{tnm} system, which is currently in its eighth revision \cite{brierley_tnm_2017}. It categorizes a patient's disease in three dimensions:

\begin{enumerate}
    \item[\textbf{T}] (Primary Tumor): Characterizes the size and extend of the primary tumor in the four main categories from T1 through T4. Subcategories, such as T1a/T1b also exist. E.g. in the oropharynx, a tumor larger than 2 cm but smaller than 4 cm is defined as T2. Additional labels exist, such as T0 that indicates that no evidence of a tumor has been found or TX, meaning that the tumor cannot be assessed.
    \item[\textbf{N}] (Regional Lymph Nodes): The labels N1, N2, and N3 categorize an increasing involvement of regional lymph nodes. In the case of an oropharyngeal \gls{scc} that is not \gls{hpv} associated, for example, N1 refers to a single, ipsilateral (meaning on the same side as the primary tumor) metastasis smaller than 3 cm in its greatest extension. However, this categorization gives no information about the location of this involved node. Subdivisions are available for this label, too.
    \item[\textbf{M}] (Distant Metastasis): This describes the absence (M0) or presence (M1) of distant metastasis.
\end{enumerate}

Fundamentally, it is also differentiated between the \emph{clinical} classification (c\gls{tnm}), which is established based on all the available evidence and data before treatment starts, and the \emph{pathological} classification (p\gls{tnm}) that is acquired by factoring in additional information from post-surgical histopathological examinations -- if available.

The extent of regional nodal involvement is mostly assessed using imaging technologies like \gls{ct}, \gls{mri} or \gls{fdg-pet} \cite{johnson_head_2020,van_den_bosch_18f-fdg-petct-based_2020}. However, these imaging modalities have a limited resolution in the order of a millimeter, which is still one or two magnitudes larger than a typical tumor cell whose diameter is in the order of tens of micrometers. This means that nodal metastasis can only be clinically diagnosed indirectly, when the involved lymph node exhibits macroscopic changes due to the presence of a large enough amount of tumor cells. These changes include enlarged nodes (e.g. $\geq$ 1 cm), a central necrotic area or the loss of fatty hilum \cite{pillsbury_iii_rationale_1997,biau_selection_2019,ludwig_dataset_2021}.

In contrast to the clinical N-categorization, the pathological staging uses resected lymph nodes -- if a \acrlong{nd} has been performed -- for examination under a microscope. This means, it can directly detect individual cancer cells. We will discuss this discrepancy between clinical and pathological examinations in a later section (see \cref{sec:future:trinary}).

\end{document}
