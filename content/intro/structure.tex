\providecommand{\relativeRoot}{../..}
\documentclass[\relativeRoot/main.tex]{subfiles}
\graphicspath{
	{\relativeRoot/figures/}
    {\subfix{./figures/}}
}


\begin{document}

\section{Thesis Structure}
\label{sec:intro:structure}

Developing personalized models of lymphatic metastatic progression faces the problem of insufficient data. The data available in publications that constitute the basis for the elective nodal treatment guidelines only quantify the prevalence of involvement for the investigated \glspl{lnl}. They do not provide the reader with the information which levels have been involved together. To the best of our knowledge, the only publication that allowed a reconstruction of per-level involvement information was written by \citeauthorandlink{sanguineti_defining_2009} (see \cref{sec:previous_work:sanguineti} for a brief explanation of their data). To combat this shortcoming, we extracted, curated and published a freely available dataset on a cohort of patients from our institution, the \gls{usz}, that included unprecedented detail w.r.t. nodal involvement and is described in \cref{chap:dataset_usz}.

Following our publication, we collaborated with the influential author of many guidelines on \gls{hnscc} treatment, Vincent Grégoire, who kindly shared with us a second dataset collected at his institution, the \gls{clb}, underlying a publication on the prevalence of cervical lymph node metastases in \gls{opscc} \cite{bauwens_prevalence_2021}. This second dataset is described in some detail in \cref{chap:dataset_clb}.

Because we want to convince the research community of the value of detailed data on lymph node involevement, we also developed an online interface to host, share, and visualize the two datasets from \cref{chap:dataset_usz,chap:dataset_clb}. This interface will be introduced and described in \cref{chap:lyprox}.

After that, we will dive into the statistical modelling part of this thesis, starting from the work of \citeauthorandlink{pouymayou_bayesian_2019}, which is briefly reviewed in \cref{sec:previous_work:bayesian_network}. \Cref{chap:unilateral} will replace the \gls{bn} that work was based on with a \gls{hmm} which is conceptually similar, but adds a temporal component to the representation of lymphatic metastatic progression. This will allow us to incorporate T-categories more naturally into the model, assuming that T-category can be seen as a surrogate for the time a tumor has grown.

Because \cref{chap:unilateral} is based on the publication by \citeauthorandlink{ludwig_hidden_2021}, which was published chronologically before the data of the \cref{chap:dataset_usz,chap:dataset_clb}, it will not actually use the information from the two patient cohorts of \cref{chap:dataset_usz,chap:dataset_clb}. Instead, it will introduce the formalism of the new model in great detail and show that it performs as good as the \gls{bn} model by \citeauthorandlink{pouymayou_bayesian_2019} when applied to the early T-category cohort reconstructed from \citeauthorandlink{sanguineti_defining_2009} (\cref{sec:previous_work:sanguineti}).

In \cref{chap:graph} then, we will make use of the datasets from the \gls{usz} and \gls{clb} to investigate how the \gls{hmm} can be extended to include the additional \glspl{lnl} V and VII into the model. The chapter will introduce methods to effectively assess the performance of multiple models. These methods will then be used to compare models that are built assuming different pathways of spread between the individual, and especially the newly added, \glspl{lnl}.

Afterwards, we will complete the coverage of the model by extending it to the contralateral side of the neck in \cref{chap:bilateral}. We will naturally adapt the formalism to represent a bilateral lymphatic system. Based on symmetry considerations of the head and neck region, we will introduce potential simplifications over a naive approach that simply doubles the model's parameters. Part of our extension to include the contralateral neck will also be how to incorporate an important factor in the prognosis of contralateral nodal metastasis: Whether the primary tumor is lateralized or extends over the mid-sagittal plane. All assumptions and additions will be carefully tested in a probabilistic manner.

Lastly, we combine our findings from the \cref{chap:unilateral,chap:graph,chap:bilateral} to build a complete probabilistic model for lymphatic tumor progression trained on two large datasets in \cref{chap:complete}. We then conclude the thesis with an outlook into the future ideas regarding this project (\cref{chap:extensions}).

\end{document}
