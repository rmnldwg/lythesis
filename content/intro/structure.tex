\providecommand{\relativeRoot}{../..}
\documentclass[\relativeRoot/main.tex]{subfiles}
\graphicspath{
    {\subfix{./figures/}}
}


\begin{document}

\section{Thesis Structure}
\label{sec:intro:structure}

Developing personalized models of lymphatic metastatic progression faces the problem of insufficient data. The data available in publications that constitute the basis for the elective nodal treatment guidelines only quantify the prevalence of involvement for the investigated \glspl{lnl}. They do not provide the reader with the information which levels have been involved together. To the best of our knowledge, the only publication that allowed a reconstruction of per-level involvement information was written by \citeauthorandlink{sanguineti_defining_2009}. To combat this shortcoming, we extracted, curated and published a freely available dataset on a cohort of patients from our institution, the \gls{usz}, that included unprecedented detail w.r.t. nodal involvement and is described in \cref{chap:dataset_usz}.

Following our publication, we collaborated with the influential author of many guidelines on \gls{hnscc} treatment, Vincent Grégoire, who kindly shared with us a second dataset collected at his institution, the \gls{clb}, underlying a publication on the prevalence of cervical lymph node metastases in \gls{opscc} \cite{bauwens_prevalence_2021}. This second dataset is described in some detail in \cref{chap:dataset_clb}.

Because we want to convince the research community of the value of detailed data on lymph node involevement, we also developed an online interface to host, share, and visualize the two datasets from \cref{chap:dataset_usz,chap:dataset_clb}. This interface will be introduced and described in \cref{chap:lyprox}.

After that, we will dive into the statistical modelling part of this work in \cref{chap:unilateral}, starting from the  work of \citeauthorandlink{pouymayou_bayesian_2019}, which is briefly reviewed in \cref{sec:previous_work:bayesian_network}. Because this chapter is based on the publication by \citeauthorandlink{ludwig_hidden_2021}, which was published chronologically before the data of the \cref{chap:dataset_usz,chap:dataset_clb}, it will not actually use the information from the two patient cohorts. Instead, it will introduce the formalism of the new model in great detail and show that it performs as good as the \gls{bn} model by \citeauthorandlink{pouymayou_bayesian_2019} when applied to the only previously available Sanguineti dataset \cite{sanguineti_defining_2009}.

In \cref{chap:graph} then, we finally make use of the new datasets and investigate how to include more \glspl{lnl} into the model. Following this, we complete the coverage of the model by extending it to the contralateral side in \cref{chap:bilateral}.

Lastly, we combine our findings from the \cref{chap:unilateral,chap:graph,chap:bilateral} to build a complete probabilistic model for lymphatic tumor progression trained on two large datasets in \cref{chap:complete}. We then conclude the thesis with an outlook into the future ideas regarding this project (\cref{chap:extensions}).

\end{document}
