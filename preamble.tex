\usepackage[utf8]{inputenc}

%% Figures
\usepackage{graphicx}
\graphicspath{
    {\relativeRoot/figures/}
    {\relativeRoot/content/previous_work/figures/}
    {\relativeRoot/content/dataset_usz/figures/}
    {\relativeRoot/content/dataset_clb/figures/}
    {\relativeRoot/content/unilateral/figures/}
    {\relativeRoot/content/graph/figures/}
    {\relativeRoot/content/bilateral/figures/}
}
\usepackage{rotating}
\usepackage[export]{adjustbox}
\usepackage{wrapfig}
\usepackage{sidecap}

%% Layout
\usepackage[a4paper,top=25mm,bottom=25mm]{geometry}
\usepackage{parskip}
\usepackage{afterpage}
\usepackage{pdflscape}

%% Setting up the headers and footers
\usepackage{fancyhdr}
% delete all headers and footers for the `empty' pagestyle in the front matter
\fancypagestyle{empty}{
    \renewcommand{\headrulewidth}{0pt}
    \fancyhf{}
}
% `plain' is the pagestyle of chapter beginnings
\fancypagestyle{plain}{
    \renewcommand{\headrulewidth}{0pt}
    \fancyhf{}
    \fancyfoot[RE,LO]{\thepage}
}
% the rest of the document will be in the pagestyle `fancy'
\pagestyle{fancy}
\fancyhead{}
\fancyhead[RE]{\rightmark}
\fancyhead[LO]{\leftmark}
\fancyfoot{}
\fancyfoot[RE,LO]{\thepage}

%% For later reference of title, author, etc.
\usepackage{titling}

%% Bibliography
\usepackage[
    style=numeric,
    minnames=3,
    maxnames=4
]{biblatex}
% define custom command to cite the author of somethign and add a link
\newcommand{\citeauthorandlink}[1]{\citeauthor*{#1} \cite{#1}}
\newcommand{\Citeauthorandlink}[1]{\Citeauthor*{#1} \cite{#1}}
\addbibresource{\relativeRoot/references.bib}
\usepackage{csquotes}

%% Using Babel allows other languages to be used and mixed-in easily
%\usepackage[ngerman,english]{babel}
\usepackage[english]{babel}
\selectlanguage{english}

%% For quotes
\usepackage{dirtytalk}
\usepackage{csquotes}

%% Maths
\usepackage{amsmath}
\usepackage{amsfonts}
\usepackage{centernot}
\usepackage{mathtools}
\DeclareMathOperator{\pa}{pa}

\usepackage{nicefrac}
\usepackage{siunitx}
\sisetup{
    group-digits=false
}

\usepackage{ntheorem}
\theoremseparator{:}
\newtheorem{hyp}{Hypothesis}

%% Glossary
\usepackage[acronym]{glossaries}
\makeglossaries
\setacronymstyle{long-short}
\loadglsentries[acronym]{glossary}

%% Cross-referencing
\usepackage{hyperref}
\usepackage{cleveref}
\usepackage{auxhook}

%% Subfiles
\usepackage{subfiles}

%% Tables
\usepackage{multirow}

%% Bullet lists, boxes, admonitions and icons
\usepackage[inline]{enumitem}
\usepackage{tcolorbox}
\usepackage{fontawesome5}

%% Custom commands
\newcommand{\repolink}[2][rmnldwg]{\href{https://github.com/#1/#2}{\faIcon{github} \texttt{#2}}}
\newcommand{\lytag}[1]{\href{https://github.com/rmnldwg/lynference/releases/tag/#1}{\faIcon{tags} \texttt{#1}}}
