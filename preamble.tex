\usepackage[utf8]{inputenc}
\usepackage{graphicx}
\graphicspath{
    {\relativeRoot/figures/}
    {\relativeRoot/content/dataset/figures/}
    {\relativeRoot/content/unilateral/figures/}
}

%% Layout
\usepackage[a4paper,top=25mm,bottom=25mm]{geometry}

%% Setting up the headers and footers
\usepackage{fancyhdr}
% delete all headers and footers for the `empty' pagestyle in the front matter
\fancypagestyle{empty}{
    \renewcommand{\headrulewidth}{0pt}
    \fancyhf{}
}
% `plain' is the pagestyle of chapter beginnings
\fancypagestyle{plain}{
    \renewcommand{\headrulewidth}{0pt}
    \fancyhf{}
    \fancyfoot[RE,LO]{\thepage}
}
% the rest of the document will be in the pagestyle `fancy'
\pagestyle{fancy}
\fancyhead{}
\fancyhead[RE]{\rightmark}
\fancyhead[LO]{\leftmark}
\fancyfoot{}
\fancyfoot[RE,LO]{\thepage}

%% For later reference of title, author, etc.
\usepackage{titling}

%% Bibliography
\usepackage[
    style=numeric
]{biblatex}
\addbibresource{\relativeRoot/references.bib}
\usepackage{csquotes}

%% Using Babel allows other languages to be used and mixed-in easily
%\usepackage[ngerman,english]{babel}
\usepackage[english]{babel}
\selectlanguage{english}

%% Maths
\usepackage{amsmath}
\usepackage{amsfonts}
\usepackage{centernot}
\usepackage{mathtools}
\DeclareMathOperator{\pa}{pa}

\usepackage{nicefrac}

\usepackage{ntheorem}
\theoremseparator{:}
\newtheorem{hyp}{Hypothesis}

%% Glossary
\usepackage[acronym]{glossaries}
\makeglossaries
\setacronymstyle{long-short}
\loadglsentries[acronym]{glossary}

%% Cross-referencing
\usepackage{hyperref}
\usepackage{cleveref}
\usepackage{auxhook}
\newcommand{\repolink}[2][rmnldwg]{\href{https://github.com/#1/#2}{\texttt{#2}}}

%% Subfiles
\usepackage{subfiles}

%% Tables
\usepackage{multirow}

%% Boxes & admonitions
\usepackage{tcolorbox}
